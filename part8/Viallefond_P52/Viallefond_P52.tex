% This is the aspauthor.tex LaTeX file


% Copyright 2010, Astronomical Society of the Pacific Conference Series





\documentclass[11pt,twoside]{article}


%\usepackage{asp2010}



%---------------------------------------------------------------------------
% begin addition from the author (for maths, including diagxy.tex)
\usepackage{amsmath}
\usepackage{amssymb}
\def\mathbi#1{\textbf{\em #1}}
\def\ledom{=\!\!\!|}

\input P52_diagxy

% my own things to associate numbers to these diagrammes, like for equations etc.
\newcounter{diagnum}
\newcommand{\diagramme}[2]{
  \parbox{12.0cm}{
    #2
    \refstepcounter{diagnum}\label{#1}
  }\hfill(\arabic{diagnum})\hfill
}
\newcommand{\diag}[1]{(\ref{#1})}

\newcommand{\sdiagramme}[2]{
  \parbox{6.5cm}{
    #2
    \refstepcounter{diagnum}\label{#1}
  }\hfill(\arabic{diagnum})\hfill
}


% end addition from the author:
%---------------------------------------------------------------------------
\usepackage{asp2010,color}



\definecolor{black}{rgb}{0,0,0}
\definecolor{red}{rgb}{1,0,0}
\definecolor{blue}{rgb}{0,0,1}
\definecolor{white}{rgb}{1,1,1}


%cercle:
\newcommand{\LogicalDiagCircle}{
\node  A(-450,+-431)[]
\node  B(250,269)[]
\node  C(+950,-431)[]
\node  D(250,-1131)[]
\node AB(-244,63)[]
\node BC(744,63)[]
\node CD(744,-925)[]
\node DA(-244,-925)[]
\arrow/@{--}@/^0.525em//[A`AB;]
\arrow/@{--}@/^0.525em//[AB`B;]
\arrow/@{--}@/^0.525em//[B`BC;]
\arrow/@{--}@/^0.525em//[BC`C;]
\arrow/@{--}@/^0.525em//[C`CD;]
\arrow/@{--}@/^0.525em//[CD`D;]
\arrow/@{--}@/^0.525em//[D`DA;]
\arrow/@{--}@/^0.525em//[DA`A;]
}
%extrema to keep the same global frame:
\newcommand{\LogicalDiagExtrema}{
\color{white}
 \place  (+683,-1182)[\boxed{\color{white}\scriptsize _0\rightleftarrows^0}]    %IdXY
 \place  (+683,+0319)[\boxed{\color{white}\scriptsize \rightarrow^{0}}]         %IdX
 \place  (-683,-0431)[\boxed{\color{white}\scriptsize ^{0}\leftharpoondown}]    %IdY
 \place  (+950,-0431)[]                                                         %C
}


\resetcounters


\markboth{Fran\c{c}ois Viallefond}{Formal semantics to model experimental data}

\begin{document}

\title{Formal semantics to model experimental data}
\author{Fran\c{c}ois~Viallefond$^1$
\affil{$^1$LERMA, Observatoire de Paris, 61 av. de l'Observatoire 75014 Paris, France}}

\abstract
The formalization of the MeasurementSet reveals the ubiquity of a 
structure, a simplicial 3-complex. A subset of this complex is 
a hexagon encompassed into a triangle.
The following shows that this hexagon also appears in the structure 
of the grammar of XML-Schema, a language used to express types or structures.
This result gives insights to interpret this ubiquity.
XML is famous for being described by itself (the schema of schema).
Here we are in the case of models with this ubiquous structure 
which are described using a language with a grammar having itself 
that structure.


\section{Introduction}
Given the evolution of the instrumentation and the very large amount
of data that they may produce the role of a data model becomes critical.
An attractive approach to model data is to transform the objects defined
using our human language into mathematical objects this leading
to robust, efficient and expressive data processors
in information systems. Let call the Measurement Set (MS) the model
of the data acquired by telescopes in observatories. This model
is shaped by action and operational semantics. We find that its
overal structure is a simplicial 2-complex, its internal hexagonal pattern,
a borromean structure.
This structure appears to be ubiquitous in this approach.

The implementation of a model requires a modeling language.
During the development of an application, {\it xsd2cpp}, which transforms 
models described using the XML-Schema language into C++ objects we find that 
this structure also organizes concepts defined in the XML-Schema grammar. 
The following presents this result and gives some insights for the interpretation.
It borrouws vocab from the XML-Schema W3C recommandation and the wsdlpull C++
schema processor. 
Due to limited space the detailed and complete study will be given elsewhere.

\section{Component and uniqueness constraints} 
When modeling an object or system, the semantic is captured by setting constraints.
Let consider an object defined by a tuple of elements. There is no element hierarchy. 
Let this tuple composed of parts, an {\it abstract} structure defined using grammar. 
Depending on how these parts are assembled there are two conceptual approaches:\\
{\it 1)} {\bf Type derivation}: this is the object oriented inheritance concept, 
the part being appended sequentially according to a type hierarchy. In type theory 
let the judgment $\Gamma \vdash A\!:\!Type$. The context $\Gamma$ is  a chain of arrows, 
$A_n\rightarrow A_{n-1}\rightarrow A\rightarrow ...\rightarrow A_0$. Constructing $A\!:\!type$ is a typed $\lambda$-calculus
the $\lambda$ terms being the constructors of the $A_i ~\forall i\le n$.
Note that the concept of multiple inheritance does not exist in XML-Schema.\\
{\it 2)} {\bf Composition}. In this case the components are composed according to 
a Component Model characterized by two things: 
{\it a) a model for the containment}: think of a component as a part in a partition. 
A part may be contained in the partition directly. Alternatively the containment is done by reference. 
In other words the definition of the part may be done either locally or at the global level,
{\it b) a model of the component}. In that case there are 3 kinds, these identified by a {\it discriminator}, 
a member of the set $\{Particle,ParticleGroup, Container\}$. 

Note that these two approaches are not exclusive from each other, a technique often used 
being to set cohesion, i.e. structured semantic, by gluing the components using a common 
type hierarchy trunk.

\noindent
\begin{minipage}[h]{5.2cm}
Using the nomenclature of the schema processor, diag. \diag{ComponentModel}
depicts the concept of a component: 
The syntactic terms are Id$_{group}$, the name, abstract, of a group 
defined at the global schema level, and Id$_{elem}$, the name of a particle,
abstract iff substitutable else concrete. The dual is an anonymous 
container defined locally, a set of 
 elements (possibly a singleton),
i.e. some kind of anonymous group or element. Underlying this diagram there are
two inseparable concepts about the property of a component, two booleans: 
``has name?'' and `has multiplicity ?'' (there is multiplicity 
when 
the cardinality of the set of elements that a part may contain may
\end{minipage}
\begin{minipage}[h]{8.8cm}
 \begin{center}
   \sdiagramme{ComponentModel}{
$$\bfig
\node       inxy(-683,-1182)[\color{black}\scriptsize \bullet]                                 %input pairs (x,y)
\node      inxRy(-683,+0319)[\color{black}\scriptsize \bullet]                                 %input relation total at X
\node     outxRy(-683,+0319)[\color{black}\scriptsize \bigcirc]                                %output relation total at X
 
\node      noFct(-183,-0182)[\boxed{\color{red}\scriptsize \text{ParticleGroup}}]              %ParticleGroup a reference
\node surjection(+683,-0682)[\boxed{\color{blue}\scriptsize \text{c}}]                         %c contentModel surjection,    epimorphis
  
\node regularFct(+250,+0067)[\boxed{\color{blue}\scriptsize \text{e}}]                         %e element regularFct,    morphism
\node invertible(+250,-0931)[\boxed{\color{red}\scriptsize \text{Container}}]                  %Container invertible,    Lan?
 
\node        IdX(+683,+0319)[\boxed{\color{black}\scriptsize \text{Id}_\text{elem}}]           % ref element IdX,           Id_x
\node partialFct(+683,-0182)[\boxed{\color{red}\scriptsize \text{Particle}}]                   %Particle partialFct,    coproduct
  
\node  injection(-183,-0682)[\boxed{\color{blue}\scriptsize \text{g}}]                         %e container injection,     monomorphism
\node        IdY(-683,-0431)[\boxed{\color{black}\scriptsize  \text{Id}_\text{group}}]         %g group an identifier Idy
  
\node  bijection(+250,-0431)[\boxed{\color{black}\scriptsize \text{contentModel}}]             %bijection,     isomorphism}
\node       IdXY(+683,-1182)[\boxed{\color{black}\scriptsize \text{\it Id}_\text{\it part}}]   %anonymous sequence IdXY,          Id_{xy}

\LogicalDiagExtrema
  
\color{black}   \arrow|l|/->/[noFct`regularFct;]
\color{black}   \arrow|r|/->/[partialFct`regularFct;*]
 
\color{black}   \arrow|r|/@{->}@<+0.15em>/[partialFct`surjection;]
 
\color{black}   \arrow|l|/->/[noFct`injection;*]
 
\color{black}   \arrow|r|/>/[invertible`surjection;*]
\color{black}   \arrow/>/[invertible`injection;]
 

\color{black}  \arrow|l|/@{->}@<-0.0em>/[partialFct`IdX;\&]
\color{black}  \arrow/@{-->}/[IdX`regularFct;]

\color{black}  \arrow/@{-->}@<-0.00em>/[IdY`injection;]

\color{black}    \arrow|a|/@{->}@<-0.0em>/[noFct`IdY;\&]

\LogicalDiagCircle

\efig$$
}

 \end{center}
\end{minipage}
\hfill
be greater than one). Such a representation for a component is generic. Polymorphic it is fuzzy.
Globally strict, it relies on the definition of this 
{\it discriminator}.


\section{Syntactic term for a uniqueness constraint} 
The {\it xsd2cpp} converter generates regular types. A type is regular when
{\it a)} it is  {\it default constructible} which means that there is an initial object,
{\it b)} it is {\it assignable} i.e. there is a copy constructor and 
{\it c)} it has the concept of  {\it equality comparable} which implies identities and therefore
there is at least one key. Think of a  key as a specific kind of part in a partition, the 
part and partition denoted respectively $Fields$ (the set of key fields) and $Selector$. That pair is 
identified by a syntactic abstract term, the name of the key.
\noindent
\begin{minipage}[h]{5.2cm}
\indent
Any partition has a model of composition of its parts: the {\it compositor} 
parameter, a member of the set $\{Choice, All, Sequence\}$. When a partition is
a $Selector$, if it has more than one part and if there is a single key then
the model of composition has to be augmented by inference rules satisfying the normal 
forms of the relational model, the hypothesis made by  {\it xsd2cpp} when the 
schema processor finds keys.

Diagram (2) depicts the concept of a generic type.
Inside the circle is the {\it internal language} defining the Type.
$Parts$ must be understood as the set of all the parts. Its dual is 
$Partition$, the whole (the set of names {\it vs} the name of the set). 

\end{minipage}
\begin{minipage}[h]{8.8cm}
 \begin{center}
   \sdiagramme{UniquenessModel}{
$$\bfig
\node       inxy(-683,-1182)[\color{black}\scriptsize \bullet]					%input pairs (x,y)
\node      inxRy(-683,+0319)[\color{black}\scriptsize \bullet]					%input relation total at X
\node     outxRy(-683,+0319)[\color{black}\scriptsize \bigcirc]					%output relation total at X

\node      noFct(-183,-0182)[\boxed{\color{red}\scriptsize \text{compositor}}] 			%ParticleGroup a reference
\node surjection(+683,-0682)[\boxed{\color{blue}\scriptsize \text{Fields}}]  			%c contentModel surjection,    epimorphis
 
\node regularFct(+250,+0067)[\boxed{\color{blue}\scriptsize \text{Partition}}]			%e element regularFct,    morphism
\node invertible(+250,-0931)[\boxed{\color{red}\scriptsize \text{Parts}}]			%Container invertible,    Lan?

\node        IdX(+683,+0319)[\boxed{\color{black}\scriptsize \text{Type}}]			% ref element IdX,           Id_x
\node partialFct(+683,-0182)[\boxed{\color{red}\scriptsize \text{Selector}}]			%Particle partialFct,    coproduct
 
\node  injection(-183,-0682)[\boxed{\color{blue}\scriptsize \text{iterator}}]			%e container injection,     mono_{partorphism
\node        IdY(-683,-0431)[\boxed{\color{black}\scriptsize  \text{element}}]		 	%g group an identifier Idy
 
\node       IdXY(+683,-1182)[\boxed{\color{black}\scriptsize \text{Key}}]			%anonymous sequence IdXY,          Id_{xy}

\LogicalDiagExtrema
 
\color{black}	\arrow|l|/->/[noFct`regularFct;]
\color{black}	\arrow|r|/->/[partialFct`regularFct;]



\color{black}	\arrow|r|/@{->}@<+0.0em>/[partialFct`surjection;]

\color{black}	\arrow|l|/->/[noFct`injection;]

\color{black}   \arrow|r|/>/[invertible`surjection;]
\color{black}   \arrow/>/[invertible`injection;]

\color{black}  \arrow|l|/@{->}@<-0.0em>/[partialFct`IdX;]
\color{black}  \arrow/@{-->}/[IdX`regularFct;]

\color{black}  \arrow/@{>}@<-0.00em>/[IdY`injection;]
\color{black}  \arrow/@{->}@<-0.0em>/[IdXY`surjection;]
\color{black}    \arrow/@{->}@<-0.0em>/[noFct`IdY;]
\color{black}   \arrow|b|/@{->}@<-0.0em>/[invertible`IdXY;]

\LogicalDiagCircle
\efig$$
}

 \end{center}
\end{minipage}
\hfill
\noindent
From the parts it is always possible to iterate linearly on them given 
the compositor parameter {\bf or} from the $Fields$ part to fork on 
the complementary parts (a set of accessors.) Dereferencing the state 
of an iterator gives a pair, its first member discriminating the kind of 
parts as shown in diagram \diag{ComponentModel}.

Partition may be seen as a generic container with two constraints
of composition of parts. {\it All:Compositor} tells that the fields of 
the key are independent from each other with no specific order, and 
{\it Selector} which forces to have the structure of a co-product would
the relation build on a functional dependency.

In the inner language the left-right symetry of the hexagon is subsumed by 
the Kan extension (LAN,RAN). 
Thinking in terms of Kan extension there are several parts. A functional relation 
implies at least two parts, the domain and codomain of the functional. It is a co-product. 
Therefore if a relation has a single part, $Fields$, it has to be 
understood as a relation with an abstract imaginary part.

Diagram \diag{UniquenessModel} reveals only one aspect of the concept 
of type. The grammar has two other booleans to tell whether a type is
abstract and whether it is substitutable.\\ 
\noindent
\begin{minipage}[h]{5.2cm}
\indent
Furthermore, when 
substitutable there are two cases: {\it a)} a derivation from an abstract type
either by restriction or by extension or alternatively {\it b)} the substitution is 
the same type as the abstract type, some kind of simultaneous restriction 
and extension. 
Diagram \diag{RealizationModel} depicts this. It also
gives insights about the meaning of the circle in the diagram \diag{UniquenessModel}.
Note that assimilating a context to the left, using symbolically ($\vdash$), 
is \underline{modeling}, using symbolically ($\models$), a rule following the judgment. 
To the right is the exposition of a field of knowledge, dual of the modeling. 
The syntactic term corresponds to the node $semantic$. It is stored in a lexicon,

\end{minipage}
\begin{minipage}[h]{8.8cm}
 \begin{center}
   \sdiagramme{RealizationModel}{
$$\bfig
\node       inxy(-683,-1182)[\color{black}\scriptsize \bullet]				%input pairs (x,y)
\node      inxRy(-683,+0319)[\color{black}\scriptsize \bullet]				%input relation total at X
\node     outxRy(-683,+0319)[\color{black}\scriptsize \bigcirc]				%output relation total at X

\node      noFct(-183,-0182)[\boxed{\color{red}\scriptsize \models}] 			%ParticleGroup a reference
\node surjection(+683,-0682)[\boxed{\color{blue}\scriptsize \dashv}]  			%c contentModel surjection,    epimorphis
 
\node regularFct(+250,+0067)[\boxed{\color{blue}\scriptsize \text{concrete}}]		%e element regularFct,    morphism
\node invertible(+250,-0931)[\boxed{\color{red}\scriptsize \text{\it abstract}}]	%Container invertible,    Lan?

\node        IdX(+683,+0319)[\boxed{\color{black}\scriptsize \text{Type}}]		% ref element IdX,           Id_x
\node partialFct(+683,-0182)[\boxed{\color{red}\scriptsize \ledom}]			%Particle partialFct,    coproduct
 
\node  injection(-183,-0682)[\boxed{\color{blue}\scriptsize \vdash}]			%e container injection,     monomorphism
\node        IdY(-683,-0431)[\boxed{\color{black}\scriptsize  \text{sample}}]		%g group an identifier Idy
 
\node  bijection(+250,-0431)[\boxed{\color{black}\scriptsize \text{\{0,e,s,T=1\}}}]	%bijection,     isomorphism}
\node  bijectionb(+250,-0431)[\boxed{\color{white}\scriptsize \text{\{0,e,s,T=1\}}}]	%bijection,     isomorphism}
\node       IdXY(+683,-1182)[\boxed{\color{black}\scriptsize \text{semantic}}]		%anonymous sequence IdXY,          Id_{xy}

\LogicalDiagExtrema
 
\color{black}	\arrow|l|/->/[noFct`regularFct;]
\color{black}	\arrow|r|/->/[partialFct`regularFct;]

\color{black}	\arrow|r|/@{->}@<+0.15em>/[partialFct`surjection;digest]

\color{black}	\arrow|l|/->/[noFct`injection;assimil]

\color{black}   \arrow|r|/>/[invertible`surjection;restrict]
\color{black}   \arrow|l|/>/[invertible`injection;extend]

\color{black}  \arrow|l|/@{->}@<-0.0em>/[partialFct`IdX;]
\color{black}  \arrow/@{->}/[IdX`regularFct;] 
\color{black}  \arrow/@{->}@<-0.00em>/[IdY`injection;]
\color{black}  \arrow/@{->}@<-0.0em>/[IdXY`surjection;]
\color{black}    \arrow/@{-->}@<-0.0em>/[noFct`IdY;]
\color{black}   \arrow|b|/@{->}@<-0.0em>/[invertible`IdXY;]

\color{black} \arrow|m|/.>/[invertible`regularFct;realization]

\LogicalDiagCircle
 

\efig$$
}

 \end{center}
\end{minipage}
\hfill
\noindent
A type realization is a formation rule, the yield of a judgment. Therefore,
there is an arrow 
$\bfig
\node model(0,50)[\models] 
\node Type(500,50)[\text{\tiny Type}]
\arrow|a|/-->/[model`Type;judgment]
\efig$.
Furthermore we may consider that an event is the arrow 
$\bfig
\node model(0,0)[\models] 
\node ledom(300,0)[\ledom]
\arrow|a|/->/[ledom`model;event]
\efig$ triggered by a new sample (or exception) added to the field of knowledge.
Given this interpretation there is a commutative square with a factorization where the 
judgment arrow is a left lifting, event a co-fibration and the arrow  
$\bfig
\node Type(0,0)[\text{Type}]
\node concrete(400,0)[\text{concrete}] 
\arrow|a|/->/[Type`concrete;]
\efig$
a fibration (one of the axioms of the Model category, see also \cite{awodey_2009}).
In an evolutionary system we have the arrow 
$\bfig
\node extend(0,0)[\vdash]
\node restrict(450,0)[\dashv] 
\arrow|a|/->/[extend`restrict;evolution]
\efig$
and in this case we denote {\it transgression} the lifting 
$\bfig
\node semantic(0,50)[\text{\tiny semantic}] 
\node extend(600,50)[\vdash]
\arrow|a|/-->/[semantic`extend;transgression]
\efig$.  
We conclude that raising a type is a judgment in an evolution the transgression
arrow which has for domain the evolution arrow and co-domain the judgment arrow.
Diagram \diag{RealizationModel} is very instructive. For example, using the bra-ket notation,
we have \begin{displaymath}
<\!\text{Type}~|~\text{\color{blue}concrete}\!
><\!\text{\color{red}model}~|~\text{sample}\!>~ =~ 
<\!\text{Type,{\color{red}model}}~|~\text{{\color{blue}concrete},sample}\!>
\end{displaymath}
The experimentation is by essence the ket $|\text{\color{blue}concrete},\text{sample}>$. Therefore
the duality 
\begin{displaymath}
<\!\text{judgment}~|~\text{experimentation}\!>.
\end{displaymath}
which subsumes the title "{\it Formal semantic to model experimental data}".

\section{The Borromean logic}
{\bf Definition} (\cite{guitart_2012}): A reduced borromean object in the category $\mathcal{C}$ with null morphisms,
a terminal and initial object, cokernels and finite sums, finite coproducts, is an object 
$\mathbi{B}$ equipped with three objects $\mathbi{R}$,$\mathbi{S}$,$\mathbi{T}$ and an epimorphic 
family of monomorphisms in  $\mathcal{C}$,
\begin{displaymath}
	r\!:\!\mathbi{R}\rightarrow\mathbi{B},\; s\!:\!\mathbi{S}\rightarrow\mathbi{B},\; t\!:\!\mathbi{T}\rightarrow\mathbi{B} 
\end{displaymath}
such that 
$\mathbi{B}/r\simeq 1$,  
$\mathbi{B}/s\simeq 1$,
$\mathbi{B}/t\simeq 1$. In the category of finite boolean algebras this notion is equivalent to a tripartition of a set.
Let 
exception=$\{e\}$ for R, 
semantic=$\{s\}$ for S,
0 the initial object and $T=1$ the constructed terminal object.

Folding the three outer triangles of the diagram \diag{RealizationModel} to join in the inner part 
leads to\\
\noindent
\diagramme{BorromeanModel}{
$$\bfig
\node       inxy(-683,-1182)[\color{black}\scriptsize \bullet]				%input pairs (x,y)
\node      inxRy(-683,+0319)[\color{black}\scriptsize \bullet]						%input relation total at X
\node     outxRy(-683,+0319)[\color{black}\scriptsize \bigcirc]				%output relation total at X

\node      noFct(-183,-0182)[\boxed{\color{red}\scriptsize \models}] 			%ParticleGroup a reference
\node surjection(+683,-0682)[\boxed{\color{blue}\scriptsize \dashv}]  			%c contentModel surjection,    epimorphis
 
\node regularFct(+250,+0067)[\boxed{\color{blue}\scriptsize \text{concrete}}]		%e element regularFct,    morphism
\node invertible(+250,-0931)[\boxed{\color{red}\scriptsize \text{\it abstract}}]	%Container invertible,    Lan?

\node partialFct(+683,-0182)[\boxed{\color{red}\scriptsize \ledom}]			%Particle partialFct,    coproduct
 
\node  injection(-183,-0682)[\boxed{\color{blue}\scriptsize \vdash}]			%e container injection,     monomorphism

\node        IdY(-683,-0431)[\boxed{\color{black}\scriptsize  \text{sample}}]		%g group an identifier Idy
 
\node  bijection(+250,-0431)[\boxed{\color{black}\scriptsize \text{\{0,e,s,T=1\}}}]	%bijection,     isomorphism}
\node  bijectionb(+250,-0431)[\boxed{\color{white}\scriptsize \text{\{0,e,s,T=1\}}}]	%bijection,     isomorphism}


\LogicalDiagExtrema
 
\color{black}	\arrow|l|/->/[noFct`regularFct;]
\color{black}	\arrow|r|/->/[partialFct`regularFct;]

\color{black}	\arrow|r|/@{->}@<+0.15em>/[partialFct`surjection;digest\implies\text{\bf Partial}]

\color{black}	\arrow|l|/->/[noFct`injection;\text{\bf Complete}~\Longleftarrow~ assimil]

\color{black}   \arrow|r|/>/[invertible`surjection;restrict]
\color{black}   \arrow|l|/>/[invertible`injection;extend]

\color{black} \arrow/->/[noFct`bijection;]
\color{black} \arrow/->/[bijection`surjection;]

\color{black} \arrow/->/[partialFct`bijectionb;]
\color{black} \arrow/->/[bijectionb`injection;]

\color{black} \arrow/->/[invertible`bijection;]
\color{black} \arrow/->/[bijection`regularFct;]
\efig$$
}
which reveals three notions:\\
{\it 1)} {\bf conceptualisation}: $Type/abstract~ =~ concrete$ which is a {\it realization}\\
{\it 2)} {\bf analysis}: $Type/model~ =~ \dashv$ which is a {\it retrospection} and\\
{\it 3)} {\bf learning}: $Type/digestion~ =~ \vdash$ which is a {\it prospection}.\\
the retrospection and prospection being the corrolaries of the realization. 
This diagram \diag{BorromeanModel} highlights also the duality between two concepts,
the pair abstraction-completude. Completude induces closure, the meaning of this circle
in the diagrams.

\section{Conclusion}
The theory of the MeasuremetSet (work in progress) reveals the ubiquity of a structure, 
a triangle encompassing a hexagon. Results presented here shows this ubiquity  now at the level 
of the XML-Schema grammar defining concepts. This ubiquity is bound to the  modeling activity
to express the meaning of things, a conceptualization. Semantics is introduced by developping
structures which have symetries, these introducing relations carriers of meaning.

\bibliographystyle{asp2010}
\bibliography{P52}
\acknowledgements {\small I am grateful to Vivek Krishna, author of the C++ schema parser
available at \url{http://wsdlpull.sourceforge.net}.  

\end{document}
