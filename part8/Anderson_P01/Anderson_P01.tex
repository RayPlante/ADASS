
\resetcounters

\bibliographystyle{asp2010}

\markboth{Anderson, Rosolowsky, and Dowler}{CyberSKA}


\title{CyberSKA Radio Imaging Metadata and VO Compliance Engineering}
\author{K.~R.~Anderson,$^1$ E.~Rosolowsky,$^1$ and P.~Dowler$^2$
\affil{$^1$University of British Columbia, Okanagan, 3333 University Way
Kelowna, BC}
\affil{$^2$National Research Council Canada, 1200 Montreal Road
Building M-58, Ottawa, Ontario}}

\aindex{Anderson, K. R.}
\aindex{Rosolowsky, E.}
\aindex{Dowler, P.}

\begin{abstract}
The CyberSKA project have written a specification for the metadata
encapsulation of radio astronomy data products pursuant to insertion
into the VO-compliant Common Archive Observation Model (CAOM) database
hosted by the Canadian Astronomy Data Centre (CADC). This
specification accommodates radio FITS Image and UV Visibility data, as
well as pure CASA\ooindex{CASA, ascl:1107.013} Tables Imaging and Visibility Measurement Sets. To
extract and engineer radio metadata, we have authored two software
packages: metaData (v0.5.0) and mddb (v1.3). Together, these python
packages can convert all the above stated data format types into
concise FITS-like header files, engineer the metadata to conform to
the CAOM data model, and then insert these engineered data into the
CADC database, which subsequently becomes published through the
Canadian Virtual Observatory. The metaData and mddb packages have, for
the first time, published ALMA imaging data on VO services. Our
ongoing work aims to integrate visibility data from ALMA and the SKA
into VO services and to enable user-submitted radio data to move
seamlessly into the Virtual Observatory. 
\end{abstract}

\section{Introduction}
The CyberSKA project  (\url{http://cyberska.org/}) is
developing a Data Query Service (DQS) for users of the web portal. It
was specified that this service should be compliant with Virtual
Observatory (VO) protocols in order that CyberSKA datasets are exposed
through VO services. In a \textit{pro tem} ``proof of concept'' effort,
CyberSKA project collaborated with CADC in order to leverage extant VO
services already available at the Centre.

To this end, the authors have developed  specifications and software
that both encapsulate the necessary metadata associated with datasets
uploaded to the  CyberSKA Data Management Service (DMS), and then
engineer that metadata to conform to the CAOM data
model \citep{dowler_2007}. The engineered data are then inserted into
the CAOM database. Once in the CAOM database, the data then become
available through various VO services, such as the Data Discovery Tool. 

\section{The Software}
Pursuant to the goals outlined above, two (2) software packages have
been developed to date: metaData, v0.5.0 and mddb, v1.3. As the name
suggests, the metaData package is responsible for encapsulating
the metadata of a variety of data formats and types.\footnote{These
  are data ``types'' in a meta sense, such as ``imaging'' and ``visibility.''}
Metadata are extracted and encapsulated only and explicitly for FITS
images, UVFITS data, Casa\ooindex{CASA, ascl:1107.013} Tables images, and Casa Tables Visibility
Measurement Sets. The mddb package is then responsible for configuring
thusly encapsulated metadata for CAOM database insertion.

\subsection{metaData, v0.5.0}
The metaData package \citep{and_2011} is arranged in an hierarchy of ``main'' modules
that provide the classes for determining the ``MIME type'' of the
previously mentioned data types. Though not strictly MIME, these types
have been specified by the CyberSKA group in order to explicitly
identify a dataset type. The metaData mime-typing facility can
identify the previously mentioned data formats and
types,\footnote{Future implementations of metaData MimeTyping and
  Handler classes could potentially accommodate future adopted formats
  such as HDF5.} and tag a CyberSKA dataset as one of the following:
\begin{itemize}
\item ``image/fits-image''
\item ``image/fits-uvw''
\item ``image/ms-image''
\item ``image/ms-uvw''
\end{itemize}
Current metaData MimeTyping and Handlers classes are outlined in Table \ref{tab:mods}.
\begin{table}[htbp]
  \centering
  \begin{tabular}{|l|p{9.0cm}|} 
    \hline
    \sc \textbf{Module} & \textbf{Classes} \\
    \hline
    fitsMimeTyping & FITSMimeTyping, FITSMimeTypeError(TypeError) \\
    msMimeTyping   & MSMimeTyping,   MimetypeError(TypeError) \\
    msHandlers     & MSHandlers,     MSTableValueError(AttributeError) \\
    fitsHandlers   & FitsHandlers \\
    casaImageHandlers  & CasaImageHandlers, CasaImageTypeError(TypeError)\\
    \hline
  \end{tabular}
  \caption{metaData Module and Class relationship}
  \label{tab:mods}
\end{table}

A set of support modules provide these classes with generalised
functional utility, shown in Table \ref{tab:subdirs}.
\begin{table}[htbp]
  \centering
  \begin{tabular}{|l|p{9.0cm}|} 
    \hline
    \sc \textbf{Support Directory} & \textbf{Modules} \\
    \hline
    convert/ & polarizationConversions, mjdConversions, timeConversions,
               directionConversions, frequencyConversions \\
    incl/   &  imageInclusion, tablesInclusion \\
    utils/  &  genUtils, runUtils \\
    \hline
    docs/   &  *.html\\
    \hline
  \end{tabular}
  \caption{metaData subdirectories and support modules}
  \label{tab:subdirs}
\end{table}

The modules found in the convert/ and incl/ subdirectories provide a very
easy and convenient way for users to reorder table selection and output, include, or not, 
various table fields, and to reorder those fields in the header file by simply rearranging
the various parameters as needed or desired.

\subsection{mddb, v1.3 (current)}
\label{sec:mddb}
The metadata database (mddb) package is structured in much the same
way as the above described metaData package in that it is a set of classes
organised around the four basic ``MIME types'' that CyberSKA currently
supports in the DQS.

There are five (5) main classes, a base class (BaseDbInsert)
providing the public interface, and four (4) classes that are
sub-classed off the base class and that implement the
necessary data engineering methods appropriate to the particular
``MIME type'' for which the respective classes have been
developed. These relationships are shown in Table \ref{tab:mddbmods}.

\begin{table}[htbp]
  \centering
  \begin{tabular}{|l|p{8.0cm}|} 
    \hline
    \sc \textbf{Module} & \textbf{Classes} \\
    \hline
    baseDbInsert & BaseDbInsert \\
    dbFitsInsert   & DbFitsInsert(BaseDbInsert) \\
    dbUVFitsInsert& DbUVFitsInsert(BaseDbInsert)\\
    dbCImageInsert& DbCImageInsert(BaseDbInsert)\\
    dbMSInsert & DbMSInsert(BaseDbInsert)\\
    \hline
  \end{tabular}
  \caption{mddb, v1.3, Module and Class relationship}
  \label{tab:mddbmods}
\end{table}

A set of support modules provide these classes with generalised
functional utility, as well as data configuration specifications, a
database environment class, and setup test scripts, and shown in Table \ref{tab:mddbsubs}.

\begin{table}[htbp]
  \centering
  \begin{tabular}{|l|p{8.0cm}|} 
    \hline
    \sc \textbf{Support Directory} & \textbf{Modules} \\
    \hline
    config/ &  mddb.cfg, mddbEnv \\
    convert/ & converters, keymaps, mjdConversions \\
    db/ &  confirmKeys, xDBKeys \\
    scripts/ &  publicUris, testrun.py \\
    utils/  & f2cArgs, rUtils \\
    \hline
    docs/   &  *.html\\
    \hline
 \end{tabular}
  \caption{mddb, v1.3, Conversion, configuration, and database support}
  \label{tab:mddbsubs}
\end{table}

\section{Engineering and Workflow}
An indicated in \S \ref{sec:mddb}, data engineering is implemented by
sub-classes that are specific to each dataset type. A good deal of
engineering is carried out in these private methods, activities that
primarily comprise evaluating and, if necessary, mitigating
deprecated, errant, and/or missing metadata. Examples include converting
\texttt{{\small CDELT}}\textit{i} and \texttt{{\small CROTA}}\textit{i} keywords to a
\texttt{{\small CD}}\textit{i\_j} matrix as required by the CAOM data model;
ensuring \texttt{{\small RADESYS}} and \texttt{{\small EQUINOX}} are specified and
consistent; converting a Casa\ooindex{CASA, ascl:1107.013} Image geodetic ``telescope position'' in
latitude, longitude and height to geocentric measures specified
by \texttt{{\small OBSGEO-X}}, \texttt{{\small OBSGEO-Y}}, \texttt{{\small OBSGEO-Z}}. The
CyberSKA project has noted that user supplied metadata can vary widely
in quality and much of the data engineering so far implemented is
designed to mitigate these variations to ensure compatibility with the
CAOM data model.

Combined in serial fashion, the metaData and mddb packages can
automatically process a given astronomical dataset and make it
available through VO services. Since these packages process only
metadata, this workflow is fast and efficient, making data VO
accessible in a matter of seconds, where much of that time is consumed
by database interactions. Because metaData creates a textual header
file, which is FITS-like in style, the mddb package does not require
access to the original dataset, but only works directly from this
header file. This feature has proved particularly useful in prototype development.

\section{Conclusion}
When used in conjunction, the two packages, metaData and mddb, form a
workflow that is capable of receiving a dataset file in one of several
formats (FITS image, UVFITS, Casa\ooindex{CASA, ascl:1107.013} Image, Visibility Measurement Set),
extract and encapsulate the dataset's metadata, write an adjudicated
header file, configure those adjudicated metadata to a form appropriate to the
CAOM data model, and insert those data into a selected CAOM
database. When the database is part of a VO services platform, those
datasets thence become registered and published through the Canadian
Virtual Observatory and the larger Virtual Astronomical Observatory
network. The metaData and mddb packages have, for the first time,
published ALMA imaging data on VO services. Future work aims to
integrate visibility data from ALMA and the SKA into VO services and
to enable user-submitted radio data to move seamlessly into the
Virtual Observatory. 

\acknowledgements This research is funded by a grant from CANARIE to
the CyberSKA collaboration and a North American ALMA Science Center
ALMA Development Study. This research used the facilities of the CADC
operated by the National Research Council of Canada with the support
of the Canadian Space Agency.

\bibliography{editor}
