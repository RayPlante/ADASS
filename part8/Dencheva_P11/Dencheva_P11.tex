% This is the aspauthor.tex LaTeX file
% Copyright 2010, Astronomical Society of the Pacific Conference Series

\documentclass[11pt,twoside]{article}
\usepackage{asp2010}

\resetcounters

\bibliographystyle{asp2010}

\markboth{Dencheva, Hack, and Fruchter}{Headerlets}

\begin{document}

\title{Headerlets: Share HST Astrometric Solutions Without The Data}
\author{Nadia~Dencheva, Warren~Hack, and Andrew~Fruchter}
\affil{Space Telescope Science Institutee, 3700 San Martin Dr., Baltimore, MD 21210}

\begin{abstract}
File format for storing astrometric metadata of images is presented. 
Software implementation of the format and methods for working with 
it are described. Possible applications and availability within 
the HST archive are discussed.
\end{abstract}

\section{Motivation}
The original motivation for this work was a WCS based replacement
of Multidrizzle, now released as Astrodrizzle, and specifically a
requirement for the availability and management of multiple WCS
sets, complete with distortion,
within one science file. However, the concept of encapsulating
astrometric solutions is more general than that since each solution
may represent a different catalog. Furthermore, computing accurate
astrometric solutions requires considerable effort and time so having
a way to distribute them, apply them to a science observation and
switch between different WCSs efficiently would facilitate many
aspects of data analysis.

\subsection{Headerlets}
A headerlet is a self-consistent representation of a single WCS,
including all distortion, for all chips/detectors of a single
exposure. This is different from alternate WCS defined in WCS
Paper I \citep{greisen_2002} in that by definition
all alternate WCSs share the same distortion model while headerlets
may be based on different distortion models. A headerlet does not 
include alternate WCSs. It has no observational data
which makes it relatively small and light to distribute. 
It is stored as a multi-extension FITS file
following the structure of the science file it was derived from.
The WCS informationin the science header is stored in the header of 
an HDU with EXTNAME 'SIPWCS'. All other HDUs in the headerlet 
(containing distortion information), have the same EXTNAME as the 
corresponding HDUs in the science file.

\subsection{SIPWCS - A New FITS Extension Type}
We introduce a new HDU with extname 'SIPWCS'. It has no data and contains 
all the WCS related keywords from a science header.
As a minimum all basic WCS keywords \citep{greisen_2002} are included. SIP
convention keywords \citep{shupe_2005} and distortion Lookup table keywords, 
if present in the science file, are included as well. A headerlet has as
many SIPWCS extensions as there are data extensions in the science
file it was derived from. 

\subsection{Headerlet File Format}
As a minimum a headerlet has a Primary header
 and a SIPWCS extension. It may have one D2IMARR HDU and a number of
WCSDVARR HDUs. The Primary header contains keywords which define a headerlet as
unique, as well as keywords which provide a way to ensure the headerlet is 
applied to the correct science file. Detailed definition of all keywords is given in two STScI software report \citep{hack_fc} and \citep{hack_hlet}.

An example of the file structure of an HST ACS/WFC science
file and a headerlet derived from it is shown in Figure 1.
The SIPWCS extension along with all WCSDVARR and the D2IMARR extensions fully describe the WCS of each chip.

\begin{figure}[!h]
\plotone{P11_f1.eps}
\caption{File structure of an ACS/WFC science file (160 MB) and the corresponding headerlet (100 KB); Colors: green - headerlet; blue - WCS for chip 1; red - WCS for chip 2}
\end{figure}

Note that a headerlet derived from a WFC3/UVIS image would only contain a 
Primary header and two SIPWCS extensions (one for each SCI extension) as
WFC3/UVIS does not currently have non-polynomial distortion or detector
defect correction.

\section{Working With Headerlets}
Headerlets are implemented in a python module, headerlet.py, part of STWCS. The functionality includes methods to:

\begin{itemize}
\item Create a headerlet
\begin{itemize}
  \item from the Primary WCS
  \item or from an alternate WCS
\end{itemize}
\item Apply a headerlet to a science file
\begin{itemize}
  \item as Primary WCS
  \item or alternate WCS
\end{itemize}
\item Attach a headerlet to a science file
\begin{itemize}
  \item the entire FITS file is attached as a NonstandardHDU
  \item transparent to FITS readers
  \item but PyFITS provides full access to it
\end{itemize}
\item Extract a headerlet extension form a science file
\begin{itemize}
  \item PyFITS is used to read a headerlet HDU (MEF FITS file by itself) and save it to disk
\end{itemize}
\item Restore a WCS from an attached headerlet HDU
\begin{itemize}
  \item replaces the Primary WCS
\end{itemize}
\end{itemize}

An optional GUI interface is available through TEAL.
The complete API and usage examples are available from STSCI's documentaion pages.

\section{Headerlet HDU - A New Type of FITS Extension}
The word `headerlet` has been used sofar in three different ways:

\begin{itemize}
\item A single WCS representation
\item The multiextension FITS file storing a WCS
\item The extension of a science file containing a headerlet (as a WCS representation)
\end{itemize}

The last usage of the term `headerlet` is discussed in this section.
When a headerlet is applied to an image, a copy of the original 
headerlet file is appended to the image's HDU list as a special extension
HDU called a `HeaderletHDU`. A HeaderletHDU consists of a simple header
describing the headerlet, and has as its data the headerlet file itself,
(which may be compressed). A HeaderletHDU has an 'XTENSION' 
value of 'HDRLET'. Support for this is provided through the implementation
of a NonstandardExtHDU in PyFITS. If stwcs.wcsutil.headerlet is imported,
PyFITS will recognize these extension as HeaderletHDU.

HeaderletHDU objects are similar to other HDU objects in PyFITS.
However, they have a special `.headerlet` attribute that returns
the actual headerlet contained in the HDU data as a `Headerlet` object.

\section{Summary}
We introduce the concept of a headerlet, present its file format
and describe its implementation. Headerlets are not HST specific.
The FITS WCS standard and all WCS conventions implemented in PyWCS
are supported. They are already used within the DrizzlePac package. 
The HST archive is in the process of defining procedures and 
implementation details to support archiving HST headerlets and making
them available to the community through a searchable tool.

\acknowledgements The support for saving FITS files as extensions to other FITS files in PyFITS was implemented by Erik Bray.

\bibliography{P11} 

\footnote{STScI technical reports are located at \url{http://stsdas.stsci.edu/tsr}}

\end{document}
