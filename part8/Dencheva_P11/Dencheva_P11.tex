
\resetcounters

\bibliographystyle{asp2010}

\markboth{Dencheva, Hack, and Fruchter}{Headerlets}

\title{Headerlets: Share \ssindex{observatories!space-based!HST}HST Astrometric Solutions Without The Data}
\author{Nadia~Dencheva, Warren~Hack, and Andrew~Fruchter
\affil{Space Telescope Science Institute, 3700 San Martin Dr., Baltimore, MD 21210}}

\aindex{Dencheva, N.}
\aindex{Hack, W.}
\aindex{Fruchter, A.}

\begin{abstract}
A file format for storing astrometric \ssindex{data!metadata}metadata of images is presented. A software implementation of the format and methods for working with it are described. Possible applications and availability within the \ssindex{archives!individual!HST Archive}\ssindex{observatories!space-based!HST}HST archive are discussed.
\end{abstract}

\section{Motivation}
The original motivation for this work was a \ssindex{World Coordinate System (WCS)}WCS based replacement of Multidrizzle, now released as Astrodrizzle, and specifically a requirement for the availability and management of multiple \ssindex{World Coordinate System (WCS)}WCS sets, complete with distortion, within one science file. However, the concept of encapsulating astrometric solutions is more general than that since each solution may represent a different catalog. Furthermore, computing accurate astrometric solutions requires considerable effort and time so having a way to distribute them, apply them to a science observation, and switch between different \ssindex{World Coordinate System (WCS)}WCSs efficiently would facilitate many aspects of data analysis.

\subsection{Headerlets}
A headerlet is a self-consistent representation of a single \ssindex{World Coordinate System (WCS)}WCS, including all distortions, for all chips/detectors of a single \ssindex{observing!exposure}exposure. This is different from alternate \ssindex{World Coordinate System (WCS)}WCS defined in \ssindex{World Coordinate System (WCS)}WCS Paper I \citep{greisen_2002} in that by definition all alternate \ssindex{World Coordinate System (WCS)}WCSs share the same distortion model while headerlets may be based on different distortion models. A headerlet does not include alternate \ssindex{World Coordinate System (WCS)}WCSs. It has no observational data which makes it relatively small and light to distribute. It is stored as a multi-extension \ssindex{data formats!FITS}FITS file following the structure of the science file it was derived from. The \ssindex{World Coordinate System (WCS)}WCS informationin the science header is stored in the header of an HDU with EXTNAME 'SIPWCS'. All other HDUs in the headerlet (containing distortion information), have the same EXTNAME as the corresponding HDUs in the science file.

\subsection{SIPWCS - A New \ssindex{data formats!FITS}FITS Extension Type}
We introduce a new HDU with extname 'SIPWCS'. It has no data and contains all the \ssindex{World Coordinate System (WCS)}WCS related keywords from a science header. As a minimum all basic \ssindex{World Coordinate System (WCS)}WCS keywords \citep{greisen_2002} are included. SIP convention keywords \citep{shupe_2005} and distortion Lookup table keywords, if present in the science file, are included as well. A headerlet has as many SIPWCS extensions as there are data extensions in the science file it was derived from. 

\subsection{Headerlet File Format}
As a minimum a headerlet has a Primary header  and a SIPWCS extension. It may have one D2IMARR HDU and a number of WCSDVARR HDUs. The Primary header contains keywords which define a headerlet as unique, as well as keywords which provide a way to ensure the headerlet is applied to the correct science file. Detailed definition of all keywords is given in two STScI software reports \citet{hack_fc} and \citet{hack_hlet}\footnote{STScI technical reports are located at \url{http://stsdas.stsci.edu/tsr}}.

An example of the file structure of an \ssindex{observatories!space-based!HST}HST \ssindex{instruments!individual!ACS}ACS/\ssindex{instruments!individual!WFC3}WFC science file and a headerlet derived from it is shown in Figure 1. The SIPWCS extension along with all WCSDVARR and the D2IMARR extensions fully describe the \ssindex{World Coordinate System (WCS)}WCS of each chip.

\begin{figure}[!h]
\plotone{part8/Dencheva_P11/P11_f1.eps}
\caption{File structure of an \ssindex{instruments!individual!ACS}ACS/\ssindex{instruments!individual!WFC3}WFC science file (160 MB) and the corresponding headerlet (100 KB); Colors: green - headerlet; blue - \ssindex{World Coordinate System (WCS)}WCS for chip 1; red - \ssindex{World Coordinate System (WCS)}WCS for chip 2}
\end{figure}

Note that a headerlet derived from a \ssindex{instruments!individual!WFC3}WFC3/UVIS image would only contain a Primary header and two SIPWCS extensions (one for each SCI extension) as \ssindex{instruments!individual!WFC3}WFC3/UVIS does not currently have non-polynomial distortion or detector defect correction.

\section{Working With Headerlets}
Headerlets are implemented in a \ssindex{computer languages!Python}python module, headerlet.py, part of \ssindex{libraries!STWCS}\ssindex{packages!STWCS}STWCS. The functionality includes methods to:

\begin{itemize}
\item Create a headerlet
\begin{itemize}
  \item from the Primary \ssindex{World Coordinate System (WCS)}WCS
  \item or from an alternate \ssindex{World Coordinate System (WCS)}WCS
\end{itemize}
\item Apply a headerlet to a science file
\begin{itemize}
  \item as Primary \ssindex{World Coordinate System (WCS)}WCS
  \item or alternate \ssindex{World Coordinate System (WCS)}WCS
\end{itemize}
\item Attach a headerlet to a science file
\begin{itemize}
  \item the entire \ssindex{data formats!FITS}FITS file is attached as a NonstandardHDU
  \item transparent to FITS readers
  \item but \ssindex{libraries!PyFITS}PyFITS\ooindex{PyFITS, ascl:1207.009} provides full access to it
\end{itemize}
\item Extract a headerlet extension form a science file
\begin{itemize}
  \item \ssindex{libraries!PyFITS}PyFITS\ooindex{PyFITS, ascl:1207.009} is used to read a headerlet HDU (MEF \ssindex{data formats!FITS}FITS file by itself) and save it to disk
\end{itemize}
\item Restore a \ssindex{World Coordinate System (WCS)}WCS from an attached headerlet HDU
\begin{itemize}
  \item replaces the Primary \ssindex{World Coordinate System (WCS)}WCS
\end{itemize}
\end{itemize}

An optional \ssindex{software!user interfaces}GUI interface is available through TEAL. The complete API and usage examples are available from STScI's documentation pages.

\section{Headerlet HDU - A New Type of \ssindex{data formats!FITS}FITS Extension}
The word `headerlet` has been used so far in three different ways:

\begin{itemize}
\item A single \ssindex{World Coordinate System (WCS)}WCS representation
\item The multiextension \ssindex{data formats!FITS}FITS file storing a \ssindex{World Coordinate System (WCS)}WCS
\item The extension of a science file containing a headerlet (as a \ssindex{World Coordinate System (WCS)}WCS representation)
\end{itemize}

The last usage of the term `headerlet` is discussed in this section. When a headerlet is applied to an image, a copy of the original headerlet file is appended to the image's HDU list as a special extension HDU called a `HeaderletHDU`. A HeaderletHDU consists of a simple header describing the headerlet, and has as its data the headerlet file itself, (which may be compressed). A HeaderletHDU has an 'XTENSION' value of 'HDRLET'. Support for this is provided through the implementation of a NonstandardExtHDU in \ssindex{libraries!PyFITS}PyFITS\ooindex{PyFITS, ascl:1207.009}. If stwcs.wcsutil.headerlet is imported, \ssindex{libraries!PyFITS}PyFITS\ooindex{PyFITS, ascl:1207.009} will recognize these extension as HeaderletHDU.

HeaderletHDU objects are similar to other HDU objects in \ssindex{libraries!PyFITS}PyFITS\ooindex{PyFITS, ascl:1207.009}. However, they have a special `.headerlet` attribute that returns the actual headerlet contained in the HDU data as a `Headerlet` object.

\section{Summary}
We introduce the concept of a headerlet, present its file format and describe its implementation. Headerlets are not \ssindex{observatories!space-based!HST}HST specific. The \ssindex{data formats!FITS}FITS \ssindex{World Coordinate System (WCS)}WCS standard and all \ssindex{World Coordinate System (WCS)}WCS conventions implemented in \ssindex{libraries!PyWCS}PyWCS are supported. They are already used within the \ssindex{packages!DrizzlePac}DrizzlePac package. The \ssindex{archives!individual!HST Archive}\ssindex{observatories!space-based!HST}HST archive is in the process of defining procedures and implementation details to support archiving \ssindex{observatories!space-based!HST}HST headerlets and making them available to the community through a searchable tool.

\acknowledgements The support for saving \ssindex{data formats!FITS}FITS files as extensions to other \ssindex{data formats!FITS}FITS files in \ssindex{libraries!PyFITS}PyFITS\ooindex{PyFITS, ascl:1207.009} was implemented by Erik Bray.


\bibliography{editor} 
