\documentclass[11pt,twoside]{article}
\usepackage{graphics}
\usepackage{asp2010}

\resetcounters

\bibliographystyle{asp2010}
\markboth{}{}

\begin{document}

\title{Delivering astronomy software with minimal user maintenance}

\author{S\'ebastien Fabbro$^1$ and Sharon Goliath$^2$
\affil{$^1$University of Victoria, Dept of Physics \& Astronomy P.O. Box 3055, Victoria, BC, Canada}
\affil{$^2$Canadian Astronomy Data Centre, 5071 W. Saanich Rd. Victoria, BC, Canada}}

\begin{abstract}
  We present an approach to deliver astronomy processing software using
  virtualization and a network file system. User-requested astronomy
  software applications are built and tested on a dedicated server, and
  distributed on-demand to cloud-based worker clients using a fast HTTP
  read-only cache file system. The worker clients are light virtual
  machines which keep overheads to processing resources very small,
  while still ensuring the portability of all software applications. The
  goal is to limit the need for astronomers to carry out software
  maintenance tasks and to keep consistency between batch processing
  and interactive analysis sessions. We describe the design and
  infrastructure of the system, the software building process on the
  server, and show an application with a multi-frame automated
  transient detection on a wide field survey, with a batch processing
  on a cloud infrastructure.
\end{abstract}

\section{Introduction}
During the past decade, several astronomical communities have started using
facilities that generate large amount of data, bringing challenges for
data storage and analysis. While data reduction
software for these facilities is typically deployed on dedicated
infrastructure, many astronomers keep performing analysis of the 
end products on their desktop. Such work flows create discontinuity between the
software stacks, turning end-to-end data analysis and result
reproducibility cumbersome. It is not clear whether it will scale
with forthcoming projects such as LSST or SKA, where even end
processed products will be too large for a typical desktop session.

Here we describe a modular approach that can be used for both
delivering legacy reliable astronomy software and that has the potential
to scale from a single astronomer own code up to distributing large
software stacks across processing nodes. Fortunately, reliable
software technologiesxs from the astronomy,
Linux and High Energy Physics communities are to achieve our goals. For
processing and storage, we rely on the Canadian Advanced Network for
Astronomical Research (CANFAR). To build and maintain software, we use the tools
from the Gentoo Linux package management system, and for distributing the
software, the CernVM File System (CVMFS) is being deployed.

\section{Components}
\subsection{CANFAR as cloud infrastructure}
CANFAR (\cite{canfar}) is a computing infrastructure designed for
astronomers. CANFAR aims to provide to its users easy access to very
large resources for both storage and processing, using a cloud based
framework. CANFAR allows astronomers to run processing jobs on a set
of computing clusters, and to store data at a set of data centres. The
same Virtual Machine (VM) can be used both for interactive analysis
and replicated across several clusters for batch processing.
There are two main components of CANFAR:
\begin{itemize}
\item Processing. Provisioning and managing virtual machines is done
  with the Nimbus \footnote{\url{http://www.nimbusproject.org/}}
  infrastructure suite. CANFAR offers the possibility
  to create a VM based on a golden image and managing it with both
  console and web based interface clients. Once configured,
  the condor high throughput computing software \footnote{\url{http://research.cs.wisc.edu/condor/}} is
  schedule VM jobs. The cloud scheduler \footnote{\url{htttp://www.cloudscheduler.org}} adds
  an extra layer to distribute VM across cluster instances.
\item Storage. A cloud storage based on the VOSpace specifications has
  been implemented for CANFAR users. The storage is used to store any
  type of data, including processing VM image files, and selective
  access controls to data. Various interfaces are offered to the
  users: console clients, a FUSE based mountable file system client,
  and a web interface. The storage location is transparent for
  the users, while practically we are trying to push towards data
  replication in the various computing clustering sites. A central
  database coupled with archiving tools developed at the Canadian
  Astronomy Data Centre are used as back-end to the VOSpace web services.
\end{itemize}

\indent CANFAR is an ongoing project. It has been tested by several
collaborations, with scheduling 3 clusters, providing up to 500
processor nodes, and a few hundred terabytes of stored data in
VOSpace.

\subsection{Distributing the software with CernVM-FS}
The processing software used on a CANFAR VM is currently embedded within the VM image
file itself. It obviously creates redundancy, as
well as scalability and maintenance problems. However the
biggest challenge users are faced so far is the installation and
maintenance of software. In an effort to remove the maintenance burden
to the non-proficient Linux user, we decided to try a ``kitchen sink''
approach, delivering a standard base of astronomy software, by means
of a powerful network file system recently developed for the Large
Hadron Collider experiments at CERN.

\indent The CernVM File System (CVMFS,\cite{blomer11}) is a client-server file system
specifically designed to deliver software onto virtual machines. CVMFS
is also implemented as a FUSE module. It creates a directory
tree stored on a web server, that appears as a local read-only file
system on the client. It transfers files on demand, verifying their
content by SHA-1 keys,  uses aggressive caching and reduction of latency. CVMFS is
part of the larger CernVM project \footnote{\url{http://cervm.cern.ch/portal}}, to distribute software and
conditions data to Grid sites. Several experiments distribute few tens of Gigabytes of software with CVMFS to worker
nodes, and various tests have shown the overheads are smaller than 5\%
processing time compared to a local disk installed software.

\begin{center}
  \begin{figure}
    \includegraphics[width=0.85\textwidth]{astrosink}
    \caption{The astronomy software system server-client and builder}
  \end{figure}  
\end{center}

\subsection{Server software building with Gentoo tools}
Currently the CANFAR VMs are built from
Scientific Linux 5. While being stable a stable distribution, software versions
are often out of date, resulting in frustrated users installing their own
compilers, maintaining their own copies of stock libraries and
sometimes developing their own package management system. There are however many UNIX/Linux distributions offering
very complete and modular package management. We resorted in
using Gentoo Linux package management (portage) and scripting formats
(ebuild) for the extreme configuration it offers, its reliability and
familiarity. The builds are done on a dedicated computer and propagated to the CVMFS
server. The packages are built from source, with the same system libraries as used on the
distributed VM for consistency and reproducibility. Every software
goes through the battery of unit testing when it exists, and a series
of integration tests. Specifically, we used the Gentoo Prefix system \footnote{\url{http://www.gentoo.org/proj/en/gentoo-alt/prefix/}}
which allows to install a full standalone software stack tree on a non
Gentoo system without root privileges.


\section{Tests}

As an example we show how we built, test and used our software stack
for a transient detection survey.
The NASA New Horizons mission is planed to fly by Pluto in 2015 and
continue onto the Kuiper Belt, providing the possibility of a close
encounter with a few Kuiper Belt Objects beyond Pluto by
2018. Potential targets must first be discovered and a coordinated
search between Subaru SuprimeCam, Magellan Megacam and IMACS, and CFHT
Megaprime to find potential target KBOs has started in 2011. Desired objects
are in the 50km size range, with magnitude at $r\approx 26$. The search lies in
Sagittarius so we need to remove a very dense star background. Each
field is visited multiple times and difference images are
batched-processed by the team. There are a few
pipelines organized across the collaboration, and the one we tested is
composed of various astronomy software of which the center block is poloka software suite, a C++
library originally developed for supernova surveys.

Every software component of the detection pipeline has
been packaged into a Gentoo ebuild and resulting binaries were tested
thoroughly on a dedicated VM, and distributed with a CernVM-FS
server. The pipeline process archived Megaprime images from the CADC archive,
plants fake KBO's, register, build references, performs single
exposure image subtractions, and shift and stacks the resulting
subtractions according to to-be-found KBO's expected velocities. Since
it uses mostly software developed in the previous decades, it is very
i/o dependent and stresses the network load and database back-end of
CANFAR. About 40,000 VM jobs were propagated through the CANFAR, with
a modest rate of 0.13\% of failures, none of which was from the software
delivery process.

\section{Development and future work}
We are now focusing our development of the described software
delivering system on building a very large stack of well tested,
reliable standard astronomy software and carry on further scalability
tests to deploy for the CANFAR users. We try as much as possible to keep the components
independent, to be ready for a very likely switch of technologies of
the fast moving cloud software. We are also starting to investigate
reducing virtualization overheads using Linux containers and improve
usability with web configuration interfaces for VM contextualization.

\bibliography{../../editor}

\end{document}
