% This is the aspauthor.tex LaTeX file
% Copyright 2010, Astronomical Society of the Pacific Conference Series

\documentclass[11pt,twoside]{article}
\usepackage{asp2010}
\usepackage{amstext}
\usepackage{amsfonts}

\resetcounters

\bibliographystyle{asp2010}

\markboth{Radhuber and Widicus Weaver}{Spectral Line Survey Analysis using MATLAB}

\begin{document}

\title{New Spectral Analysis Software for Global Analysis of Broadband Line Surveys in MATLAB}
\author{Mary L. Radhuber and Susanna L. Widicus Weaver
\affil{Department of Chemistry, Emory University, 1515 Dickey Drive, Atlanta, GA 30322}}

\begin{abstract}
New observatories with broadband spectral capabilities such as the Herschel Space Observatory, the Stratospheric Observatory for Infrared Astronomy, and the Atacama Large Millimeter Array necessitate the development of more robust and efficient spectral assignment methods.  These observatories provide the opportunity to tightly constrain physical conditions in astronomical sources through the large number of transitions observed for a given molecule in a single dataset.  However, these datasets also present a daunting analysis challenge due to the vast amount of data and the presence of blended line features which cannot be met by traditional line identification and rotation diagram analysis.  To meet this new challenge, we have written a program in the MATLAB numerical computing environment.  This new program has been designed to achieve simultaneous multi-molecule, multi-component fitting for an entire broadband spectrum.  This program improves on existing analysis methods in that it does an iterative fit to an observational spectrum using a global analysis method.  This global analysis includes global fitting for multiple molecules and single molecules with multiple density, temperature, and velocity components.  The advantage of this global method over traditional line-by-line analysis is that it is less sensitive to blended features which are so prevalent in many of these datasets.  The current version of the program is limited to the local thermodynamic equilibrium (LTE) approximation because radiative transfer information is not available for all of the complex molecules that are now routinely observed.  However, an advantage of this LTE approximation is that its simplicity allows for rapid analysis of a line survey.
\end{abstract}

\section{Introduction}
Broadband spectral line surveys provide an invaluable opportunity to constrain physical conditions of astronomical sources through observation of many transitions for a single molecule.  A range of excitation energies can be probed for each molecule, revealing information about the physical conditions of the source that can then be used to refine astrochemical models.  To date over 160 molecules have been detected in the interstellar and circumstellar medium \citep{Muller_2005}, with a large fraction of those being complex organic molecules (COMs). Here we follow the convention of a COM being a molecule containing six or more atoms \citep{Herbst_2009}. These COMs are of particular interest to many astrochemists, as they can be used to constrain astrochemical modeling networks \citep{Garrod_2008, Laas_2011}, and many are considered of prebiotic interest \citep{Carroll_2010}.

Due to the advent of new observational facilities such as the Herschel Space Obervatory, the Atacama Large Millimeter/submillimeter Array (ALMA), and the Stratospheric Observatory for Infrared Astronomy (SOFIA), the field of astrochemistry is quickly becoming overwhelmed with high quality spectral data, and the challenge now becomes how to best mine physical and chemical information from these large datasets.  While radiative transfer may be the most realistic method of analysis for many of these datasets, molecular collision data is not available for many molecules of interest \citep{vanderTak_2007}.  To answer the challenge of broadband spectral analysis within the constraints of the available molecular information, a global analysis program using the assumption of Local Thermodynamic Equilibrium (LTE) was created.  An overview of the program and a test of its initial performance is presented here.

\section{The Program}
Observational datasets formatted as an xy file with frequency data in MHz and intensity in $T_a^*(K)$ can be loaded directly into the program.  The spectrum can be calibrated to the $T_{MB}(K)$ or $T_b(K)$ scale, depending on knowledge of source size.  The spectrum is calibrated to the $T_{MB}(K)$ scale using the conversion:
$$T_{MB}(K) = T_a^*(K)/\eta$$
where $\eta$ is the telescope efficiency.  The data can be further calibrated to the $T_b(K)$ scale using the conversion:
$$\text{T}_\text{b}(\text{K}) = \text{T}_\text{MB}(\text{K}) B$$
where $B$ is the beam filling factor calculated by:
$$B = \frac{\theta_s^2}{\theta_s^2 + \theta_b^2}$$
where $\theta_s$ is the source size and $\theta_b$ is the beam size \citep{NummelinAA_1998}.
After the spectrum has been calibrated, the user should determine the noise level of the calibrated spectrum.  If the noise level of the uncalibrated data is known, the calibrated noise level can be determined by $\text{noise}_\text{calib} = \text{noise}_\text{raw} /\eta$ or $\text{noise}_\text{calib} = \text{noise}_\text{raw} \cdot B/\eta$ for source size unknown or known, respectively.  Once the calibrated noise level is determined, the user can input this parameter as the noise threshold for line identification.  The user can specify the number of standard deviations above the noise required to consider a peak a spectral feature.  The program defaults to a $3$-$\sigma$ threshold.

After the observational dataset is loaded and properly calibrated, spectral line catalogs for each molecule of interest are loaded in the standard JPL catalog format \citep{Pickett_1998}. Processing time is reduced by removing from the catalog line list any transitions that are below a user specified $\log_{10}$ intensity and above a frequency uncertainty level in MHz.  If the user does not specify thresholds for these parameters, no transitions will be removed from the data.  The catalog files are internal to the program and unique to each observational project, so that local catalog files will not be corrupted through this process.  An additional time saving procedure is incorporated such that the program truncates the catalog files so that only the frequency range of the observational data is included in the simulation.

The program simulates spectra using an interpolated partition function from data that can be found in the molecule documentation of the JPL and CDMS catalogs \citep{Pickett_1998, Muller_2005}.  The program assumes a rotational partition function:
$$Q(T) = \sqrt{\frac{\pi}{ABC}\left(\frac{k T}{h}\right)^3}$$
The catalog information is used along with this partition function information provided at 300~K to determine the Einstein~A coefficients of the transitions:
$$Ag_u = \frac{2.7964\times10^{-16}\cdot I \cdot \nu^2 \cdot Q(300)}{e^{-E_l/k300}- e^{-E_u/kT}}$$
where $I$ is the line strength ($I = 10^{\log_{10}(LGINT)}$), $Q(300)$ is the partition function at 300~K, and $E_L$ and $E_U$ are the lower and upper state energies in joules.  $Ag_u$ is the Einstein~A coefficient of the transition times the upper state degeneracy.  The molecular spectrum is simulated for a given column density ($N_T$), spectral line full width half maximum ($FWHM$) in km/s, rotational temperature ($T$) in Kelvin, and frequency shift in MHz.  The simulated spectrum is a sum of Gaussian line profiles for all transitions.  Each transition has the form of a Gaussian:
$$\text{line} = T_b~e^{\frac{-(x - \nu + shift)^2}{\left(FWHM/\left(2\sqrt{2\ln(2)}\right)\right)^2}}$$
where $T_b$ is the intensity of the transition in K calculated by:
$$T_b = \frac{h c^3}{8\pi k\nu^2}\frac{Ag_u N_T}{Q(T) 1.064 FWHM}e^{\frac{-E_U}{k T}}1\times 10^{-11}$$
This expression was derived from the equation for the integrated intensity of a transition:
$$\int_{-\infty}^{\infty}T_b dv = \frac{h c^3}{8\pi k\nu^2}\frac{Ag_u N_T}{Q(T)}e^{-E_u/k T}$$
\citep{NummelinApJS_1998} and the integrated intensity for a Gaussian line shape:
$$\int_{-\infty}^{\infty}T_b dv \approx 1.064~T_b~\Delta v.$$

The program determines the best set of physical parameters ($N_T,~FWHM,~T$, shift) for the molecule specified by the user through comparison of simulated data to observational data.  Up to four components can be successfully fit simultaneously for a given observational data set.  These components can either be different molecules, separate sets of physical parameters for the same molecule, or any combination thereof.  The fit is preformed using the pattern search function in the MATLAB global optimization toolbox \citep{MATLAB}.  This function is appropriate because, as a derivative free method, it is useful for minimization of non-smooth multivariate functions.  The pattern search method is also useful in that it lends itself well to parallel processing, enabling simultaneous fitting of many components.  As a typical optimization will utilize four cores, this program is well suited for use on laptop and office desktop computers, and ensured access to a computer cluster is not necessary for use of this program.  A separate version of the program designed specifically for use on clusters will be available in the future.

It is recommended that the user attempt to fit the observational data for the molecules with the most intense lines first, and then work down in intensity from there.  Optimization for additional molecules takes into account fit values for previously fit molecules.  The parameters for the previous fits can either be fixed, or left open for further optimization on a limited range.  The user is warned that the computation cost required for an optimization increases exponentially with the number of components.

Once a satisfactory fit is achieved, the program can then match the frequencies of the observational peaks and the simulated peaks to assign observed transitions.  The user can then export the final result as the frequency, observed intensity, predicted intensity, Einstein~A coefficient, upper state degeneracy, and quantum numbers for each transition.  The program also exports the best fit parameters for each molecule and the xy files of the simulation for each molecule, the total simulation, and the residual.


\bibliography{P23}

\end{document}
