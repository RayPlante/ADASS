\documentclass[11pt,twoside]{article}
\usepackage{asp2010}

\resetcounters
\bibliographystyle{asp2010}

\markboth{Currie}{Automated removal of bad-baseline heterodyne spectra}

\begin{document}

\title{Automated removal of bad-baseline spectra from ACSIS/HARP
heterodyne time series}
\author{Malcolm J. Currie}
\affil{Joint Astronomy Centre, 660, N.Aohoku Place, Hilo, Hawaii 96720, USA}

\begin{abstract}
Heterodyne time-series spectral data often exhibit distorted or noisy 
baselines. These are either transient due to external interference or 
pickup; or affect a receptor throughout an observation or extended 
period, possibly due to a poor cable connection. While such spectra can 
be excluded manually, this is time consuming and prone to omission, 
especially for the high-frequency interference affecting just one or two 
spectra in typically several to twenty thousand, yet can produce undesirable 
artefacts in the reduced spectral cube. Further astronomers have tended 
to reject an entire receptor if any of its spectra are suspect; as a 
consequence the reduced products have lower signal-to-noise, and 
enhanced graticule patterns due to the variable coverage and detector 
relative sensitivities.

This paper illustrates some of the types of aberrant spectra for
ACSIS/HARP on the James Clerk Maxwell Telescope and the algorithms
used to identify and remove them applied within the ORAC-DR pipeline,
and compares an integrated map with and without baseline filtering. 
\end{abstract}

\section{Introduction}

ACSIS/HARP \citep{buckle_2009} is a heterodyne system installed on
the James Clerk Maxwell Telescope. Its now 15-element focal-plane
array receiver operates in the submillimetre from 325 to 375 GHz
generating cubes with spectral, receptor, and time-series axes.
Typical observations are scanned in one spatial direction followed by
an orthogonal scan to form a `basket weave' thus different receptors
are used to sample the same sky point.

The data are automatically reduced using the ORAC-DR pipeline system
\citep{cavanagh_2008,jenness_2008} to generate spectral cubes
(long,lat,velocity). It encompasses an iterative process to mask the
spectral lines to improve baseline subtraction.  It also performs
quality-assurance checks on all the input time series.

Under certain conditions not fully understood---likely sources include 
cables and their connections, and certain telescope motions---the 
baselines of the spectra can exhibit interference in the form of noise, 
uneven or distorted baselines.  Left unfiltered these appear as stripes 
in the final products, degrading the data both cosmetically and 
astrophysically.

While many affected spectra could be removed manually, a single 
observation usually has at least 2000 spectra and typically a few to 
several times that, 14 or 15 working receptors; and a minimum dataset 
comprises two orthogonal scan directions.  Not only is manual detection 
and removal time consuming, it is prone to omission.  Some of the 
defects appear in isolated spectra easily missed by visual inspection 
yet they have a dramatic affect on the pipeline products.  Some 
distortions are subtle or fine-grained and require panning across 
thousands of spectral elements, and zooming to view individual pixels.  
In practice astronomers rejected a receptor if it showed any clear bad 
baselines, as if the issue were instrinsic to the receptor itself rather 
than external and usually transient sources.

The goal was to automate the filtering of these anomalous spectra in 
ORAC-DR, yet not be too expensive in cpu time as the biggest maps
already could take several hours to a day to reduce.

\section{Bad Spectra}

The bad baselines can be divided roughly into two classes: 
high-frequency, high amplitude; and low frequency, lower amplitude. The 
former appear mostly in single isolated spectra or in narrow bands,
but can also manifest as spiky spectra, or weaker-amplitude striations 
persistent over tens of spectra.  Figure~1 presents the most-common forms.

\begin{figure}[!ht]
\plottwo{P61_f1a}{P61_f1b}
\caption{Examples of high-frequency interference in spectral-time axes. 
The left panel shows high-amplitude interference affecting single
spectra; and part of an affected spectrum plotted below. The amplitude
dwarfs the normal signal by an order of magnitude. The right panel shows 
bands of phase-shifting interference.}
\end{figure}

The low-frequency ripples tend to occur in time-series blocks that are
often visible because of baseline drift, but can apply to all spectra
for a receptor. They have a wide range of morphologies such as
sinusoids, irregular ripples, and curved, twin headlight-like beams
that start strong but gradually pan out and fade with time.  Figure~2
displays some examples.


\begin{figure}[!ht]
\plottwo{P61_f2a}{P61_f2b}
\caption{Examples of low-frequency interference.  The top-left 
graphic shows a band of low-frequency oscillations; below it is the 
band's average spectrum.  The lower-left spectrum has a periodic weak
ripple hard to detect visually.  The right panel shows time-averaged
(clipped mean) spectra for each receptor in which the fourth and ninth
from the bottom exhibit global non-linearity.}
\end{figure}

\section{Adopted Solution}

The pipeline applies three steps in the quality-assurance stage: 
\begin{itemize}
\item Laplacian filtering of high-frequency noise;
\item non-linearity detection for individual spectra; and
\item global non-linearity to reject whole receptors.
\end{itemize}

\subsection{Masking of High-frequency Noise}
 
The recipe applies a one-dimensional Laplacian edge filter to all the
spectra for each receptor trimming the outer 15\% where noise is
always present.  This approximates to a difference-of-Gaussian filter.
It next sums the rms `edginess' along the spectral axis to form a
profile through the time series.  Drifts or steps in the profile are
removed.  The final step is to eject spectra whose rms edginess
exceeds the median level by a nominated number of clipped standard
deviations.  Affected spectra are easily dilineated.  An optional
second iteration removes most of the striation noise once the
pronounced edginess peaks are masked.

\subsection{Non-linearity Filtering}

The low-frequency rippled and wobbly baselines are addressed by 
determining the non-linearity of each spectrum.  First the recipe
excludes non-baseline features that would dilute the non-linearity
signal.  These comprise a threshold to remove spikes and masking a
central region where the astronomical signal may be present. It
estimates the background level, effectively smoothing to remove
structure smaller than a nominated scale.  Next it fits linear
baselines to these and calculates the rms residuals to provide a
rectified signal.  Then it averages the signal along the spectral axis
to form a non-linearity profile through the time series for each
good receptor.

The non-linear profiles are much noisier than the summed Laplacians
and discrimation is harder.  To identify anomalous spectra the recipe
reduces the noise to obtain a smooth profile, correct for drifts or
steps in the profile.  It rejects spectra whose mean non-linearity
exceeds the mean level above a nominated number of clipped standard
deviations.  The derived standard deviation allows for positive
skewness.  It applies a mask of rejected spectra to the input cube.

The global non-linearity test is applied last so that a block of transient 
highly deviant spectra will not cause the whole receptor to be rejected.  It 
operates in a similar fashion to the above.  It diverges by determining a 
mean rms residual from non-linearity per detector, from which it evaluates 
the median and standard deviation of the distribution of mean rms 
residuals from the entire observation, and performs iterative sigma 
clipping above the median to reject those detectors whose deviations from 
linearity are anomalous.  There is a tunable minimum threshold.

\section{Results}

The methods appear highly effective at cleaning the pipeline products.
It has been used to re-reduce one survey and several other datasets.
Figure~3 presents an example which had originally failed quality assurance,
but now can be used for science.  


\begin{figure}[!ht]
\plottwo{P61_f3a}{P61_f3b}
\caption{Comparison of an integrated intensity map using the same
percentile scaling.  The left panel shows a map without bad-baseline
removal.  The right panel shows the same data processed using the
new filters.}
\end{figure}

\acknowledgements This work was funded by the UK Science and Technology Facilities Council.

\bibliography{P61}

\end{document}
