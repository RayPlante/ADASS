
\resetcounters

\bibliographystyle{asp2010}

\markboth{Jeschke, Inagaki, and Kackley}{Introducing the Ginga FITS Viewer}

\title{Introducing the \ssindex{applications!Ginga}Ginga FITS Viewer and Toolkit}
\author{Eric~Jeschke, Takeshi~Inagaki, and Russell~Kackley
\affil{Subaru~Telescope,~National~Astronomical~Observatory~of~Japan}}

\aindex{Jeschke, E.}
\aindex{Inagaki, T.}
\aindex{Kackley, R.}

\begin{abstract}
We introduce \ssindex{applications!Ginga}Ginga, a new open-source \ssindex{data formats!FITS}FITS
viewer and toolkit based on \ssindex{computer languages!Python}Python astronomical packages such as \ssindex{libraries!PyFITS}pyfits\ooindex{PyFITS, ascl:1207.009}, \ssindex{libraries!numpy}numpy, \ssindex{libraries!scipy}scipy, \ssindex{libraries!matplotlib}matplotlib, and \ssindex{libraries!PyWCS}pywcs.  For developers, we present a set of \ssindex{computer languages!Python}Python classes for viewing \ssindex{data formats!FITS}FITS files under the modern \ssindex{software!tools!Gtk}Gtk and \ssindex{software!tools!Qt}Qt widget sets and a more  full-featured viewer that has a plugin architecture. We further describe how plugins can be written to extend the viewer with many different capabilities.  

The software may be of interest to software developers who are looking for a solution for integrating \ssindex{data formats!FITS}FITS \ssindex{visualization}visualization into their \ssindex{computer languages!Python}Python programs and end users interested in a new and different \ssindex{data formats!FITS}FITS viewer that is not based on \ssindex{computer languages!Tcl/Tk}Tcl/Tk widget technology.  The software has been released under a \ssindex{license!BSD}BSD license.
\end{abstract}

\section{Introduction}
In recent years there has been a rapid uptake of the \ssindex{computer languages!Python}Python programming language in the astronomical community for software development and data reduction tasks. This has been driven not only by the productivity gains of writing in a very high-level object-oriented language, but also by the development of several very useful scientific packages for \ssindex{computer languages!Python}Python, including  {\tt \ssindex{libraries!numpy}numpy}, {\tt \ssindex{libraries!PyFITS}pyfits}\ooindex{PyFITS, ascl:1207.009} \citep{Barrett99}, {\tt \ssindex{libraries!scipy}scipy} and {\tt \ssindex{libraries!matplotlib}matplotlib} \citep{matplotlib}, all of which let you script high performance scientific\ssindex{computer languages!C} C routines from \ssindex{computer languages!Python}Python. Collectively these modules make a very powerful open-source toolkit for the processing and analysis of astronomical data.

At \ssindex{observatories!Earth-based!Subaru}Subaru Telescope, the Observation \ssindex{software!control system}Control System software development team has leveraged these tools to develop a second generation OCS \citep{Jeschke10B}. Because almost all of the code is in \ssindex{computer languages!Python}Python, we had a need for a \ssindex{computer languages!Python}Python \ssindex{data formats!FITS}FITS file viewing widget.  We have developed a \ssindex{data formats!FITS}FITS viewing widget that can be embedded in \ssindex{computer languages!Python}Python programs using the modern \ssindex{software!tools!Gtk}Gtk or \ssindex{software!tools!Qt}Qt widget sets. Based on this widget we have also developed a more full-featured standalone viewer, called \emph{Ginga}\footnote{Ginga means ``galaxy'' in Japanese. Acceptable pronunciations for the name of the software include ``Ging-ga'' (best) or ``Jing-ga'' (ok).}, that implements most of its features via a modular plugin system. This software has been released under an open-source \ssindex{license!BSD}BSD license \citep{Jeschke12}. In this paper we will describe the basic capabilities and usage of the software and describe briefly the plugin architecture for extending the viewer.
\ooindex{ginga, ascl:1303.020}

\section{A \ssindex{computer languages!Python}Python FITS Viewing Widget}
At the core of the \ssindex{applications!Ginga}Ginga software is a set of object classes that implement the viewing widget.  These classes allow the creation of a widget that displays a \ssindex{data formats!FITS}FITS file and supports scaling (zooming), panning, cut levels, color and intensity mapping, image transforms and plotting of several types of simple line graphics. Figure \ref{fig:onion} illustrates the object-oriented inheritance diagram of these classes.
\label{sec:coreclasses}
\begin{figure}
  \begin{center}
    \begin{tabular}{c}
      \includegraphics[width=3in]{part9/Jeschke_O17/O17_f1.eps}
    \end{tabular}
  \end{center}
  \caption[example] 
          { \label{fig:onion} 
            Object oriented design of the core rendering widgets in \ssindex{applications!Ginga}Ginga.} 
\end{figure} 
The base class implements most of the functionality of the viewing widget, handling all of the details of the rendering out to a generic RGB image array. The subclasses are responsible for rendering the RGB array to a particular drawing widget in a widget set and for handling user input events. Finally, mixin classes \citep{mixins} provide the support for specific mapping of events to actions and overlaying graphics on the image. This architecture allows the widget to be ported fairly easily to new platforms, because almost any widget set for any platform has a basic drawing widget that can accept RGB data and display it.

For developers interested in a basic \ssindex{data formats!FITS}FITS display widget for their \ssindex{computer languages!Python}Python programs one of these classes may be suitable. The \ssindex{applications!Ginga}Ginga widgets are complementary to, and interoperate with, other scientific \ssindex{computer languages!Python}Python packages; images are passed in as {\tt \ssindex{libraries!numpy}numpy} arrays, {\tt \ssindex{libraries!PyFITS}pyfits}\ooindex{PyFITS, ascl:1207.009} objects, buffers, or loaded from a file. There are several example programs distributed with the package that illustrate how to use the widget by itself in either \ssindex{software!tools!PyGtk}pygtk or \ssindex{software!tools!PyQt}pyqt, which are the \ssindex{computer languages!Python}Python mappings to the \ssindex{software!tools!Gtk}Gtk and \ssindex{software!tools!Qt}Qt widget sets. These two modern widget sets are available for the Linux, Windows and Mac OS X platforms (and selected other Unix platforms).

\section{\ssindex{applications!Ginga}Ginga: A Full-Featured FITS Viewer}
For end-users, or developers interested in starting with something more comprehensive and customizing it, there is the \ssindex{applications!Ginga}Ginga viewer.  The viewer is essentially a highly configurable multi-container with a collection of \emph{plugins}--encapsulated modules that perform various operations. This provides a viewer with many features similar to the venerable \ssindex{applications!Skycat}skycat\ooindex{SkyCat, ascl:1109.019} \citep{skycat} or \ssindex{applications!DS9}DS9 (although not yet as feature complete). The viewer can be easily reconfigured to show the plugin \ssindex{software!user interfaces}Graphical User Interfaces (\ssindex{software!user interfaces}GUIs) in different layouts and arrangements. Figure \ref{fig:ginga} shows an example of the viewer with several plugins active. 
\begin{figure}
  \begin{center}
    \begin{tabular}{c}
      \includegraphics[width=5in]{part9/Jeschke_O17/O17_f2.eps}
    \end{tabular}
  \end{center}
  \caption[example] 
          { \label{fig:ginga} 
            \ssindex{applications!Ginga}Ginga full viewer displaying M33. Plugins for panning, \ssindex{World Coordinate System (WCS)}WCS info and stellar evaluation are shown.} 
\end{figure} 

\subsection{A Plugin Architecture}
Almost all of the features \ssindex{applications!Ginga}Ginga has, outside of those provided by the basic display widget, are implemented via the plugin mechanism. \ssindex{applications!Ginga}Ginga comes with a collection of two dozen or so plugins for things like preferences, browsing files, stellar evaluation (``pick''), star catalog access, pixel values along a line (``cuts''), histograms, zoom display, distance measurement, etc.

Plugins follow an Application Programming Interface (API), which allows each plugin to both control the viewer and to be managed by the viewer. Aside from following the API, users writing custom plugins have the full power of the \ssindex{computer languages!Python}Python language at their disposal, allowing a great deal of freedom and capability in terms of what can be done.   

\subsection{A workspace-oriented layout}
\ssindex{applications!Ginga}Ginga has a flexible panel/workspace layout algorithm that allows a lot of customization into the appearance of the program.  The majority of the interface is constructed as hierarchical series of workspaces. A workspace contains either a \ssindex{data formats!FITS}FITS viewing widget, a plugin \ssindex{software!user interfaces}GUI, or recursively, a nested workspace containing multiple items. Several kinds of workspace containers are supported including sliding panels, notebook widgets, stack widgets, or fixed subspaces. On some platforms, items can be dynamically moved between workspaces or even dragged out onto the desktop, forming a new, detached workspace. A couple of tables in the top-level script control the layout and names of the workspaces and the mapping of items to those workspaces; by modifying these tables the user can customize the viewer layout in a myriad of ways, without writing any new code. 

\section{Conclusion}
We have made the \ssindex{applications!Ginga}Ginga \ssindex{software!source code}source code publicly available under a \ssindex{license!BSD}BSD license, similar to that used by the other open-source scientific \ssindex{computer languages!Python}Python modules it relies on. We hope that the software may be useful to other astronomical software developers, particularly those using \ssindex{computer languages!Python}Python-based technologies, and end users simply interested in trying out the capabilities of a new viewer running in a more modern widget set.  The viewer runs on the three major PC platforms: Linux, Apple Macintosh and Microsoft Windows, as well as a few other Unix platforms that support either the \ssindex{software!tools!Gtk}Gtk or \ssindex{software!tools!Qt}Qt widget sets. The software, documentation and installation instructions can be found at the project web site: \url{http://ejeschke.github.com/ginga}.

\acknowledgements We would like to acknowledge the support and encouragement of the \ssindex{observatories!Earth-based!Subaru}Subaru Telescope/NAOJ staff, particularly Hiroshi Terada, Daigo Tomono, and Akito Tajitsu from the Gen2 OCS advisory committee. 

\bibliography{editor}
