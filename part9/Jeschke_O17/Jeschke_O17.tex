% This is the aspauthor.tex LaTeX file
% Copyright 2010, Astronomical Society of the Pacific Conference Series

\documentclass[11pt,twoside]{article}
\usepackage{asp2010}

\resetcounters

\bibliographystyle{asp2010}

\markboth{Eric Jeschke, Takeshi Inagaki and Russell Kackley}{Introducing the Ginga FITS Viewer}

\begin{document}

\title{Introducing the Ginga FITS Viewer and Toolkit}
\author{Eric~Jeschke, Takeshi~Inagaki, and Russell~Kackley}
\affil{Subaru~Telescope,~National~Astronomical~Observatory~of~Japan}

\begin{abstract}
We introduce Ginga, a new open-source FITS
viewer and toolkit based on Python astronomical packages such as pyfits,
numpy, scipy, matplotlib and pywcs.  For developers, we present a set of
python classes for viewing FITS files under the modern Gtk and Qt widget
sets and a more  full-featured viewer that has a plugin architecture.
We further describe how plugins can be written to extend the viewer with
many different capabilities.  

The software may be of interest to software developers who are
looking for a solution for integrating FITS visualization into their
python programs and end users interested in a new and different FITS
viewer that is not based on Tcl/Tk widget technology.  The software has
been released under a BSD license.
\end{abstract}

\section{Introduction}
In recent years there has been a rapid uptake of the Python
programming language in the astronomical community for software
development and data reduction tasks.
This has been driven not only by the productivity gains of writing in a
very high-level object-oriented language, but also by the development of
several very useful scientific packages for python, including 
{\tt numpy}, {\tt pyfits} \citep{Barrett99}, {\tt scipy} and 
{\tt matplotlib} \citep{matplotlib}, all of which let you script high
performance scientific C routines from Python. 
Collectively these modules make a very powerful open-source toolkit for the
processing and analysis of astronomical data.

At Subaru Telescope, the Observation Control System software development
team has leveraged these tools to develop a second generation
OCS \citep{Jeschke10B}.  
Because almost all of the code is in python, we had a need for a python
FITS file viewing widget.  We have developed a FITS viewing widget that
can be embedded in python programs using the modern Gtk or Qt widget
sets.
Based on this widget we have also developed a more full-featured
standalone viewer, called \emph{Ginga}\footnote{Ginga means ``galaxy''
  in Japanese. Acceptible pronounciations for the name of the software
  include ``Ging-ga'' (best) or ``Jing-ga'' (ok).}, 
that implements most of its features via a modular plugin system.
This software has been released under an open-source BSD
license \citep{Jeschke12}.
In this paper we will describe the basic capabilities and usage of the
software and describe briefly the plugin architecture for extending the
viewer.

\section{A python FITS viewing widget}
At the core of the Ginga software is a set of object classes that
implement the viewing widget.  These classes allow the creation of
a widget that displays a FITS file and supports scaling (zooming),
panning, cut levels, color and intensity mapping, image transforms and
plotting of several types of simple line graphics.   
Figure \ref{sec:coreclasses} illustrates the object-oriented inheritance
diagram of these classes.
\label{sec:coreclasses}
\begin{figure}
  \begin{center}
    \begin{tabular}{c}
      \includegraphics[width=3in]{O17_f1.eps}
    \end{tabular}
  \end{center}
  \caption[example] 
%% %>>>> use \label inside caption to get Fig. number with \ref{}
          { \label{fig:onion} 
            Object oriented design of the core rendering widgets in Ginga.} 
\end{figure} 
The base class implements most of the functionality of the viewing
widget, handling all of the details of the rendering out to a generic
RGB image array.
The subclasses are responsible for rendering the RGB array to a
particular drawing widget in a widget set and for handling user input
events. 
Finally, mixin classes \citep{mixins}
provide the support for specific mapping of events to actions and
overlaying graphics on the image.
This architecture allows the widget to be ported fairly easily to new
platforms, because almost any widget set for any platform has a basic
drawing widget that can accept RGB data and display it.

For developers interested in a basic FITS display widget for their
python programs one of these classes may be suitable.
The Ginga widgets are complementary to, and interoperate with, other
scientific python packages; images are passed in as {\tt numpy} arrays,
{\tt pyfits} objects, buffers or loaded from a file.  
There are several 
example programs distributed with the package that illustrate how
to use the widget by itself in either pygtk or pyqt, which are the
python mappings to the Gtk and Qt widget sets. 
These two modern widget sets are available for the Linux, Windows and
Mac OS X platforms (and selected other Unix platforms).

\section{Ginga: a full-featured FITS viewer}
For end-users, or developers interested in starting with something more
comprehensive and customizing it, there is the Ginga viewer.  The viewer is
essentially a highly configurable multi-container with a collection of
\emph{plugins}--encapsulated modules that perform various operations.
This provides a viewer with many features similar to the venerable
skycat \citep{skycat} or DS9 (although not yet as feature complete).
The viewer can be easily reconfigured to show the plugin Graphical User
Interfaces (GUIs) in different layouts and arrangements.  
Figure \ref{fig:ginga} shows an example of the viewer with several
plugins active. 
\begin{figure}
  \begin{center}
    \begin{tabular}{c}
      \includegraphics[width=5in]{O17_f2.eps}
    \end{tabular}
  \end{center}
  \caption[example] 
          { \label{fig:ginga} 
            Ginga full viewer displaying M33. Plugins for panning, WCS
            info and stellar evaluation are shown.} 
\end{figure} 

\subsection{A plugin architecture}
Almost all of the features Ginga has, outside of those provided by the
basic display widget, are implemented via the plugin mechanism.
Ginga comes with a collection of two dozen or so plugins for things
like preferences, browsing files, stellar evaluation (``pick''), star
catalog access, pixel values along a line (``cuts''), histograms,
zoom display, distance measurement, etc.

Plugins follow an Application Programming Interface (API), which allows
each plugin to both control the viewer and to be managed by the viewer.
Aside from following the API, users writing custom plugins have the full
power of the python language at their disposal, allowing a great deal of
freedom and capability in terms of what can be done.   

\subsection{A workspace-oriented layout}
Ginga has a flexible panel/workspace layout algorithm that allows a
lot of customization into the appearance of the program.  The majority
of the interface is constructed as hierchical series of workspaces.
A workspace contains either a FITS viewing widget, a plugin GUI, or
recursively, a nested workspace containing multiple items.
Several kinds of workspace containers are supported including sliding
panels, notebook widgets, stack widgets, or fixed subspaces. 
On some platforms, items can be dynamically moved between workspaces or
even dragged out onto the desktop, forming a new, detached workspace.
A couple of tables in the top-level script control the layout and names of
the workspaces and the mapping of items to those workspaces;
by modifying these tables the user can customize the viewer layout in a
myriad of ways, without writing any new code. 

\section{Conclusion}
We have made the Ginga source code publically available under a BSD
license, similar to that used by the other open-source scientific python
modules it relies on. 
We hope that the software may be useful to other astronomical software
developers, particularly those using python-based technologies, and end
users simply interested in trying out the capabilities of a new viewer
running in a more modern widget set.  The viewer runs on the three major
PC platforms: Linux, Apple Macintosh and Microsoft Windows, as well as a
few other Unix platforms that support either the Gtk or Qt widget sets.
The software, documentation and installation instructions can be found
at the project web site: \url{http://ejeschke.github.com/ginga}.

\medskip
\acknowledgements We would like to acknowledge the support and
encouragement of the Subaru Telescope/NAOJ staff, particularly Hiroshi
Terada, Daigo Tomono and Akito Tajitsu from the Gen2 OCS advisory
committee. 

\bibliography{../../editor}

\end{document}
