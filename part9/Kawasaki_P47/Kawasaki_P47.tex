
\resetcounters

\bibliographystyle{asp2010}

\markboth{Kawasaki et al.}{\ssindex{applications!Vissage}\ssindex{packages!Vissage}Vissage: An ALMA-VO Desktop Application}

\title{\ssindex{applications!Vissage}\ssindex{packages!Vissage}Vissage: An ALMA-VO Desktop Application}
\author{Wataru~Kawasaki, Satoshi~Eguchi, Yuji~Shirasaki, Yutaka~Komiya, George~Kosugi, Masatoshi~Ohishi, 
and Yoshihiko~Mizumoto
\affil{National Astronomical Observatory of Japan, 2-21-1 Osawa, Mitaka, Tokyo, 181-8588, Japan}}

\aindex{Kawasaki, W.}
\aindex{Eguchi, S.}
\aindex{Shirasaki, Y.}
\aindex{Komiya, Y.}
\aindex{Kosugi, G.}
\aindex{Ohishi, M.}
\aindex{Mizumoto, Y.}

\begin{abstract}
\ssindex{applications!Vissage}\ssindex{packages!Vissage}Vissage is a \ssindex{data formats!FITS}FITS browser primarily targeting \ssindex{data formats!FITS}FITS data cubes obtained from \ssindex{observatories!Earth-based!ALMA}ALMA. It currently offers some basic functionalities to view three-dimensional data cube and to connect with the JVO's \ssindex{observatories!Earth-based!ALMA}ALMA data service. We will describe the present development status of \ssindex{applications!Vissage}\ssindex{packages!Vissage}Vissage, its aims, and future plans.
\end{abstract}

\section{Introduction}
Since it has been one year from the beginning of scientific operations based on selected observing proposals, the Atacama Large Millimeter/submillimeter Array (\ssindex{observatories!Earth-based!ALMA}ALMA) will soon be releasing data whose proprietary periods have expired, in addition to Science Verification data that have already been released. The released \ssindex{observatories!Earth-based!ALMA}ALMA data will be available from websites of the three \ssindex{observatories!Earth-based!ALMA}ALMA  Regional Centres located at East Asia, North America and Europe, and we are planning to distribute \ssindex{observatories!Earth-based!ALMA}ALMA data via the \ssindex{Virtual Observatory (VO)!individual!Japanese Virtual Observatory (JVO)}Japanese \ssindex{Virtual Observatory (VO)}Virtual Observatory (JVO) as well. We are then faced with two problems in distributing \ssindex{observatories!Earth-based!ALMA}ALMA data in the framework of \ssindex{Virtual Observatory (VO)}Virtual Observatory (hereafter\ssindex{Virtual Observatory(VO)} VO) system, namely, its huge size and high dimensionality. 

\ssindex{observatories!Earth-based!ALMA}ALMA's unprecedented performance, especially when it gets to its full specifications in the near future, leads us to an obvious expectation that even a single dataset from \ssindex{observatories!Earth-based!ALMA}ALMA could be on the Terabyte-scale. As for dimensionality, \ssindex{observatories!Earth-based!ALMA}ALMA data will be provided as data cubes with 3 or 4 dimensions as typical \ssindex{astronomy!radio}radio imaging observations: spatial 2D and spectral  1D, often plus \ssindex{astronomy!polarization}polarisational 1D. Compared with ordinary 2D images, data cubes have a much wider variety in ways of viewing including moment maps, channel maps, Position-Velocity diagrams, etc. 

To cope with such huge datasets with high dimensionality in JVO, we have been developing a new mechanism and software \citep{O10_adassxxii, D5_adassxxii}. Our project covers both the server-side and the desktop side: the former includes the new functions of JVO \ssindex{observatories!Earth-based!ALMA}ALMA portal including the ALMA Web Quick Look System (hereafter WebQL; \citet{O10_adassxxii}). 

As a solution for the latter, we are developing \ssindex{applications!Vissage}\ssindex{packages!Vissage}Vissage ({\it VISualisation Software for Astronomical Gigantic data cubEs}), a brand-new FITS browser. Its primary use case includes (1) viewing \ssindex{data formats!FITS}FITS data cubes downloaded from the JVO portal or WebQL for research on interesting regions and (2) connecting the JVO seamlessly to search and download \ssindex{observatories!Earth-based!ALMA}ALMA data for the same object but with a different spatial/spectral resolutions from the one you are viewing to better fit your scientific purposes. 

\section{Current Status of \ssindex{applications!Vissage}\ssindex{packages!Vissage}Vissage}
\subsection{Distribution Package}
\ssindex{applications!Vissage}\ssindex{packages!Vissage}Vissage is now available from the JVO website (\url{http://jvo.nao.ac.jp/download/Viss} \url{age}) as a tarball. It contains a front-end program (.exe) for Windows and a main program (executable jar file). The main program is implemented with \ssindex{computer languages!Java}Java/Swing, so it runs on all platforms with JRE (Java Runtime Environment) 6 or newer. Though it is directly executable, Windows users are strongly encouraged to launch it via the front-end program. The front-end for Windows platform enables you to open images by drag and drop onto a shortcut of \ssindex{applications!Vissage}\ssindex{packages!Vissage}Vissage and computes an appropriate heap size for the Java Virtual Machine (the latter is critical especially when you use\ssindex{computing!architecture} 32-bit \ssindex{computer languages!Java}Java). 

\subsection{Available Functions}
\ssindex{applications!Vissage}\ssindex{packages!Vissage}Vissage now offers basic functions to show major two-dimensional views of data cubes. Currently available functions include:
\begin{itemize}
  \item Open \ssindex{data formats!FITS}FITS data
    \begin{itemize}
      \item by drag and drop onto the main window (for all platforms)
      \item by drag and drop onto \ssindex{applications!Vissage}\ssindex{packages!Vissage}Vissage icon or shortcut (for Windows only)
    \end{itemize}
  \item Readable \ssindex{data formats!FITS}FITS files
    \begin{itemize}
      \item ALMA data
      \item Nobeyama 45m OTF data
    \end{itemize}
  \item Two-dimensional view of data cubes
    \begin{itemize}
      \item integrated intensity map
      \item first/second moment maps
      \item flipbook (viewing images for each frequency channel)
      \item channel maps
      \item Position-Velocity diagram
    \end{itemize}
  \item Connect with JVO
    \begin{itemize}
      \item with ALMA Web QL
      \item with JVO ALMA portal via \ssindex{protocols!TAP}TAP (Table Access Protocol)
    \end{itemize}
  \item Image operation
    \begin{itemize}
      \item drag/pan images
      \item zoom-in/zoom-up
      \item change colorset/bias/contrast
      \item multiple images with flexible layout
    \end{itemize}
\end{itemize}

You can see how \ssindex{applications!Vissage}\ssindex{packages!Vissage}Vissage works in Figures \ref{Kfig1} and \ref{Kfig2}. 

\begin{figure}[tb]
  \centering
  \includegraphics[width=12cm]{part9/Kawasaki_P47/P047_f1.eps}
  \caption{Showing \ssindex{observatories!Earth-based!ALMA}ALMA images in a single \ssindex{applications!Vissage}\ssindex{packages!Vissage}Vissage window. The top left, bottom left and the bottom right images are integrated intensity maps with different colorsets. The top middle image is the same object as the top left one but is a mean velocity (1st moment) map. The top right is a channel map.}
  \label{Kfig1}
\end{figure}

\section{Future Plans}
As a latecomer to astronomical viewer software, we would like to make \ssindex{applications!Vissage}\ssindex{packages!Vissage}Vissage to be equipped with user-friendly interfaces as well as many useful functions, some of which might have been missed in the previous ones. Though the main keywords of the functionality of \ssindex{applications!Vissage}\ssindex{packages!Vissage}Vissage are 'easy handling data cubes' and 'tight connection with JVO', 'multi-wavelength' and 'multi-image' can also be added for future planning since data available from general\ssindex{Virtual Observatory(VO)} VOs cover a wide range of wavelength. The functions being implemented or planned for the near future are as follows: 

\begin{itemize}
  \item Connect with\ssindex{Virtual Observatory(VO)} VOs
    \begin{itemize}
      \item with JVO (other than \ssindex{observatories!Earth-based!ALMA}ALMA, including \ssindex{observatories!Earth-based!Subaru}Subaru, etc.)
      \item with other\ssindex{Virtual Observatory(VO)} VOs
    \end{itemize}
  \item Read \ssindex{data formats!FITS}FITS data from other telescopes than \ssindex{observatories!Earth-based!ALMA}ALMA
    \begin{itemize}
      \item \ssindex{observatories!space-based!Spitzer}Spitzer, \ssindex{observatories!space-based!Herschel}Herschel, \ssindex{observatories!Earth-based!Subaru}Subaru, \ssindex{observatories!space-based!HST}HST, \ssindex{observatories!space-based!Chandra}Chandra, etc.
    \end{itemize}
  \item Overlay multiple images and catalogues
  \item Contour
  \item View \ssindex{data formats!FITS}FITS data
    \begin{itemize}
      \item 2D view for arbitrary combination of cube axes
      \item 3D view, 1D view (spectra)
    \end{itemize}
  \item Annotation
  \item File export
\end{itemize}

\ssindex{applications!Vissage}\ssindex{packages!Vissage}Vissage is new software and still has only a limited number of functionalities, but we aim to evolve it to one of the favourite gadgets in astronomy in the future. 

\begin{figure}[tb]
  \centering
  \includegraphics[width=12cm]{part9/Kawasaki_P47/P047_f2.eps}
  \caption{Sample including P-V diagram. The middle image is an integrated intensity (0th moment) map of M100 with a P-V cut overlayed as a red area. The right images are P-V diagram of M100 together with a rotated original image. The top left, middle left and bottom left images are 0th, 1st and 2nd moment maps of the same object, respectively.}
  \label{Kfig2}
\end{figure}

\bibliography{editor}
