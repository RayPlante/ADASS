
\resetcounters

\bibliographystyle{asp2010}

\markboth{Kawasaki et al.}{Vissage: An ALMA-VO Desktop Application}

\title{Vissage: An ALMA-VO Desktop Application}
\author{Wataru~Kawasaki, Satoshi~Eguchi, Yuji~Shirasaki, Yutaka~Komiya, George~Kosugi, Masatoshi~Ohishi, 
and Yoshihiko~Mizumoto
\affil{National Astronomical Observatory of Japan, 2-21-1 Osawa, Mitaka, Tokyo, 181-8588, Japan}}

\aindex{Kawasaki, W.}
\aindex{Eguchi, S.}
\aindex{Shirasaki, Y.}
\aindex{Komiya, Y.}
\aindex{Kosugi, G.}
\aindex{Ohishi, M.}
\aindex{Mizumoto, Y.}

\begin{abstract}
Vissage is a FITS browser primarily targetting FITS data cubes obtained from 
ALMA. It currently offers some basic functionalities to view three-dimensional 
data cube and to connect with the JVO's ALMA data service. We will describe the 
present development status of Vissage, its aims and future plans.
\end{abstract}

\section{Introduction}
Having one year after the beginning of scientific operations based on the 
selected observing proposals, the Atacama Large Millimeter/submillimeter Array 
(ALMA) will soon be releasing data whose proprietary periods have expired in 
addition to Science Verification data that have already been out. 
The released ALMA data will be available from websites of the three ALMA 
Regional Centres located at East Asia, North America and Europe, and we are 
planning to distribute ALMA data via Japanese Virtual Observatory (JVO) as well. 
We then faced with two problems in distributing ALMA data in the framework of 
Virtual Observatory (hereafter VO) system, namely, huge size and high 
dimensionality. 

ALMA's unprecedented performance, especially when it gets to its full 
specifications in the near future, leads us to an obvious expectation that even 
a single data from ALMA could be as huge as Terabyte-scale. 
As for dimensionality, ALMA data will be provided as data cube with 3 or 4 
dimensions as typical radio imaging observation: spatial 2D and spectral 
1D, often plus polarisational 1D. Compared with ordinary 2D images, data cube 
has much wider variety in viewing ways including moment maps, channel maps, 
Position-Velocity diagram, etc. 

To cope with such huge datasets with high dimensionality in JVO, we have been 
developing a new mechanism and softwares \citep{O10_adassxxii, D5_adassxxii}. 
Our project covers both server-side and desktop side: the former includes the 
new functions of JVO ALMA portal including the ALMA Web Quick Look System 
(hereafter WebQL; \citet{O10_adassxxii}). 

As a solution for the latter, we are developing Vissage ({\it VISualisation 
Software for Astronomical Gigantic data cubEs}), a brand-new FITS browser. 
Its primary use case includes (1) viewing FITS data cube downloaded from JVO 
portal or WebQL to research on interesting regions and (2) connecting JVO 
seamlessly to search and download ALMA data for the same object but with a 
different spatial/spectral resolution from the one you are viewing to better 
fit your scientific purposes. 

\section{Current status of Vissage}
\subsection{Distribution Package}
Vissage is now available from the JVO website 
(\url{http://jvo.nao.ac.jp/download/Vissage}) as a tarball. It contains a 
front-end program (.exe) for Windows and a main program (executable jar file). 
The main program is implemented with Java/Swing, so it runs on all platforms 
with JRE (Java Runtime Environment) 6 or newer. Though it is directly 
executable, Windows users are strongly encouraged to launch it via the 
front-end program. The front-end for Windows platform enables you to open images 
by drag and drop onto a shortcut of Vissage and computes an appropriate heap 
size for Java Virtual Machine (the latter is critical especially when you use 
32-bit Java). 

\subsection{Available Functions}
Vissage now offers basic functions to show major two-dimensional views of 
data cube. Currently available functions include:
\begin{itemize}
  \item Open FITS data
    \begin{itemize}
      \item by drag and drop onto the main window (for all platforms)
      \item by drag and drop onto Vissage icon or shortcut (for Windows only)
    \end{itemize}
  \item Readable FITS files
    \begin{itemize}
      \item ALMA data
      \item Nobeyama 45m OTF data
    \end{itemize}
  \item Two-dimensional view of data cube
    \begin{itemize}
      \item integrated intensity map
      \item first/second moment maps
      \item flipbook (viewing images for each frequency channel)
      \item channel maps
      \item Position-Velocity diagram
    \end{itemize}
  \item Connect with JVO
    \begin{itemize}
      \item with ALMA Web QL
      \item with JVO ALMA portal via TAP (Table Access Protocol)
    \end{itemize}
  \item Image operation
    \begin{itemize}
      \item drag/pan images
      \item zoom-in/zoom-up
      \item change colorset/bias/contrast
      \item multiple images with flexible layout
    \end{itemize}
\end{itemize}

You can see how Vissage works in Figures \ref{Kfig1} and \ref{Kfig2}. 

\begin{figure}[tb]
  \centering
  \includegraphics[width=12cm]{part9/Kawasaki_P47/P047_f1.eps}
  \caption{Showing ALMA images in a single Vissage window. The top left, bottom left and the bottom right images are integrated intensity maps with different colorset. The top middle image is the same object as the top left one but is a mean velocity (1st moment) map. The top right is a channel map.}
  \label{Kfig1}
\end{figure}

\section{Future Plans}
As a latecomer to astronomical viewer softwares, we would like to make Vissage 
to be equipped with user-friendly interfaces as well as many useful functions, 
some of which might have been missed in the previous ones. 
Though the main keywords of the functionality of Vissage are 'easy handling 
data cubes' and 'tight connection with JVO', 'multi-wavelength' and 
'multi-image' can also be added for future planning since data available from 
general VOs cover wide range of wavelength. 
The functions being implemented or planned for the near future are as follows: 

\begin{itemize}
  \item Connect with VOs
    \begin{itemize}
      \item with JVO (other than ALMA including Subaru etc.)
      \item with other VOs
    \end{itemize}
  \item Read FITS data from other telescopes than ALMA
    \begin{itemize}
      \item Spitzer, Herschel, Subaru, HST, Chandra, etc.
    \end{itemize}
  \item Overlay multiple images and catalogues
  \item Contour
  \item View FITS data
    \begin{itemize}
      \item 2D view for arbitrary combination of cube axes
      \item 3D view, 1D view (spectra)
    \end{itemize}
  \item Annotation
  \item File export
\end{itemize}

Vissage is just a new software and still has only limited number of 
functionalities, but we aim to evolve it to one of the most favourite gadgets 
in astronomy in the future. 

\begin{figure}[tb]
  \centering
  \includegraphics[width=12cm]{part9/Kawasaki_P47/P047_f2.eps}
  \caption{Sample including P-V diagram. The middle image is an integrated intensity (0th moment) map of M100 with a P-V cut overlayed as a red area. The right images are P-V diagram of M100 together with a rotated original image. The top left, middle left and bottom left images are 0th, 1st and 2nd moment maps of the same object, respectively.}
  \label{Kfig2}
\end{figure}

\bibliography{editor}
