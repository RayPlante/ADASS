
\resetcounters

\bibliographystyle{asp2010}

\markboth{Berry et al.}{Starlink 2012 - The Kapuahi Release}

\title{ 2012 - The Kapuahi Release}

\author{David~Berry,$^1$ Malcolm~Currie,$^1$ Tim~Jenness,$^1$ Peter~Draper,$^2$ Graham~Bell,$^1$ and Remo~Tilanus$^1$
\affil{$^1$Joint Astronomy Centre, Hilo, Hawaii}
\affil{$^2$Dept. of Physics, University of Durham, UK}}

\aindex{Berry, D.}
\aindex{Currie, M.}
\aindex{Jenness, T.}
\aindex{Draper, P.}
\aindex{Bell, G.}
\aindex{Tilanus, R.}

\begin{abstract}
We present details of the most recent release of the \ssindex{packages!Starlink}Starlink\ooindex{Starlink, ascl:1110.012} Software
Collection, code-named "Kapuahi".
\end{abstract}

\section{What is \ssindex{packages!Starlink}Starlink?}

\ssindex{packages!Starlink}Starlink\ooindex{Starlink, ascl:1110.012} (\url{www.starlink.ac.uk}) is an open-source project hosted on \ssindex{code!repository!GitHub}github that provides a wide range of astronomical applications and associated infrastructure libraries. From an initial start in 1980, the software has been in development and use ever since, and is now supported by the Joint Astronomy Centre, Hawaii \citep{SSC_2009}.

\subsection{Selected Applications}
\begin{description}
\item[\ssindex{applications!GAIA}GAIA] An interactive display and analysis tool for use with images and cubes (\url{star-www.dur.ac.uk/~pdraper/gaia/gaia.html}).
\item[SPLAT] An interactive spectral display and analysis tool (\url{star-www.dur.ac.uk/~pdraper/splat/splat.html}).
\item[ORAC-DR] An automated data processing pipe-line system
(\url{www.oracdr.org}).
\item[KAPPA] Nearly 200 general purpose commands for the display and manipulation of N-dimensional arrays.
\item[CCDPACK] Commands for the reduction of CCD data.
\item[POLPACK] Commands for the creation and display of \ssindex{astronomy!polarization}polarisation vectors from arrays of analysed intensity.
\item[SMURF] Sub-mm data reduction facilities for the James Clerk Maxwell Telescope.
\item[CUPID] Provides implementations of various clump-finding algorithms such as \ssindex{algorithm!ClumpFind}ClumpFind\ooindex{Clumpfind, ascl:1107.014}, \ssindex{algorithm!GaussClumps}GaussClumps, and others, together with other supporting commands.
\item[ATOOLS] A command-line interface to the \ssindex{World Coordinate System (WCS)}WCS management functions provided by the \ssindex{libraries!AST}AST library.
\item[CONVERT] Commands for converting to and from the \ssindex{packages!Starlink}Starlink\ooindex{Starlink, ascl:1110.012} NDF data format. Several foreign data formats are supporting, including \ssindex{data formats!FITS}FITS.
\end{description}

\subsection{Selected Infra-structure Libraries}
\begin{description}
\item[\ssindex{libraries!AST}AST] Provides a high-level object-oriented approach to handling spatial, spectral and temporal \ssindex{World Coordinate System (WCS)}WCS meta-data, with facilities for handling \ssindex{data formats!STC-S}STC-S format and a wide range of \ssindex{data formats!FITS}FITS formats
(\url{www.starlink.ac.uk/ast}).

\item[NDF] The main data access library for \ssindex{packages!Starlink}Starlink-implements an extensible, hierarchical astronomical data model. Native NDF files use the HDS format, but the other formats such as \ssindex{data formats!FITS}FITS can also be read.
\item[\ssindex{libraries!PAL}PAL] Provides low-level functions for manipulating celestial coordinate systems-a re-write in\ssindex{computer languages!C} C of the earlier \ssindex{computer languages!Fortran}Fortran \ssindex{libraries!SLALIB}SLALIB library, using the \ssindex{libraries!SOFA}SOFA library.
\item[PCS] A collection of libraries that provide the integrated messaging and parameter system upon which the \ssindex{packages!Starlink}Starlink\ooindex{Starlink, ascl:1110.012} environment is built.
\item[HDS] A low-level hierarchical data format.
\end{description}

\section{The Kapuahi Release - What's New?}
Released in September 2012, the current release of the \ssindex{packages!Starlink}Starlink\ooindex{Starlink, ascl:1110.012} Software Collection has the code name ``Kapuahi'' - the Hawaiian name for Aldebaran. This section lists a few of the more important changes since the previous release (``Namaka''). The development version adds support for data values stored as 64-bit integers.

\subsection{Global Changes}
\begin{itemize}
\item Mutexes are used to serialise access to \ssindex{computer languages!Fortran}Fortran subroutines from within\ssindex{computer languages!C} C code, making it safe to call them within a thread.
\item All use of the \ssindex{libraries!SLALIB}SLALIB library within\ssindex{computer languages!C} C code has been replaced by the new \ssindex{libraries!PAL}PAL library \citep{P56_adassxxii}. \ssindex{computer languages!Fortran}Fortran code still uses \ssindex{libraries!SLALIB}SLALIB.
\end{itemize}

\subsection{Application Changes}

\subsubsection{ATOOLS}
All commands now allow the output WCS information to be created in various textual formats, including\ssindex{data formats!AST} AST, \ssindex{data formats!XML}XML, \ssindex{data formats!STC-S}STC-S, or various \ssindex{data formats!FITS}FITS header formats.

\subsubsection{CUPID}
\begin{itemize}
\item \ssindex{algorithm!GaussClumps}GaussClumps now excludes peaks below the specified threshold from the output catalogue and NDF.
\item \ssindex{algorithm!GaussClumps}GaussClumps has improved its ability to decompose merged clumps.
\item FINDBACK now uses multiple processors or cores (if available) to speed up the the processing of independent slices.
\item FINDBACK can now find and use an independent default noise value for each slice of supplied NDF.
\end{itemize}

\subsubsection{GAIA}
\begin{itemize}
\item A snapshot can now be taken of the whole image, including the off-screen parts, so it is now possible to capture a graphic of the whole image at full or higher resolution when it is larger than the main window.
\item The "Intensity Map:" selector in the main panel has been replaced by a "Colour Scale:" selector, which gives more careful transformation of intensity into log-like space, reducing the false contouring seen in low intensities.
\item The Coords menus in the cube and spectral display now also offers a selection of standards of rest.
\item Several issues with the axes plotting in the three-dimensional visualisation toolboxes have been fixed. There is a new free-form entry area for\ssindex{data formats!AST} AST attributes so that the axes can be configured.
\end{itemize}

\subsubsection{KAPPA}
\begin{itemize}
\item HISTOGRAM now allows weights to be associated with each input pixel value.
\item MFITTREND now permits processing of very large datasets.
\item PASTE now allows a constant shift of origin between successive pasted data. For instance, this can be used to paste a set of 2D images into a \ssindex{spectroscopy!3D}3D cube.
\item SCATTER now displays the Pearson coefficient of the data in the scatter plot, and writes it to a parameter.
\item SETAXIS can now create axes describing a linear approximation to the \ssindex{World Coordinate System (WCS)}WCS in the supplied NDF.
\item SQORST now has an option to conserve flux.
\item The REGRID, WCSALIGN, and WCSMOSAIC commands now have more control over the details of how (as well as if) flux is conserved.
\end{itemize}

\subsubsection{SMURF}
\begin{itemize}
\item Many improvements have been made that allow better maps to be created from \ssindex{instruments!individual!SCUBA}SCUBA-2 data (\url{\ssindex{instruments!individual!SCUBA}scuba2.wordpress.com}).
\item FIT1D is a new command that fits profiles to spectral lines.  It can fit complex spectra with multiple components and non-Gaussian shapes.
\end{itemize}


\subsection{Library Changes}

\subsubsection{AST}
\begin{itemize}
\item Now has a \ssindex{computer languages!Python}Python interface-\ssindex{libraries!PyAST}PyAST \citep{P011_adassxxi}. It is hosted on \ssindex{code!repository!GitHub}github (\url{github.com/timj/starlink-pyast}).
\item Now included in Fedora and Debian repositories.
\item Now based on the \ssindex{libraries!SOFA}SOFA and \ssindex{libraries!PAL}PAL libraries rather than \ssindex{libraries!SLALIB}SLALIB.
\item Extended support for a variety of distorted \ssindex{data formats!FITS}FITS projections including SIP, TPV, TNX, ZPX, SAO, and SCAMP\ooindex{SCAMP, ascl:1010.063}.
\item Boundaries of complex Regions are now plotted much faster.
\end{itemize}

\subsubsection{NDF}
\begin{itemize}
\item Now has a \ssindex{computer languages!Python}Python interface-PyNDF (\url{github.com/timj/starlink-pyndf}).
\item Problems with the interpretation of NDF sections specified in \ssindex{World Coordinate System (WCS)}WCS coordinates have been fixed.
\item Foreign format NDFs may now have dots in the file base-name.
\end{itemize}

\subsubsection{PAL}
A new\ssindex{computer languages!C} C library that provides the functionality of the older \ssindex{computer languages!Fortran}Fortran \ssindex{libraries!SLALIB}SLALIB library through use of equivalent \ssindex{libraries!SOFA}SOFA routines where available, or by a re-writing of the \ssindex{libraries!SLALIB}SLALIB \ssindex{computer languages!Fortran}Fortran code otherwise.

\subsubsection{THR}
A new library that provides facilities to manage pools of persistent threads, allowing jobs to be run in parallel from within\ssindex{computer languages!C} C applications.

\section{The \ssindex{packages!Starlink}Starlink \ssindex{code!repository!GitHub}Github Repository}
The \ssindex{packages!Starlink}Starlink\ooindex{Starlink, ascl:1110.012} \ssindex{code!repository!Git}git repositories have been moved from the Joint Astronomy Centre, Hawaii, and are now hosted on \ssindex{code!repository!GitHub}github (\url{github.com/Starlink/starlink}).

\section{Obtaining \ssindex{packages!Starlink}Starlink Software}
The \ssindex{packages!Starlink}Starlink\ooindex{Starlink, ascl:1110.012} Software Collection may be obtained in three ways:
\begin{itemize}
\item Downloading a pre-built Kapuahi release. These are available for both Linux and OS X (32 and\ssindex{computing!architecture} 64 bit). See the \ssindex{packages!Starlink}Starlink\ooindex{Starlink, ascl:1110.012} home page for instructions.
\item Copying the current cutting-edge build from the Joint Astronomy Centre, using rsync. Again, see the \ssindex{packages!Starlink}Starlink\ooindex{Starlink, ascl:1110.012} home page for instructions.
\item Checking out the source files from the \ssindex{code!repository!GitHub}github repository and building from source. See the README file in the checkout root directory for instructions.
\end{itemize}


\bibliography{editor}
