
\resetcounters

\markboth{Tibbetts et al.}{Near Earth Object Data Discovery and Query}

\title{NEOview: Near Earth Object Data Discovery and Query}
\author{M.~Tibbetts,$^1$ M.~Elvis,$^1$ J.~L.~Galache,$^1,2$ P.~Harbo,$^1$ J.~C.~McDowell,$^1$ M.~Rudenko,$^1,2$ D.~Van~Stone,$^1$ and P.~Zografou$^1$
\affil{$^1$Smithsonian Astrophysical Observatory, 60 Garden St, Cambridge, MA 02138, USA}
\affil{$^2$Minor Planet Center, 60 Garden St, Cambridge, MA 02138, USA}}

\aindex{Tibbetts, M.}
\aindex{Elvis, M.}
\aindex{Galache, J. L.}
\aindex{Harbo, P.}
\aindex{McDowell, J. C.}
\aindex{Rudenko, M.}
\aindex{Van Stone, D.}
\aindex{Zografou, P.}

\begin{abstract}
Missions to Near Earth Objects (NEOs) figure prominently in NASA's "Flexible Path" approach to human space exploration. NEOs offer insight into both the origins of the Solar System and of life, as well as a source of materials for future missions. With NEOview scientists can locate NEO datasets, explore metadata provided by the archives, and query or combine disparate NEO datasets in the search for NEO candidates for exploration.  NEOview is a software system that illustrates how standards-based interfaces facilitate NEO data discovery and research.  NEOview software follows a client-server architecture. The server is a configurable implementation of the International Virtual Observatory Alliance (IVOA) Table Access Protocol (TAP), a general interface for tabular data access, that can be deployed as a front end to existing NEO datasets. The TAP client, \textit{seleste}, is a graphical interface that provides intuitive means of discovering NEO providers, exploring dataset metadata to identify fields of interest, and constructing queries to retrieve or combine data. It features a powerful, graphical query builder capable of easing the user's introduction to table searches.  Through science use cases, NEOview demonstrates how potential targets for NEO rendezvous could be identified by combining data from complementary sources. Through deployment and operations, it has been shown that the software components are data independent and configurable to many different data servers. As such, NEOview's TAP server and \textit{seleste} TAP client can be used to create a seamless environment for data discovery and exploration for tabular data in any astronomical archive.
\end{abstract}

\section{Introduction}
Near Earth Objects (NEOs) figure prominently in NASA's Flexible Path approach to human space exploration\citep{flexible_path}.  Missions to NEOs offer not only insight into both the origins of the Solar System and of life, but also offer a source of materials for future missions.  The NEOview\footnote{\url{http://neo.cfa.harvard.edu/neoview.html}} software suite illustrates how standards-based interfaces can facilitate NEO data discovery and research.  With the NEOview framework, scientists can locate NEO datasets, explore metadata provided by the archives, and query or combine disparate NEO datasets in the search for NEO candidates for exploration.  What follows is a brief overview of the architecture of the NEOview client-server applications.

\section{Architecture}
NEOview software consists of two components, server and client, that rely on the Table Access Protocol (TAP) Recommendation\footnote{\url{http://www.ivoa.net/Documents/TAP/20100327/}} from the International Virtual Observatory Alliance (IVOA).  TAP is a RESTful web service protocol that gives access to both tabular data and metadata that describes the data and relationships between tables.

\subsection{Server}
The TAP server is a Java Platform, Enterprise Edition (JEE), application consisting of a servlet handling user requests, a messaging service to queue jobs, and a database interface to access service data and state information.  The server architecture, shown in Figure \ref{server_architecture}, supports any JEE compliant server combined with any SQL compliant relational database management server (RDBMS).  In NEOview, production TAP servers were deployed in front of an existing archive at the Minor Planet Center\footnote{\url{http://www.minorplanetcenter.net/}} (MPC) at SAO as well as in front of a project-specific archive constructed as a proof of concept by combining data from other NEO data providers.  Each service instance utilizes a different combination of JEE and RDBMS servers, allowing service administrators to target the service deployment to their preferred JEE and RDBMS platforms.

\articlefigure{part9/Tibbetts_P030/P030_fig1}{server_architecture}{TAP Server Application Architecture}

\subsection{Client}
The TAP client, \textit{seleste}, is a graphical interface that provides intuitive means of discovering NEO data providers, exploring dataset metadata to identify fields of interest, and constructing queries to retrieve or combine data.  The architecture, shown in Figure \ref{client_architecture}, consists of a graphical interface layered upon an application programming interface.  Through the use of TAP and additional IVOA interfaces, such as Registry Interfaces\footnote{\url{http://www.ivoa.net/Documents/RegistryInterface/20091104/}} and Simple Application Messaging Protocol\footnote{\url{http://www.ivoa.net/Documents/SAMP/20120411/index.html}}, the TAP client integrates into the end user's work flow for discovering, exploring, querying and combining datasets.
\articlefigure{part9/Tibbetts_P030/P030_fig2}{client_architecture}{TAP Client Application Architecture}

Building on successes with a similiar application\citep{cscview}, the TAP client \textit{seleste}, shown in Figure \ref{screenshot}, consists of a query builder and a job manager; the TAP client provides an intuitive interface to ease the user's introduction to RDBMS table searches.  It also provides TAP metadata the describes the data and the relationship amongst tables.  The job manager allows the user to manage and view both queries and results across multiple services.  Users can check the status of running queries, view the results of completed queries, and feed previous results back into new queries to combine datasets. 

\articlefigure{part9/Tibbetts_P030/P030_fig3}{screenshot}{TAP Client Graphical User Interface, \textit{seleste}}

\section{Summary}
Through science use cases, NEOview demonstrates how potential targets for NEO rendezvous could be identified by combining data from complementary sources: orbits from the MPC at SAO, lightcurves from the Asteroid Lightcurve Database at the Palmer Divide Observatory and spectra from the MIT-UH-IRTF Joint Campaign. Through deployment and operations, it has also shown that the software components are data independent and configurable to many different data servers. As such, NEOview's TAP server and \textit{seleste} TAP client can be used to create a seamless environment for data discovery and exploration for tabular data in any astronomical archive.  In the future, we hope to deploy TAP server interfaces to additional NEO data archives.

\acknowledgements Supported by the Smithsonian Institution's Atherton Seidell Grant for Dissemination of Previously Published Scientific Research, the Chandra X-ray Center, operated by the Smithsonian Astrophysical Observatory for and on behalf of the National Areonautics and Space Administration under contract NAS8-03060, and the U.S. Virtual Astronomical Observatory, which is sponsored by the National Science Foundation and the National Aeronautics and Space Administration.

\bibliographystyle{asp2010}
\bibliography{editor}
