
\resetcounters

\bibliographystyle{asp2010}

\markboth{Jenness and Berry}{PAL -- A Positional
  Astronomy Library}

\title{PAL -- A Positional Astronomy Library}
\author{Tim~Jenness and David~S.~Berry
\affil{Joint~Astronomy~Centre, 660~N.~A`oh\={o}k\={u}~Place, Hilo, HI, 96720, U.S.A.}}

\aindex{Jenness, T.}
\aindex{Berry, D. S.}

\begin{abstract}
  PAL is a new positional astronomy library written in C that attempts
  to retain the SLALIB API but is distributed with an open source GPL
  license. The library depends on the IAU SOFA library wherever a SOFA
  routine exists and uses the most recent nutation and precession
  models. Currently about 100 of the 200 SLALIB routines are
  available. Interfaces are also available from Perl and Python. PAL
  is freely available via github.
\end{abstract}

\section{Introduction}

The SLALIB library, written by Patrick Wallace
\citep{1994ASPC...61..481W}, is a very popular positional astronomy
library that is available in Fortran as part of the Starlink Software
Collection \citep[e.g.][]{2009ASPC..411..418J} with a GPL license. An
updated version is also available directly from the author as a
proprietary and extended C library with a non-commercial
non-redistribution license. The latter clause makes it hard to ship
software relying on the C library and impossible to include the code
in a public source code repository.

To overcome the distribution issues and to make use of current
precession and nutation models, we have written, with advice and input
from Patrick Wallace, the PAL library. The PAL library is written in C
and has a GPL license. In most cases the API is designed to be
identical to SLALIB except for the use of a \texttt{pal} prefix instead of a
\texttt{sla} prefix in function calls. Where appropriate, IAU SOFA routines
\citep{Hohenkerk:2010,1996ASPC..101..207W}, of which Patrick Wallace
was the primary author, are called. Where the SOFA library does not
have equivalent functionality the Fortran has been ported to C from
the GPL Fortran library included with the Starlink distribution.

We have not ported the full SLALIB functionality to PAL but are adding
routines as we need them for applications. Approximately 100 functions
have been ported.

\section{Implementation}

PAL is written in very portable C with SOFA as the only
dependency. Perl and Python wrappers have also been written and they
are distributed with their own copies of PAL and SOFA to make
installation as easy as possible.

The example code below shows the deliberate similarities between the
SLALIB and PAL API:

\newpage
\begin{description}

\item[SLALIB Fortran:] \mbox{}

\begin{verbatim}
GMST = SLA_GMST( UT1 )
CALL SLA_DE2H( HIN, DIN, DP, DA, DE )
CALL SLA_DMOON( DATE, PV )
CALL SLA_OBS( N, C, NAME, W, P, H )
\end{verbatim}

\item[SLALIB C:] \mbox{}

\begin{verbatim}
gmst = slaGmst( ut1 );
slaDe2h( hin, din, dp, &da, &de );
slaDmoon( date, pv );
slaObs( 0, "JCMT", telname, &w, &p, &h );
\end{verbatim}

\item[PAL C:] \mbox{}

\begin{verbatim}
gmst = palGmst( ut1 );
palDe2h( hin, din, dp, &da, &de );
palDmoon( date, pv );
status = palObs( 0, "JCMT", short, slen,
                 long, llen, &w, &p, &h );
\end{verbatim}

\item[PAL Python:] \mbox{}

\begin{verbatim}
import palpy as pal
gmst = pal.gmst( ut1 )
(da, de) = pal.de2h( hin, din, dp )
pv = pal.dmoon( date )
obsdata = pal.obs()
mmt = obsdata["MMT"]
\end{verbatim}

\item[PAL Perl:] \mbox{}

\begin{verbatim}
use Astro::PAL qw/ :all /;
$gmst = palGmst( $ut1 );
($da,$de) = palDe2h( $hin, $din, $dp );
@pv = palDmoon( $date );
($id, $name, $w, $p, $h) = palObs( 22 );
($id, $name, $w, $p, $h) = palObs( "JCMT" );

\end{verbatim}

\end{description}

In general a simple renaming of the SLALIB function will be sufficient
to switch to PAL with one exception. As can be seen above the
\texttt{palObs} routine now expects to be given the lengths of
supplied string buffers and does not reuse the short name buffer,
allowing const strings to be supplied without triggering compiler
warnings.

The Python and Perl interfaces return results where appropriate rather
than modifying arguments. The Python interface uses numpy arrays
\citep[e.g.][]{Walt2011}\footnote{\url{http://numpy.scipy.org}} for
  all vectors and matrices, and the C library is wrapped using
  Cython\footnote{\url{http://cython.org}}. The Perl interface is
  wrapped with the standard perl/XS system \citep[e.g.][]{JennessBook} and
  uses simple lists for vectors and matrices rather than adding a
  dependency on the Perl Data Language \citep{PDL}.

The Fortran test suite was ported to C to test the PAL library. There
are minor changes due to differences in the more modern precession
and nutation models implemented in SOFA.

The PAL library is now used within the Starlink AST library
\citep{2012ASPC..461..825B} and in all Starlink C applications that
previously used SLALIB. It was shipped with the Starlink
\textit{kapuahi} release \citep{P005_adassxxii}.

The PAL library also has an advantage over Fortran SLALIB in that it
is inherently thread-safe.


\section{Obtaining the Software}

PAL is available from
github\footnote{\url{https://github.com/Starlink/pal/downloads}} and
is also distributed with
Starlink\footnote{\url{http://www.starlink.ac.uk}}.  The Python and
Perl wrappers are also on
github\footnote{\url{https://github.com/Starlink/palpy} and
  \url{https://github.com/timj/perl-Astro-PAL}} and distributions can
be obtained from
PyPI\footnote{\url{http://pypi.python.org/pypi/palpy}} and
CPAN\footnote{\url{http://search.cpan.org/~tjenness/Astro-PAL/}}.

\bibliography{editor}
