% This is the aspauthor.tex LaTeX file
% Copyright 2010, Astronomical Society of the Pacific Conference Series



% \newcommand{\etal{{\it et. al}}}
\newcommand{\etal}{{\it et. al}}


\documentclass[11pt,twoside]{article}
\usepackage{asp2010}

\usepackage{savesym}
\savesymbol{iint}
\savesymbol{iiint}
\savesymbol{iiiint}
\savesymbol{idotsint}
\usepackage{amsmath}
\restoresymbol{TXF}{iint}
\restoresymbol{TXF}{iiint}
\restoresymbol{TXF}{iiiint}
\restoresymbol{TXF}{idotsint}
\resetcounters

\bibliographystyle{asp2010}

\markboth{Teuben, Ip, Mundy and Varshney}{Science Mining and Characterization of ALMA Large Data Cubes}


\begin{document}

\title{Science Mining and Characterization of ALMA Large Data Cubes}
\author{Peter Teuben$^1$, Cheuk Yiu Ip$^2$, Lee Mundy$^1$, and Amitabh Varshney$^2$
\affil{$^1$Astronomy Department, University of Maryland, College Park}
\affil{$^2$Institute for Advanced Computer Studies, University of Maryland, College Park}}

\begin{abstract}

We are using Multilevel Segmentation of the Intensity-Gradient
Histograms and Description Vectors to show unique ways to visualize
and analyse complex structures in large ALMA data cubes. In particular
higher dimensional data cubes with many spectral lines are a challenge
both algorithmically and visually. In this poster we show some examples
of both theoretical and observational data cubes and how novel
techniques developed outside of our field can be applied to Astronomy.



\end{abstract}



\section{Introduction}

Discuss large datasets, need for automatically finding interesting things
in those data, extracting them via description vectors.

\section{Procedure}

This procedure effectively reduces millions of pixels to hundreds or
tens of parameterized objects.  We present an automatic algorithm that
identify salient regions from a radio astronomy image and we
parameterized and index these regions by fitting ellipses to the
identified regions.  Our algorithm is inspired by the computational
saliency model which detects regions of interest from images.
{\it this still needs (optionally) to be extended to 3D in our case of
velocity datacubes}

A saliency map shows which part of an image is likely to attract the
most attention of the low-level human visual system.
Itti~\etal~\cite{itti98:_model_of_salien_based_visual} have proposed a
computational model of visual saliency by using multiscale image
processing.  Multiscale image processing techniques analyze an image
at different scales to simulate the retinal receptive fields.  Their
image saliency model aggregates the results from three features of an
image -- intensity, color opponencies, and orientation. For this
astronomy application we only consider the signal intensity for the
computing image saliency.

We next briefly review the traditional algorithm for computing a
saliency map of an image.

\begin{align}
  \label{eqn:sal}
  F  &\leftarrow \mathrm{Image}, \notag\\
  G_{j} &= \mathcal{G}(j) \otimes F, \quad j
  \in{\sigma,2\sigma,4\sigma,8\sigma,16\sigma \cdots}\notag\\
  D_{j,k} &= |G_{i,j} - G_{i,k}|, \quad k \in \{4j,8j\}\notag\\
\end{align}


We convolve the images $F$
with Gaussian kernels, $\mathcal{G}$, at different scales $j$.  We
find contrasting regions by computing the difference of Gaussians
($DoG$) images at each scale, $G_{j}$.  We compute the $DoG$ images
at scales $\{\sigma,4\sigma\}, \{\sigma,8\sigma\}$.  The $DoG$
operation mimics the contrast detecting receptive fields of retinal
ganglion cells.  Fig <the car DoG figure> shows the $DoG$ operation
extracts contrasting cars from the background.  Normally a
normalization procedure is applied to extract the most salient feature
from the image.  However, as we are interested in peak regions, in
addition to the dominating peak, we skipped this peak promotion
normalization.  We also analyze these $DoG$ map separately.

We threshold the $DoG$ images to extract the salient pixels and we
group these salient pixels into segments.  We first suppress pixels
with low signal by thresholding.  Then, we segment these $DoG$ images
by finding the connected components with the region growing method.
We perform this thresholding and pixel grouping operation at all
scales of the image.

For each segment, we fit ellipses to the pixels by using principal
component analysis.  The center of the ellipse is given by the center
of mass of the pixels.  The angle, major, and minor axes can be
computed by decomposing the covariance matrix of the pixels locations.
We can use these parameters along with saliency measures at different
scales to index each of these segments.

\section{testing}


this is for testing, not to be included in final version.



\section{The Template}
To fill in this template, make sure that you read and follow the ASPCS Instructions for Authors and Editors available for download online.  Hints and tips for including graphics, tables, citations, and other formatting helps are available there.

\subsection{The Author Checklist}
The following checklist should be followed when writing a submission to a conference proceedings to be published by the ASP.

\begin{itemize}
\checklistitemize
\item Article is within page limitations set by editor. 
\item Paper compiles properly without errors or warnings.
\item No fundamental modifications to the basic template are present, including special definitions, special macros, packages, \verb"\vspace" commands, font adjustments, etc. %(� 3.3, p. 10)
\item Commented-out text has been removed. %(� 3.3, p. 10,11)
\item Author and shortened title running heads are proper for the paper and shortened so page number is within the margin. %(� 3.1, p. 4)
\item Paper checked for general questions of format and style, including, but not limited to, the following:
\begin{itemize}
  \item capitalization, layout, and length of running heads, titles  and \\sections/subsections;  % (� 3.1, p. 4) (� 3.2, p. 5) (� 3.3, p. 8)
  \item page numbers within margin; % (� 3.1, p. 4)
  \item author names spelled correctly and full postal addresses given; % (� 3.2, p. 5-6)
  \item abstracts; % (� 3.2, p. 7);
  \item all margins---left, right, top and bottom; % (�3.1, p. 4; �3.2, p. 5; �3.3, p. 9; �3.6, p. 21);
  \item standard font size and no Type 3 fonts; %(� 3.3, pp. 10-11; � 3.6, p. 23; � 4.2, p. 25);
  \item spacing; % (� 3.3, pp. 9-10);
  \item section headings. % (� 3.3; p. 8).
\end{itemize}
\item All tables are correctly positioned within margins, are properly formatted, and are referred to in the text.  %(� 3.5, pp. 16-20)
\item All figures are correctly positioned within margins, are minimum 300 dpi resolution, not too dark or too light, do not contain embedded fonts, and are referred to in the text.  All labeling or text will be legible with 10\% reduction.  Questionable images printed, checked and replaced if necessary.  Figures do not cover text or running heads, and proper permissions have been granted and acknowledged.  %(� 3.6, pp. 21-24)
\item All acknowledgments and discussions are in proper format.  % (pp. 11-12, p. 20)
\item If there are acknowledgments at the end of the article, ensure that the author has used
the \verb"\acknowledgments" command and not the commands
\\ \verb"\begin{Acknowledgments}", \verb"\end{Acknowledgments}".
Acknowledgments should only be used for thanking institutions,
groups, and individuals who have directly contributed to the work.
\item All references quoted in the text are listed in the bibliography; all items in the bibliography have been referred to in the text.  % (� 4, pp. 24-28)
\item All bibliography entries are in the proper format, using one of the referencing styles given. Each of the references is bibliographically complete, including full names of authors, editors, publishers, place of publication, page numbers, years, etc. % (�� 4.2-4.3, pp. 25-28)
\item A complete Bib\TeX\ file is ready to submit to the editor.
\item References to preprints replaced with publication information when possible.
\end{itemize}


\acknowledgements The ASP would like to the thank the dedicated researchers that are publishing with the ASP.

\bibliography{P059}

\end{document}
