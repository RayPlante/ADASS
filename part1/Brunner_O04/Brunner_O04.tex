
\resetcounters


\markboth{Brunner}{Practical Informatics}


\title{Practical Informatics: Training the Next Generation}
\author{Robert J.~Brunner}
\affil{Department of Astronomy, University of Illinois}

\aindex{Brunner, R. J.}

\begin{abstract}
A commonly discussed yet infrequently addressed problem in the scientific community is the inadequate training our students receive in dealing with large data, a subject more popularly known as informatics. Yet as presented by the late Jim Gray, we now have a fourth paradigm for scientific research, namely data intensive science. Over the last few years, I have tried to address this educational deficiency at the University of Illinois at Urbana-Champaign. Initially, I added relevant informatics content into the standard Astronomy curricula in order to increase the student's exposure to this new paradigm. Realizing that this was merely a band-aid solution, I next created and offered a new course, entitled \textit{Practical Informatics for the Physical Sciences} that was warmly received by undergraduate and graduate students in several science and engineering disciplines. More recently, I have been tasked by the University with expanding this material into an online course to introduce informatics concepts and techniques to a wider audience.

In this paper, I present my initial motivation for adopting informatics material into the Astronomy curricula, my thoughts and experiences in developing the Practical Informatics course, lessons learned from the entire process, and my progress in developing a new, online informatics course. I hope that others can make use of these lessons to more broadly improve the training of the next generation of scientists.
\end{abstract}

\section{Background}
Many in the scientific community have witnessed the growing disparity between what our students learn in the classroom and the practical knowledge they need to work with large data. Not only this is a problem when they begin their formal research, but it is also a problem if they decide to leave academia as they will be lacking vital skills now in demand in the corporate world~\citep[see, \textit{e.g.},][]{loukides}. Given the overproduction of students in many scientific disciplines, this latter point is an important one that is too often overlooked. Clearly, we face a real and growing problem.

Responding to this challenge is not simple, however, as there are several hurdles we face. First, departments have well established (and necessary) curricula that form the basis for a student's education. This required course load does not provide many new paths for additional instruction. Second, many departments do not have sufficient instructors to allow someone to create and teach a new informatics course. And finally, assuming the first two can be overcome, what exactly should be taught in such a course?

I first tackled this challenge in 2010, when I added informatics content to an existing course within our undergraduate curriculum. I have taught this course two additional times in the interim, and I summarize my experiences in \S~\ref{r1}. Realizing that the need went beyond our undergraduate students, I subsequently created and taught a new dual level course, which is discussed in \S~\ref{r2}. Currently, I am developing an entirely new online course for the informatics program at the University of Illinois, which is briefly presented in \S~\ref{r3}.

\section{Round One~\label{r1}}

The simplest technique to address the lack of data skills our students were acquiring was to simply add relevant material to the existing curricula. In the Fall of 2010,  I was scheduled to teach \textit{ASTR 406: Galaxies and the Universe}, an advanced undergraduate course covering extragalactic astronomy. Given the nature of this course, I decided that the best method for adding new informatics material was through homework assignments. Previous assignments included topics like luminosity functions, surface brightness profiles, and stellar distribution models. Using these as guides, I replaced select analytic problems with new, related informatics problems.

\articlefiguretwo{part1/Brunner_O04/m86pixs.eps}{part1/Brunner_O04/4-8-z.eps}{class}{Visualizations from the solutions to two informatics problems from \textit{ASTR 406: Galaxies and the Universe}. (Left) The pixels above a threshold for M86 with the axial ratios over plotted. (Right) A top-down view of a three-dimensional model of an exponential disk with different parameters for the different stellar types.}

First, the students had to download sources within different areas on the sky from the Sloan Digital Sky Survey Catalog Archive Server, by issuing SQL queries. With these data, they wrote a Python program to classify stars and galaxies and subsequently visualize these classified sources to see the difference between the distribution of stars and galaxies at high and low Galactic latitudes. Another task required the students to read a FITS image of M86 obtained from the NASA Extragalactic Database into a Python program. In subsequent problems, the students would (a) perform thresholding to find pixels in the image belonging to the object, (b) find the center of the pixel distribution, (c) compute the flux inside a series of concentric apertures, thereby creating a surface brightness plot, and (d) compute the axial ratio of the pixel distribution (see the left panel of Figure~\ref{class}). A third task was to write a Python program to create and visualize a three-dimensional model of a disk galaxy. This model is built by using a double exponential stellar distribution with different scale heights and scale lengths for different stellar types. By changing the glyph size and color for the different stellar types the visualization clearly shows how the stellar distribution varies for the different stellar types in both the radial and vertical dimensions (see the right panel of Figure~\ref{class}).

I have now taught this course three times and have made clarifications and improvements based on student feedback. First, the main problem students had was to simply get the Python libraries installed onto their computer. To simplify this process, I have created a virtual appliance by using Oracle's VirtualBox that has all necessary software pre-installed. Second, since a programming class has not been a pre-requisite to this class, I now provide template code for each problem that are copied from the solutions. During this process, I removed a few lines of code that provide key functionality, and I replaced this code segment with a comment that summarized the missing code and specified how many lines were removed. Finally, I moved the informatics problems to a separate assignment so that they are decoupled from the rest of the class, allowing the students more time to complete each informatics assignment.

\section{Round Two~\label{r2}}

While student responses indicated my initial efforts were generally successful, I felt that I was only scratching the surface, as our students were still missing basic data science skills. Furthermore, I believed that the students would welcome the opportunity to acquire these skills if given a suitable opportunity. As a result, in the Fall of 2011 I created a new, dual level undergraduate/graduate course,  \textit{ASTR 496/596: Practical Informatics for the Physical Sciences}, that I taught as an overload. In the end, enrollment was better than expected with both undergraduate and graduate students participating, most of whom were from Astronomy. But a surprisingly large number were from Atmospheric Sciences, Physics, and Nuclear Engineering. The material was presented online by using the Moodle course management system from the College of Liberal Arts \& Sciences at the University of Illinois (see the left hand panel of Figure~\ref{moodle}). This course provided greater flexibility to introduce a much broader list of topics, including databases, high performance computing, statistical analysis, and software engineering.

\articlefiguretwo{part1/Brunner_O04/roundii.eps}{part1/Brunner_O04/roundiii.eps}{moodle}{The Moodle home pages for (left) \textit{ASTR 496/596: Practical Informatics for the Physical Sciences}, and (right) \textit{INFO 301/501: Introduction to Data Science}.}

\section{Round Three~\label{r3}}

With the success of these previous efforts, the University of Illinois community saw the need for this material to be offered more broadly. Currently, I am in the process of creating another new course, \textit{INFO 301/501: Introduction to Data Science}, which will provide an introduction to the many tools and techniques that are necessary to work with large data. This course will be delivered entirely online, by using a newer version of the Moodle course management system (see right hand panel of Figure~\ref{moodle}), although discussions have begun to explore hosting this class via Coursera. Specifically, this class will introduce concepts as diverse as working in the BASH shell, using software versioning tools like git, and modeling data by using Python and R.

\section{Conclusions}
While the feedback from the students has indicated a strong appreciation for being exposed to the informatics concepts (especially at the end of the courses!), I would be remiss if I did not acknowledge the personal rewards I have enjoyed by teaching these new concepts. This process has been hard work, but I have gained an even better understanding of the challenges of analyzing large data.

While I have developed all material for the course, I did not do so in a vacuum. In fact one problem faced by newcomers to the informatics venue is the wealth of available material, which is often quite detailed and perhaps inaccessible to someone new to data intensive computing. Specific references I found useful and recommend include the documentation available online for both the Python~\citep{python} and R~\citep{rproject} programming languages. In addition, I leveraged articles I previously wrote about Python~\citep{rjbpython}, the BASH shell~\citep{rjbbash}, and the Apache Derby database~\citep{rjbderby}. Finally, I have found {\em The Visual Display of Quantitative Information}~\citep{tufte09}, {\em Think Stats}~\citep{downey11}, and {\em Data Analysis with Open Source Tools}~\citep{janert10} to provide interesting viewpoints. 

\acknowledgements I would like to thank the Illinois Informatics Initiative for seed funding through CPATH EAE: iCUBED: Informatics and Computation throughout Undergraduate Baccalaureate Education (PI: Leonard Pitt; NSF-CNS-0722327) to support the work discussed in this paper. Also, I would like to thank my department chairs over the last few years: Professors You-Hua Chu and Charles Gammie, for allowing me to create and teach these courses. And most importantly, I would like to thanks the students who enrolled in these courses and were fellow participants in this journey.

\bibliographystyle{asp2010}
\bibliography{editor}

