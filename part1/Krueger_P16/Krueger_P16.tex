
\resetcounters


\markboth{Krueger and Swade}{OpTIIX Overview and E/PO}


\title{OpTIIX Mission Overview and Education/Public Outreach}
\author{Tony~Krueger and Daryl~Swade
\affil{Space Telescope Science Institute,\\
3700 San Martin Drive,  Baltimore,  Md, 21228, USA}
}
\aindex{Krueger, T.}
\aindex{Swade, D.}

\begin{abstract}
The Optical Testbed and Integration on ISS Experiment (OpTIIX) is a technology demonstration to design, develop, deliver, robotically assemble, and successfully operate an observatory on the International Space Station (ISS).  An OpTIIX Education and Public Outreach (EPO) program is being designed to bring OpTIIX and its discoveries to amateur observers, students, educators, and the public.  In addition OpTIIX will be available to the professional community for additional tests using the assembled OpTIIX configuration. 
\end{abstract}

\section{Technology Demonstration Goals}
OpTIIX is being designed to demonstrate the robotic delivery and assembly of an optical telescope in space, light weight nanolaminate active mirrors and laser metrology, and multi-segment telescope wave front detection and correction capability.

\section{OpTIIX Payload}
The OpTIIX payload is a modularized telescope using deformable primary mirror segments that will be launched and assembled on the ISS robotically.  It is comprised of various modules to be launched separately and transported using either a Japanese H-II Transfer Vehicle (HTV) or SpaceX Dragon vehicle.  Figure 1 below shows OpTIIX’s system configuration.  The OpTIIX payload will be robotically assembled and mounted on the ISS at Express Logistics Carrier 3 (ELC3) by the Special Purpose Dexterous Manipulator (SPDM).  ELC3 is a zenith-looking location on ISS.  Figure 2 below shows OpTIIX’s location on ISS. The SPDM will be controlled and commanded from Johnson Space Flight Center. SPDM-compatible robotic interfaces and attachment mechanisms will be located on all telescope modules handled by robots: interfaces for soft and hard mating, electrical power and data connections, grasp fixtures, and optical targets.  The Imaging Camera will be part of the core structures of the telescope (Fig. 1). It will be installed in the core module on the ground and is designed so it can be replaced in space to enable future upgrades. The NRL/JMAPS Program provided a 4k x 4k complementary metal-oxide semiconductor (CMOS) detector array (H4RG HyViSI), already at TRL 6. The camera will include a filter wheel with eight positions, including four wide-band and three narrow-band filters, and one dark.

\begin{figure}[ht!]
\plotone{part1/Krueger_P16/P16_f1.eps}
\caption{OpTIIX System Configuration}
\end{figure}

\begin{figure}[ht!]
\plotone{part1/Krueger_P16/P16_f2.eps}
\caption{OpTIIX at ELC3 on ISS}
\end{figure}

\section{OpTIIX Timeline}
Once all the OpTIIX modules are delivered to ISS, OpTIIX will be robotically assembled and commissioned.  The assembly and commissioning period will take about 3 months.  After commissioning, normal operations will begin and continue nominally for 3 months.  OpTIIX is being built to last up to 2 years.

\section{Education and Public Outreach (EPO)}
An OpTIIX Education and Public Outreach (EPO) program is being designed to bring OpTIIX and its discoveries to amateur observers, students, educators, and the public.

\subsection{Early Release Observations}
A series of observations will be taken during the commissioning period.   These observations will demonstrate the capabilities of OpTIIX observing and verify the success of OpTIIX technologies.

\subsection{Web-Based Education}
Educational materials will include a comprehensive package of resources that will enable educators to incorporate concepts related to real-world astronomical research and optical and robotic technology into their programs. The goal is to encourage collaborative efforts between amateur observers, astronomers, students, and educators that broaden the knowledge and understanding of OpTIXX and its discoveries. These collaborations will also create a community of OpTIXX users that will be supported by an online community workspace. OpTIXX and its discoveries will be available to the public through online outreach activities, social media, and participation in key outreach events

\subsection {OpTIIX Observing}
OpTIIX observing will be available to the public with associated educational materials geared towards Grade 6 to Grade 12 students. Figure 3 below, shows a data flow of the process for requesting observations to receiving the data.  The public will submit observing requests for OpTIIX observing.   These requests will be reviewed by the Time Allocation Committee with emphasis on their educational value to students. STScI will augment these requests if necessary to ensure a good mix of educational targets.  An approved target list will be created.  Figure 4 below shows a representative set of targets for the first 3 months of OpTIIX observing.  Approved targets will be scheduled on a 3 month timeline.  OpTIIX is collaborating with the MicroObservatory Robotic Telescope Network operated by Harvard-Smithsonian Center for Astrophysics.  Students and the public will go to the MicroObservatory portal to select OpTIIX targets they want to observe.  Since OpTIIX data is not proprietary, if an observation is requested by more than one individual, the data will be provided to all the requestors within 1 week of the request.  The data will also be available via the Mikulski Archive for Space Telescopes (MAST) portal at STScI.

\begin{figure}[ht!]
\plotone{part1/Krueger_P16/P16_f3.eps}
\caption{Observing with OpTIIX Data Flow}
\end{figure}

\begin{figure}[ht!]
\plotone{part1/Krueger_P16/P16_f4.eps}
\caption{Representative Educational Targets by RA/DEC}
\end{figure}


