\documentclass[11pt,twoside]{article}
\usepackage{asp2010}

\resetcounters

\bibliographystyle{asp2010}

\markboth{Yamauchi}{2MASS Catalog Server Kit version 2.1}

\begin{document}

\title{2MASS Catalog Server Kit version 2.1}
\author{Chisato Yamauchi$^1$
\affil{$^1$Astronomy Data Center,
    National Astronomical Observatory of Japan,
    2-21-1 Osawa, Mitaka, Tokyo, 181-8588, Japan}
}

\begin{abstract}
2MASS Catalog Server Kit is an open source software for use 
in easily constructing a high performance search server for 
important astronomical catalogs. 
This software utilizes the open source RDBMS PostgreSQL, therefore, 
any users can setup the database on their local computers by following
step-by-step installation guide. 
The kit provides highly optimized stored functions for positional search
similar to SDSS SkyServer. 
Together with them, powerful SQL environment of PostgreSQL will meet
various user's demands.

We released 2MASS Catalog Server Kit version 2.1 on May 2012 
that supports latest WISE All-Sky catalog (563,921,584 rows) and 9 
major all-sky catalogs.
Local databases are often indispensable for observatories with 
unstable or narrow-band network or severe use such as
retrieving large amount of records
within a small period of time.
This software is the best for such purposes, and
increasing supported catalogs and improvements of version 2.1 
can cover a wider range of applications including
advanced calibration system, scientific studies
using complicated SQL queries, etc.

Official page: {\tt http://www.ir.isas.jaxa.jp/\~{}cyamauch/2masskit/}
\end{abstract}


\section{Purposes}

One of the most important points of 2MASS Catalog Server Kit 
\citep{yam_2011a} is
that users directly input SQL statements to the database server.
SQL has powerful flexibility of the programming interfaces for searching
and manipulating tables, 
therefore, 2MASS Kit can adapt wide variety of purposes.

Typical use cases will be 
1) Astrometric and flux calibration at observatories with 
unstable or narrow-band network, and
2) Severe use such as retrieving large amount of entries or 
complicated SQL queries for calibration, scientific studies, etc.
Actually CFHT observatory and some Japanese institutes/observatories 
employ our 2MASS Kit. 


\section{Supported Catalogs}

2MASS Kit version 2.1 supports 10 major all-sky catalogs
in optical and infrared bands:
\begin{center}
\begin{tabular}{ll}
\hline
 2MASS PSC (470,992,970 rows)    &        GSC-2.3.2 (945,592,683 rows) \\
 WISE All-Sky (563,921,584 rows) &       UCAC3 (100,766,420 rows) \\
 USNO-B1.0 (1,045,175,762 rows) &       PPMXL (910,468,710 rows) \\
 AKARI IRC PSC (870,973 rows)    &       Tycho-2 (2,539,913 rows) \\
 AKARI FIS BSC (427,071 rows)    &       IRAS PSC (245,889 rows) \\
\hline
\end{tabular}
\end{center}
WISE 3-Band Cryo Release (261,418,479 rows) and UCAC4
will be supported in the near future.


\section{Optimized Search Functions}

Users can instantly use 2MASS Kit's high-speed positional search
implemented by reasonable database design and high-level tuning of 
indices and stored functions.
2MASS Kit has three types of stored functions for
positional search:
1) Radial Search of J2000, Ecliptic, Galactic and B1950,
2) Box Search of J2000, Ecliptic, Galactic and B1950, and
3) Rectangular Search of J2000.


\section{System Requirements}

Installing 2MASS Kit requires
64-bit or 32-bit OS on which PostgreSQL-8.4 (or later) server works.
A 3 Tbyte of hard drive has to be prepared to register all supported catalogs.
Of course, you can install only necessary catalogs.
Required disk space for each catalog is shown in 2MASS Kit Web page.

External USB-2.0 hard drive can be used for typical search,
but we recommend a system with RAID10 and/or SSD for severe use.
In 2MASS Kit database, each table/index is registered in one of six
table spaces (each resides in a separate directory)
for flexible tuning. 
This allows only the essential parts to be easily moved onto fast devices.


\section{How useful?}

We show actual SQL statements to use 2MASS Kit database
and their results.
We recommend you to try following examples with changing 
arguments of stored functions to know real speed of our
database.

\begin{verbatim}
-- An SQL statement to perform a radial search of 2MASS with J2000:
SELECT * FROM fTwomassGetNearbyObjCel('j2000', 266,-29, 0.2);
\end{verbatim}

{\small
\begin{verbatim}
   objid   |    lon     |    lat     |     distance      
-----------+------------+------------+-------------------
 684105609 | 265.999015 | -29.002653 | 0.167362064774342
 684575681 | 266.002656 | -28.999596 | 0.141471788449169
\end{verbatim}
}

\begin{verbatim}
-- Join the return value from the function to the Main Table:
SELECT o.ra, o.dec, o.j_m, o.h_m, o.k_m, n.distance
FROM fTwomassGetNearbyObjEq(266, -29, 0.2) n, Twomass o
WHERE n.objid = o.objid;
\end{verbatim}

{\small
\begin{verbatim}
     ra     |    dec     |  j_m   |  h_m   |  k_m   |     distance     
------------+------------+--------+--------+--------+-------------------
 265.999015 | -29.002653 | 12.502 | 11.488 | 10.551 | 0.167362064774342
 266.002656 | -28.999596 | 14.579 | 11.091 |  9.283 | 0.141471788449169
\end{verbatim}
}

Shown in next example,
2MASS Kit provides stored functions for coordinate conversion.
Some C functions of WCSTools \citep{min_2006}
are used to implement them.

\begin{verbatim}
-- To display positions in sexagesimal and in Galactic coordinate:
SELECT fDeg2LonStr(o.ra) as ra, fDeg2LatStr(o.dec) as dec, 
       fJ2L(o.ra,o.dec) as l, fJ2B(o.ra,o.dec) as b
FROM fTwomassGetNearbyObjEq(266, -29, 0.2) n, twomass o
WHERE n.objid = o.objid;
\end{verbatim}

{\small
\begin{verbatim}
     ra      |     dec     |        l         |         b         
-------------+-------------+------------------+-------------------
 17:43:59.76 | -29:00:09.6 | 359.757732609713 | 0.268108804975495
 17:44:00.64 | -28:59:58.5 | 359.762004949764 | 0.266998167472491
\end{verbatim}
}

\begin{verbatim}
-- Search elongated region along Galactic plane:
SELECT count(*) 
FROM fTwomassGetNearbyObjFromBoxCel('galactic',0,0,600,1);
\end{verbatim}

{\small
\begin{verbatim}
 count 
-------
 85410
\end{verbatim}
}

{\small
\begin{verbatim}
-- Perform a radial search of IRAS and a cross-id between 2MASS and
-- returned result of IRAS.  Using SSD is recommended for performance:
SELECT r.ra as ra_iras, r.dec as dec_iras,
       p.ra as ra_2mass, p.dec as dec_2mass,
       fDistanceArcMinEq(r.ra,r.dec,p.ra,p.dec) as distance
FROM (
 SELECT o.*
 FROM fIrasGetNearbyObjEq(180,2,60) n, Iras o
 WHERE o.objid = n.objid
) r
LEFT JOIN Twomass p
ON fTwomassGetNearestObjIDEq(r.ra,r.dec, 1.0) = p.objID;
\end{verbatim}
}

{\small
\begin{verbatim}
  ra_iras   | dec_iras |  ra_2mass  | dec_2mass |      distance      
------------+----------+------------+-----------+--------------------
 179.849487 | 1.827165 | 179.849741 |  1.826719 | 0.0307915522123298
 180.673096 | 1.977994 | 180.675986 |  1.976742 |  0.188877689977676
 179.247971 | 2.409427 | 179.242712 |  2.412174 |  0.355745637503231
\end{verbatim}
}


\section{Database Design}

\subsection{Spatial Indexing}

Neither HTM \citep{kun_2000} nor HEALPix \citep{gor_2005} are used.
To enable high-speed radial and box search,
we use a composite index on unit vector of positions 
({\tt cx}, {\tt cy}, {\tt cz}) and select entries with following procedure:
1) Catch objects within a cube (or multiple cubes) 
using index scan on ({\tt cx}, {\tt cy}, {\tt cz}), and
2) Select objects within the strict search circle (or box) on the celestial
sphere from the result of step 1.
Figure 2 in \citet{yam_2011a} shows the concept of our radial search
that uses $x$$y$$z$ coordinate for the database index.

The feature of our algorithm is that it requires almost no calculation
before executing the index scan, and the efficiency is quite high for a
small search radius.
In addition, we do not have to implement special processing for polar
singularity.

\subsection{Special Design for Huge Catalogs}

\begin{figure}[t]
\plotone{db_design.eps}
\end{figure}

If we apply a straightforward implementation
for huge catalogs, two problems will be arisen: 
1) Composite indices of $x$$y$$z$
of {\tt FLOAT8} type enlarges disk access,
2) Larger height of indices on huge tables also slow down processing speed.

To solve these problems,
we use table partitioning and {\it expression index}.
Shown in Figure, 
we create a special table set for positional search
that has only J2000 positions of 4-byte integer on which indices are created.
All catalog entries are distributed into child tables (partitions)
divided by their declination, and
stored functions written in C and PL/pgSQL quickly select only required
partitions.
Using {\it expression index} technique can reduce size of indices, i.e.,
we can create a composite index on $x$$y$$z$ of
{\bf 2-byte integers} on table columns  
({\tt ra}, {\tt dec}) or ({\tt cx}, {\tt cy}, {\tt cz}) of any types.  
This minimizes disk access of index scan.
Details are described in Section 4.3.4 in \citet{yam_2011a}.


\section{Performance}

Although some test results are shown in \citet{yam_2011a},
we recommend you to test our software. 
To obtain our finished product as soon as possible,
we offer you our Hard Drive Copy Service. 
Please visit our Web page and contact us to use this service. 

\acknowledgements We thank 
Dr. Satoshi Takita, Dr. Shinki Ooyabu, Dr. Norio Ikeda, and Dr. Yoshifusa Ita
for their hacks and valuable suggestions.

\bibliography{../../editor}


\end{document}
