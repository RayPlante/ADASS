
\resetcounters

\bibliographystyle{asp2010}

\markboth{Shortridge et al.}{The Software System for the AAO's HERMES Spectrograph}

\hyphenation{AAO-mega}

\title{The Software System for the AAO's \ssindex{instruments!individual!HERMES}HERMES Spectrograph}
\author{Keith~Shortridge, Tony~Farrell, Minh~Vuong, Michael~Birchall, and Ron~Heald
\affil{Australian Astronomical Observatory, PO Box 915, North Ryde, NSW 1670, Australia}}

\aindex{Shortridge, K.}
\aindex{Farrell, T.}
\aindex{Vuong, M.}
\aindex{Birchall, M.}
\aindex{Heald, R.}

\begin{abstract}
The AAO's \ssindex{instruments!individual!HERMES}HERMES \ssindex{spectrograph}spectrograph will start operation in 2013. Its primary project will be a Galactic Archaeology survey that aims to reconstruct the early history of our Galaxy through precise measurements of the chemical abundances of one million stars. This paper describes some of the software aspects of the \ssindex{instruments!individual!HERMES}HERMES project: how it has evolved from the earlier AAO \ssindex{instruments!individual!2dF}2dF system, the extensive use of \ssindex{astronomy!simulation}simulation for testing, the overall observing system, and the data reduction \ssindex{data!pipelines!reduction}pipeline.
\end{abstract}

\section{Introduction}

The AAO's new \ssindex{instruments!individual!HERMES}HERMES instrument \citep{Hermes_2010} is due to begin operations in mid-2013. The existing \ssindex{instruments!individual!2dF}2dF robotic fibre positioner will feed the output from ~400 optical fibres at a time into a powerful new \ssindex{spectrograph}spectrograph covering four optical bands simultaneously. The primary science project for \ssindex{instruments!individual!HERMES}HERMES is the 'Galactic Archaeology' (GA) Survey, which aims to reconstruct the history of our Galaxy's formation from precise multi-element abundance measurements of one million stars.

The software for \ssindex{instruments!individual!HERMES}HERMES is an extension of that used for the existing \ssindex{instruments!individual!2dF}2dF and \ssindex{instruments!individual!AAOmega}AAOmega instruments. A simplistic way of looking at \ssindex{instruments!individual!HERMES}HERMES is to see it as \ssindex{instruments!individual!AAOmega}AAOmega with a new \ssindex{spectrograph}spectrograph and with four, larger detectors instead of two. The modular design of the original 2dF software has greatly simplified its evolution, first for use with \ssindex{instruments!individual!AAOmega}AAOmega, now for use with \ssindex{instruments!individual!HERMES}HERMES.

The main software changes involved in the transition from \ssindex{instruments!individual!AAOmega}AAOmega to \ssindex{instruments!individual!HERMES}HERMES are:

\begin{itemize}
\itemsep0em
\item A completely new \ssindex{spectrograph}spectrograph control task.
\item Changes to the 2dFdr data reduction package and its algorithms to match the \ssindex{instruments!individual!HERMES}HERMES data.
\item Extension of the overall control task to handle four detectors rather than two.
\item Changes to the detector control software to handle the new, larger, 4-amplifier, \ssindex{instruments!individual!HERMES}HERMES detectors.
\end{itemize}

To minimise (and preferably to eliminate) the dead period between the hardware becoming available and all the software working properly, extensive use has been made of \ssindex{astronomy!simulation}simulation:

\begin{itemize}
\itemsep0em
\item A detailed simulator of the \ssindex{spectrograph}spectrograph and its \ssindex{computers!hardware!CANBus}CANBus-based electronics has been written for use in testing the spectograph control software.
\item An equally detailed data simulator, based on the optical design of the \ssindex{spectrograph}spectrograph, has been used to generate accurate test images that can be used to  test the data reduction software.
\end{itemize}

The main frame of the \ssindex{spectrograph}spectrograph is now complete, and the optics for one arm (the blue) are being tested. The other three arms will follow shortly.

\section{Observing System}

The existing 2dF/AAOmega observing software has been evolving continuously since it was first designed in 1994. It consists of a collection of tasks all written using the AAO's DRAMA data acquisition environment \citep{Drama_1995}. The modular design of the system has simplified the evolution of the software to match that of the hardware from \ssindex{instruments!individual!2dF}2dF to \ssindex{instruments!individual!AAOmega}AAOmega (mainly a replacement of the \ssindex{spectrograph}spectrographs) and now to \ssindex{instruments!individual!HERMES}HERMES (again a new \ssindex{spectrograph}spectrograph, but also with new optical fibres that can feed either \ssindex{instruments!individual!HERMES}HERMES or \ssindex{instruments!individual!AAOmega}AAOmega, and four detectors rather than two).

The overall structure of the \ssindex{instruments!individual!HERMES}HERMES observing system is shown in Figure 1. Real-time DRAMA tasks, such as those that control the robot fibre positioner or read out the CCD detectors \citep{AAO2_2004},  run on VxWorks CPUs in VME chassis. The remainder run on Linux or Solaris systems (although Solaris is slowly being phased out). Most development is now done on OS X systems, using simulator tasks to emulate the hardware.

\begin{figure}
\plotone{part10/Shortridge_P027/P027_f1.eps}
\caption{The \ssindex{instruments!individual!2dF}2dF observing system, running in \ssindex{instruments!individual!HERMES}HERMES mode.}
\end{figure}

Overall control is provided by the main 2dF control task, which coordinates the operation of the rest of the system. Not shown in the diagram are the fibre configuration program, which handles the assignment of the fibres to the target objects, and the data reduction system, 2dFdr, which normally runs autonomously alongside the observing system. \ssindex{instruments!individual!HERMES}HERMES will have slightly more coordination between the observing system and 2dFdr in order to provide an automated focus and alignment facility. The Galactic Archaeology project supplies its own analysis \ssindex{data!pipelines!reduction}pipeline; 2dFdr removes the instrumental effects from the data and provides calibrated spectra to the \ssindex{data!pipelines!reduction}pipeline, which then handles the scientific analysis.

The system can be run in either \ssindex{instruments!individual!AAOmega}AAOmega or \ssindex{instruments!individual!HERMES}HERMES mode.

\section{Spectrograph Control Task and Simulator}

In both modes, \ssindex{instruments!individual!AAOmega}AAOmega and \ssindex{instruments!individual!HERMES}HERMES, there is a \ssindex{spectrograph}spectrograph control task. This task is responsible for all communication with the hardware and implements high-level DRAMA commands for moving the \ssindex{spectrograph}spectrograph into various observing configurations. Because the two \ssindex{spectrograph}spectrographs are quite different, there are two separate \ssindex{spectrograph}spectrograph control tasks, although both interact with the overall control task in the same way.

For all its optical complexity, the \ssindex{instruments!individual!HERMES}HERMES \ssindex{spectrograph}spectrograph operates in fixed wavelength bands and does not pose a particularly difficult control problem. The main software complication came from the decision to move to a field bus system (\ssindex{computers!hardware!CANBus}CANBus) for the \ssindex{spectrograph}spectrograph electronics. \ssindex{computers!hardware!CANBus}CANBus was new to the AAO, and we needed to be sure that the highly multi-threaded 3rd party \ssindex{libraries!CANOpen}CANOpen library (CML) we planned to use would work properly with DRAMA

We also needed to be able to test the new \ssindex{spectrograph}spectrograph control task long before the hardware would become available, since we could not afford the traditional long delays involved in finally getting hardware and software working together. This, combined with the need to be confident about the \ssindex{libraries!CANOpen}CANOpen library in a DRAMA environment, led us to use a highly elaborate low-level \ssindex{astronomy!simulation}simulation not just of the \ssindex{spectrograph}spectrograph but also of the underlying \ssindex{computers!hardware!CANBus}CANBus hardware.

Rather than provide a simulation version of the \ssindex{libraries!CANOpen}CANOpen library, we chose to use the real library in all its multi-threaded glory, intercepting the \ssindex{computers!hardware!CANBus}CANBus packets it creates at the point they would normally be written to the \ssindex{computers!hardware!CANBus}CANBus interface hardware. These are routed instead to a UNIX pipe, connected to an AAO-written task that emulates \ssindex{computers!hardware!CANBus}CANBus operations at this very low level, and which mimics the operation of a set of \ssindex{computers!hardware!CANBus}CANBus nodes (digital I/O and amplifier modules) which work with a detailed simulation of the \ssindex{instruments!individual!HERMES}HERMES \ssindex{spectrograph}spectrograph to produce a test bed that not only allows the control task to be tested, but also provides confidence that the \ssindex{libraries!CANOpen}CANOpen library works without problems in a DRAMA task. This essentially involved reverse engineering the \ssindex{libraries!CANOpen}CANOpen protocol, and represented a significant effort, but was something that could be done early in the project and so was not on the critical path.

The simulator has its own \ssindex{software!user interfaces}GUI, and this can be used to provide control over how the simulated \ssindex{spectrograph}spectrograph operates. For example, a slider on the \ssindex{software!user interfaces}GUI controls the simulated air pressure for the pneumatic mechanisms, and can be used to reduce this pressure to the point that they no longer work. The control task has to spot this, handle the resulting failures, and report the error condition to the overall \ssindex{instruments!individual!HERMES}HERMES control task.

This approach, based on previous use of detailed simulation at AAO \citep{TCS_2010}, has been very successful. So far, integration of the hardware as it becomes available with the already functionally complete control task has been remarkably unexciting. We are able to run in a partial simulation mode, using the actual hardware that is available and simulating the hardware still to be completed. We have almost no test programs as such; all testing is done with the complete, operational, control task running with varying amounts of simulation.

\section{Data Reduction and the Data Simulator}

The AAO's existing fibre instrument reduction package, 2dFdr, is being adapted to support \ssindex{instruments!individual!HERMES}HERMES. 2dFdr is quite an old package now (it is written in \ssindex{computer languages!Fortran}Fortran and originally was based around the \ssindex{packages!Starlink}Starlink\ooindex{Starlink, ascl:1110.012} package) but has been refurbished and refactored in recent years and the necessary changes are relatively easy to implement. The main changes relate partly to changes in the fibre data itself (there is more overlap between the data from adjoining fibres, meaning optimal extraction techniques become very important, and the resolution is higher) and partly to the stringent new signal to noise requirements of the Galactic Archaeology survey, which mandates much more precise handling of many aspects of the data reduction, particularly the arc identification and wavelength calibration.

The survey timetable means we cannot afford the usual delay as the instrumental characteristics become apparent from the initial batches of data. We need to test the data reduction software in advance of the construction of the actual instrument. To allow this, a very detailed data simulator has been written,

The data simulator combines the optical design of the \ssindex{spectrograph}spectrograph with spectra from stars with well-known abundances to produce images that should provide a realistic representation of the actual spectra from the instrument, and these are used to test the algorithms in 2dFdr. Now that we have some real data from one of the \ssindex{spectrograph}spectrograph arms, albeit so far with only one input fibre, we can check the accuracy of the simulation and correct any discrepancies. This process will continue as more of the actual instrument becomes available for testing. Most of the changes to 2dFdr can also be used to improve the \ssindex{instruments!individual!AAOmega}AAOmega data reduction and are being tested with real \ssindex{instruments!individual!AAOmega}AAOmega data as well.

Changes to the 2dFdr algorithms for \ssindex{instruments!individual!HERMES}HERMES have included: using polynomials derived from the data simulator as the basis for locating the spectra in the image; an improved optimal extraction algorithm to extract the individual fibre spectra \citep{OptExt_2010}; sky subtraction based on \ssindex{methods!Principal Components Analysis (PCA)}Principal Components Analysis (PCA) \citep{PCA_2010}; and improved wavelength calibration using an algorithm based on Whale Shark identification \citep{Sharks_2005}.

\bibliography{editor}
