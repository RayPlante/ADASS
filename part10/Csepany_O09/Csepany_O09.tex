
\resetcounters

\bibliographystyle{asp2010}

\markboth{Cs\'ep\'any et al.}{The Fly's Eye Camera System}

\title{The Fly's Eye Camera System -- An Instrument Design for Large \'Etendue Time-domain Surveys}
\author{%
Gergely~Cs\'ep\'any,$^{1}$
Andr\'as~P\'al,$^{1,2}$
Kriszti\'an~Vida,$^{1}$
Zsolt~Reg\'aly,$^{1}$
L\'aszl\'o~M\'esz\'aros,$^{1,2}$
Katalin~Ol\'ah,$^{1}$
Csaba~Kiss,$^{1}$
L\'aszl\'o~D\"obrentei,$^{1}$
Attila~Jask\'o,$^{1}$
Gy\"orgy~Mez\H{o},$^{1}$ and
Ern\H{o}~Farkas$^{1}$
\affil{
$^1$MTA Research Centre for Astronomy and Earth Sciences, 
Konkoly Thege Mikl\'os \'ut 15-17, Budapest, H-1121, Hungary;\\
$^2$Department of Astronomy, Lor\'and E\"otv\"os University, 
P\'azm\'any P\'eter s\'et\'any 1/A, Budapest H-1117, Hungary
}}

\aindex{Cs\'ep\'any, G.}
\aindex{P\'al, A.}
\aindex{Vida, K.}
\aindex{Reg\'aly, Z.}
\aindex{M\'esz\'aros, L.}
\aindex{Ol\'ah, K.}
\aindex{Kiss, C.}
\aindex{D\"obrentei, L.}
\aindex{Jask\'o, A.}
\aindex{Mez\H{o}, G.}
\aindex{Farkas, E.}

\begin{abstract}
In this paper we briefly summarize the design concepts of the {\it Fly's Eye Camera System}, a proposed high resolution all-sky monitoring device which intends to perform high cadence \ssindex{astronomy!time domain}time domain astronomy in multiple optical passbands while still accomplish a high \'etendue. Funding has already been accepted by the Hungarian Academy of Sciences in order to design and build a {\it Fly's Eye} device unit. Beyond the technical details and the actual scientific goals, this paper also discusses the possibilities and yields of a network operation involving $\sim10$ sites distributed geographically in  a nearly homogeneous manner. Currently, we expect  to finalize the mount assembly -- that performs the sidereal tracking during the \ssindex{observing!exposure}exposures -- until the end of 2012 and to have a working prototype with a reduced number of individual cameras sometime in the spring or summer of 2013.
\end{abstract}

\section{Introduction}

The key to learning more about the Universe and unveiling the physical processes beyond various astronomical phenomena is to monitor the alterations of observable quantities, such as fluxes, in several spectral regimes of the sky. Although some of the astrophysical processes have their own characteristic timescales, most of the complex systems exhibit variations on a broader, currently unexplored temporal spectrum. Hence, continuous monitoring of the whole sky will reveal currently unknown phenomena and quantify properties of physical events ongoing in stellar and planetary systems as well as in their neighborhood. We aim to design, build and operate the core of the {\it Fly's Eye Network}: a network of geographically distributed large \'etendue and high resolution all-sky monitoring devices, extending the currently developed single-station operation. This network provides a homogeneous and truly full sky investigation of the \ssindex{astronomy!time domain}time domain of astrophysical events that covers $\sim6$ magnitudes from the data acquisition cadence of some minutes up to the range of the expected operations, i.e. several years. The novel hexapod-based arrangement of the camera platform of the {\it Fly's Eye} device allows us to install and maintain a setup independently of the geographical location and without the need of polar adjustment in an enclosed dome from which the detectors watch the sky through optical windows. Moreover, the construction is highly fault tolerant and lacks unique components. All of these enable an easily sustainable instrumentation even in harsh environments. Due to its design parameters, the resulting network will yield an \'etendue that is comparable to that of the Large Synoptic Survey Telescope \citep[\ssindex{observatories!Earth-based!LSST}LSST,][]{ivezic2008}. Moreover, our concept provides an innovative instrument that is fully complementary to the latter, since the {\it Fly's Eye} faint limit is roughly the same as the expected saturation brightness of the \ssindex{observatories!Earth-based!LSST}LSST and the employed spectral passbands also match.

\begin{figure}
\begin{center}
\resizebox{60mm}{!}{\includegraphics{part10/Csepany_O09/O09_f1}}%
\resizebox{60mm}{!}{\includegraphics{part10/Csepany_O09/O09_f2}}
\end{center}
\caption{%
{\it Left panel:} a simple on-scale visualization of the camera mount and optics. The payload platform with the 19 camera -- lens pairs has an effective diameter of nearly 1\,m. {\it Right panel:} the optical light-collecting phase volume, or \'etendue and effective resolution for various known, mostly optical telescope systems. The proposed design of a single {\it Fly's Eye Camera} unit yields a value that is comparable to the available instruments which have the largest \'etendue: a single unit of the \emph{Pan-STARRS} telescope(s) and the \emph{Kepler} space telescope. A network of nine {\it Fly's Eye} devices(see labeled as {\it FE Net}) yields an \'etendue comparable to the proposed design of \emph{LSST}.}
\label{fig:flyseye}
\end{figure}

\section{Design Concepts}

Although ``normal'' all-sky cameras are assembled with fixed optics and detectors, the expected per-image \ssindex{observing!exposure}exposure times and the required resolution\footnote{The resolution is nonetheless smaller than the average imaging optical telescopes but definitely higher than of the usual all-sky cameras.} requires sidereal tracking of the {\it Fly's Eye} optical mount. In order to provide a solution for sidereal tracking, we employ a hexapod-based design\footnote{\url{http://en.wikipedia.org/wiki/Stewart_platform}} for supporting the cameras and optics. These hexapod mounts allow us to perform spatial rotations in all of the three rotational degrees of freedom without any parametric singularity (e.g. gimbal lock). Hence, this design yields not only a device that is exactly the same independently from the actual geographical location but also makes the otherwise complicated procedure of polar adjustment unnecessary. Moreover, hexapods are fault-tolerant mechanisms in this type of application since the degrees of freedom associated with the spatial displacements are not exploited. In other words, even 3 out of the 6 legs can be broken but the device is still capable of tracking and compensating for the apparent rotation of Earth. 

The camera platform itself consists of 19 wide-field cameras, where each camera lens-system is built from $4{\rm k}\times 4{\rm k}$ detectors with a pixel size of $9\times9\,{\rm\mu m}$ while the optics are 85mm/f1.2 lenses. This setup yields an effective resolution of $22''/{\rm pixel}$ and an \'etendue of $40\,{\rm m}^2{\rm deg}^2$ and covers the half of the visible sky. The visualization of the camera assembly, the hexapod mount and the resulted field-of-view are displayed in the left panel of Figure~\ref{fig:flyseye}. Due to the arrangement and the large number of individual camera-lens units, the whole design resembles the compound eyes of insects, hence the name of the project.

The expected data flow rate in continuous operation is about $100$~TB/year. Taking into account a yearly duty cycle of $0.40$ (for Hungary), the data flow reduces to 40--50~TB/year without compression. By employing data compression (eg. FPACK) we except to have about 30~TB data to store each year. Since the most affordable hard disk size nowadays is the 2~TB disk, having 19 disks (one for each camera, totaling in $19\times2=38$~TB) can roughly hold up to one year's data. To secure the data and provide redundancy, the data can be distributed on 24 disks, by adding 5 extra disks to the 19. The 24 disks can be grouped as $6\times4$~disks or $4\times6$~disks, therefore the data can be recovered even if a full unit breaks (in the former setup) or if one disk dies in each unit (in the latter setup). The full redundancy is achieved by using Galois field arithmetic over $GF(2^8)$ or $GF(2^{32})$ and Vandermonde transformation.

\section{Scientific Applications}

The main goals of the proposed {\it Fly's Eye} Camera System and the further Network cover several topics in astrophysics. In the following, without attempting to be comprehensive, we list some of these sub-fields of astrophysics. 

\subsection{Solar System} 
Even with its moderate resolution, the {\it Fly's Eye} device is capable of detecting meteors and mapping their tracks with an effective resolution of $\sim10$\,m/pixel. Hence, and due to the large light collecting power of the device, a more accurate distribution of \ssindex{astronomy!solar system}Solar System dust can be derived. In addition, for the bright-end of the main-belt asteroid family members, an unbiased sample will be available for their rotation and shape properties \citep[see also][]{durech2011}. These data are essential to understanding the aspects of \ssindex{astronomy!solar system}Solar System dynamics and, more importantly, its evolution. 

Moreover, nearby flybys of small bodies that are potentially hazardous to the Earth can be traced (see e.g. the cases of 2005\,YU\ensuremath{_{55}} and 2012\,BX\ensuremath{_{34}}{}). Due to the continuous sampling, such information is also recoverable in an \emph{a posteriori} manner, i.e. if deeper surveys discover such an object and dynamical calculations confirm a former approach in the field of view of one or more {\it Fly's Eye} device.

\subsection{Stellar and Planetary Systems} 
Young stellar objects are complex astrophysical systems and show signs of both quasi-periodic and sudden  \ssindex{astronomy!transients}transient, eruptive processes. By monitoring their intrinsic variability, one is able to obtain several constraints regarding to the ongoing processes \citep{hartmann1996,abraham2009}. Persistent monitoring of numerous young stellar objects or candidates for young stellar objects will reveal the nature of the  currently unexplored domains of stellar birth. Since observing campaigns are organized mostly on daily or yearly bases, the behavior of such systems is practically unknown on other timescales. 

Stars with magnetic activity show photometric variability on all the time-domains of the planned instrument, from minutes through hours to years, just like the Sun does \citep{strassmeier2009}. Continuous monitoring of the sky opens up a new research area for active stars: the proposed device allows us to obtain good flare statistics since flares occur on minute-hour timescales, and to monitor starspot evolution, differential rotation  and activity cycles of the same star \citep[see e.g.][]{hartman2011,walkowicz2011,olah2009}.

Observations of eclipsing binaries provide direct measurements of stellar masses and radii that are essential to understand their evolution and even the basic physical processes ongoing in the stellar cores \citep{latham2009}. Similar to eclipsing binaries, transiting extrasolar planets are also expected to be discovered by the {\it Fly's Eye Network}, since instruments with nearly similar types of optics are found to be rather efficient \citep{pollacco2004,bakos2004,pepper2007}.

\subsection{In the \ssindex{astronomy!extragalactic}extragalactic environment}
Continuous monitoring of brighter super\ssindex{astronomy!novae}novae in nearby galaxies yield valuable data that can be exploited by combining other kinds of measurements. The {\it Fly's Eye} camera is capable of observing the brightest super\ssindex{astronomy!novae}novae directly for even up to a month during their peak brightness \citep[see e.g.][]{vinko2012} and by combining images, it is possible to go even deeper in brightness using more sophisticated photometric techniques. 

\acknowledgements 
The {\it Fly's Eye} project is supported by the Hungarian Academy of Sciences via the ``Lend\"ulet'' grant LP2012-31/2012. Additional support is also received via the ESA grant PECS~98073.

\bibliography{editor}
