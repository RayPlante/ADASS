% This is the aspauthor.tex LaTeX file
% Copyright 2010, Astronomical Society of the Pacific Conference Series

\documentclass[11pt,twoside]{article}
\usepackage{asp2010}



\def \degmark{^\circ}
\def \ergsec{\hbox{erg s$^{-1}$}}
\def \ergcmsec{\hbox{erg cm$^{-2}$ s$^{-1}$}}
\def \phcmsec{\hbox{photons cm$^{-2}$ s$^{-1}$}}
\def \ferg {erg cm$^{-2}$ s$^{-1}$}
\def \arcmin {\hbox{$^\prime$}}
\def \arcsec {\hbox{$^{\prime\prime}$}}
\def \chisq {$\chi ^{2}$}
\def \rchisq {$\chi_{\nu} ^{2}$}
\def \gray {$\gamma$-ray }
\def \enmev {$E > 100 \: MeV$}

\resetcounters



\markboth{Bulgarelli, A.}{AGILE-GRID science alert monitoring system }

\begin{document}

\title{AGILE/GRID science alert monitoring system: the workflow and the Crab flare case}
\author{A.\ Bulgarelli\altaffilmark{1}, M.\ Trifoglio\altaffilmark{1}, F.\ Gianotti\altaffilmark{1}, 
M. \ Tavani\altaffilmark{2}, V. Conforti\altaffilmark{1}, N. Parmiggiani\altaffilmark{3}}

\affil{$^1$National Institute for Astrophysics - IASF Bologna, Italy}
\affil{$^2$National Institute for Astrophysics - IAPS Roma, Italy}
\affil{$^3$University of Modena and Reggio Emilia}

\begin{abstract}
During the first five years of the AGILE mission we have observed many gamma-ray transients of Galactic and extra-Galactic origin. A fast reaction to unexpected transient events is a crucial part of the AGILE monitoring program, because the follow-up of astrophysical transients is a key point for this space mission. We present the workflow and the software developed by the AGILE Team to perform the automatic analysis  for the detection of gamma-ray transients. In addition, an App for iPhone will be released enabling the  Team to access the monitoring system through mobile phone. In September 2010 the science alert monitoring system presented in this paper has recorded a transient phenomena from the Crab Nebula, generating an automated alert sent via email and SMS two hours after the end of an AGILE satellite orbit, i.e. two hours after the Crab flare itself: for this discovery AGILE won the 2012 Bruno Rossi prize. The design of this alert system is maximized to reach the maximum speed, and in this as in many other cases AGILE has demonstrated that the reaction speed of the monitoring system is crucial for the scientific return of the mission. 
\end{abstract}

\section{Introduction}
The search for $\gamma$-ray transients (Galactic and extragalactic)
      detectable on timescales of 1-2 days is one of the major
      activities performed by the AGILE Collaboration.
           Quick reaction times allow us focusing on transient events
          detected by AGILE in the MeV-GeV energy range. Serendipity
          transient discoveries during the AGILE observation is
          possible given the large field of view and the AGILE
          sensitivity.
          The AGILE quick-look Science Alert Monitoring system  is a software
          and hardware automated analysis system developed and used by
          AGILE team for alert generation of flares from astrophysical
          sources.
        
          
          AGILE (Astrorivelatore Gamma ad Immagini LEggero - Light Imager
      for Gamma-ray Astrophysics) is a scientific mission of the
      Italian Space Agency (ASI) launched on April 23, 2007 \citep{Tavani:2009ht}.
      The Gamma-Ray Imaging Detector
      (GRID) for observations in the 30 MeV-50 GeV $\gamma$ energy range works with a very
      large fields of view (FOV) of more than 120 degrees across, i.e.,
      2.5 sr.
      
      
 The AGILE orbital characteristics (quasi-equatorial with an
      inclination angle of 2.5 degrees and average 530 km altitude) are
      optimal for low-background  $\gamma$-ray observations. Each orbit last
      about 96'.
        
      From 2009 Nov the AGILE $\gamma$-ray observations of the sky are operated  with the
          satellite operating in �spinning mode� (the satellite axis sweeps an entire circle in the
          sky in approximately 7 min) with a typical
          accumulating pattern (see Fig. \ref{fig_fov}).            
                    
The orbital characteristics, the large FOV and the operating modes has a great impact on the
      design of the science monitoring system and one of the key
      software requirements of the system is that the data processing
      should be completed within the next orbit to avoid or reduce backlog. This operating mode has required two different design of the science alert monitoring system. In addition an evaluation of the maximum likelihood method (the analysis method used by this system) for detecting short-term variability of AGILE $\gamma$-ray sources is a key point of this monitoring system \citep{Bulgarelli:2012ds}.
                   
\begin{figure}[t]
\epsscale{0.93}
\plotone{O.05_1c.eps}
\caption{An example of a two-days counts map of AGILE-GRID showing the entire sky in spinning mode in 2012 Oct. The science monitoring system analyze the entire sky for each orbit, taking into account the regions with low exposure level.} \label{fig_fov}
\end{figure}

\section{The workflow}

The AGILE data are down-linked approximately every orbit and sent
      to the AGILE Data Center (ADC), which is part of the ASI Science
      Data Center (ASDC) for data reduction, scientific processing,
      and archiving. The ASDC then forwards the AGILE data to the AGILE
      team local sites where a quick look analysis is performed.
    
    The overall science monitoring alert system of AGILE is compound
      by two independent pipelines that process the data with different
      data quality results. The INAF/IASF Bologna pipeline processes the
      data in the fastest possible way, loosing the tail of the data
      acquired during an AGILE orbit (some minutes of acquisition) but
      it generates alert within 1.5-2 hours from the time of the last
      GRID event acquired in orbit. The ASDC pipeline is more accurate
      because all events are considered during the analysis but the
      alerts are generated 3-3.5 hours after.
      
      The description of the software and hardware data flow is reported in \citep{2009ASPC..411..362B}.
      
      The data is analyzed each orbit (when is received) and this
      produces a sliding window in the generated light curves of $\gamma$-ray
      sources (see Fig. \ref{fig_crab}). The selection of candidate flares is performed with two independent
      methods:
      (1) with a blind search method called spotfinder
      (2) using a list of known sources from different source
          catalogs.
At the end we obtain a list of candidate flares with its related
      statistical significance.
    For a subclass of selected candidate flares with $\sqrt(T_s) \ge 4.5$ an
      alert is generated and sent via SMS and e-mails.



During the daily monitoring two people are involved to check the
      alerts generated by science alert monitoring, one of the AGILE Team (the institutes
      involved are INAF IASF Bologna, Milan, Palermo, IAPS Rome and
      INFN Trieste) and one at ASDC.
     
     The alerts generated by IASF Bologna pipeline are cross-checked
      with alerts generated by ASDC pipeline. 
      
       For the most interesting candidate flares a manual analysis is performed; the detections that survive to this manual check and are
          above a well defined threshold are taken into consideration
          for an Astronomical Telegram. Usually we use$\sqrt(T_s) \ge 5$,
          but when there is an evidence of activity from other
          wavelenghts a lower significance level is considered. 

In addition an iPhone iOS App has been developed by AGILE Team at INAF/IASF Bologna. This App is divided
      in a public section with news, gallery of images, scientific
      results and in a private, password protected section, where the
      last analyzed data are shown to AGILE Team in an effective way,
      including the sky maps with the possibility of zooming the maps.
      
      
\section{The Crab flare case}

In this case a maximum likelihood analysis determines that the Crab has a persistent flux of
F = $(2.2 +/- 0.1) \cdot 10^{-6} \; ph \; cm^{-2} \; s^{-1}$ for $E > 100$ MeV at a significance of sqrt(TS) = 30.0 with data from 2007 Jul - 2009 Oct, taking into account the diffuse gamma-ray background with Galactic and isotropic components, and is obtained considering all nearby sources with a fixed flux. Fig. \ref{fig_crab} reports the sliding light curve of the  2010 Sep Crab flare as seen by AGILE science monitoring system running at INAF/IASF Bologna. 
The first alert  from Crab with a flux level that exceed $1-\sigma$ mean flux level (1.a in Figure) was received at 2010/09/20 02:04:04 UTC (the yellow arrow 1.b in Figure report the alert generation). The Crab nebula reaches its maximum flux in AGILE data (see 2.a in Figure)
      between 2010/09/19 01:54:43 UTC and 2010/09/20 23:47:51
      in the continuous integration procedure with two days integration
      time. The alert was generated via email and SMS at
      2010/09/21 02:00:54 UTC (see red arrow 2.b), about two hours after the maximum of the physical phenomena.  The ATEL 2855 \citep{Tavani:ATEL} was posted at 2010-09-22 14:45:00 UTC.
      
      The main advantage  of this approach is that the AGILE Team was alerted about the Crab flaring activities at the beginning of the flare itself and thank to the continuous integration time performed by science alert monitoring the AGILE Team was able to follow the evolution of the overall flaring phenomena in near real-time, providing a fast communication via Astronomical Telegram, enabling other observatories to follow the astrophysical phenomena during its evolution.
      


\begin{figure}[t]
\epsscale{0.93}
\plotone{O.05_2.eps}
\caption{The 1.5 hours sliding light curve (2-days integration time) of the September 2010 Crab flare as seen by AGILE science monitoring system. Errors are $1-\sigma$, and time is given in MJD. The yellow lines    show the average Crab flux and the $3-\sigma$ uncertainty range. 1.a and 1.b (in orange) are respectively the detected flux and the time of the alert generation by science monitoring system when Crab nebula reaches a flux level that exceed $1-\sigma$ the mean flux level; 2.a and 2.b (in red) are related to the reached maximum flux level. The green arrow indicate when the Astronomical Telegram was posted.} \label{fig_crab}
\end{figure}

\section{Conclusion}

The fast data processing and alert generation of INAF/IASF
      Bologna pipeline that maximize the fast data processing with
      respect to the data quality and the availability of two
      independent science alert pipelines is one of the key factors for
      the success of the AGILE mission.
    
    The effectiveness of the science alert monitoring system is demonstrated on a daily basis and by the great number of ATel published by AGILE Team; very important discoveries were started from this
      monitoring system, first of all the discovery of the Crab flares
      that earned the 2012 Bruno Rossi prize to Marco Tavani and AGILE
      Team.
      
        This system detects unexpected flaring astrophysical events
          with a very fast reaction time; this allows an effective
          follow-up of flaring sources enabling AGILE to maximize the
          scientific return in the field of short-timescale
          phenomenology.

\acknowledgements The AGILE Mission is funded by the Italian Space Agency (ASI) with 
scientific and programmatic participation by the Italian Institute 
of Astrophysics (INAF) and the Italian Institute of Nuclear 
Physics (INFN). 

\bibliographystyle{asp2010}
\bibliography{../../editor}



\end{document}
