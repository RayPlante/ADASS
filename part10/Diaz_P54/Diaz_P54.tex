
\resetcounters

\bibliographystyle{asp2010}

\markboth{Diaz et al.}{HST Cycle 21 Exposure Time Calculator}

\title{HST Cycle 21 Exposure Time Calculator}
\author{R.~I.~Diaz,$^1$ V.~G.~Laidler,$^2$ I.~Busko,$^1$ M.~Davis,$^1$ C.~Hanley,$^1$ M.~Sienkiewicz,$^1$ C.~Sontag,$^1$ and B. York$^1$
\affil{$^1$Space Telescope Science Institute, 3700 San Martin Dr, Baltimore MD 21218}
\affil{$^2$Computer Sciences Corporation at STScI, 3700 San Martin Dr, Baltimore MD 21218}}

\aindex{Diaz, R. I.}
\aindex{Laidler, V. G.}
\aindex{Busko, I.}
\aindex{Davis, M.}
\aindex{Hanley, C.}
\aindex{Sienkiewicz, M.}
\aindex{Sontag, C.}
\aindex{York, B.}

\begin{abstract}
We will present the most recent updates that will be part of Exposure Time Calculator for 
Cycle 21 (ETC 21.1). We also touch on 
some of the future improvements that are being consider for next Cycle. 
\end{abstract}

\section{Introduction}
The ETCis an integral part of the  Hubble Space Telescope (HST) planning tools, which
evolves over time with the changes in the HST
instruments and with changes in the software and hardware \citep{Diaz3_2010, Diazetal_2010}.
The ETC is used by HST observers
to assess the feasibility of their science programs and to plan their observations.
As such, the ETC calculations should accurately predict the observatory response
for each of the instruments and observing modes \citep{Diaz_2012}.

The current ETC, or pyetc because it is written in python \citep{pyetc2010}, provides the engine to
put together all he pieces of the puzzle together.: instrument response, spectral models for different
sources, normalizations at different passbands, and background contributions that will be used to derive
the observing time needed to achieve a given signal-to-noise (S/N) or the S/N that can be attained after
a given observing time.

The instrument response is divided in two parts: the throughput files for all the HST instruments'
observing modes  and constrains given by  the aperture
transmission, the  fraction of energy contained in the aperture, and the type of source selected (point or extended source).
The sensitivity from the different filters and gratings is contained in throughput tables
that are accessed via the pysynphot package (http://stsdas.stsci.edu/pysynphot/). These tables are updated, if needed,
before each release in order to account for changes in the sensitivity of the detectors \citep{cdbs_2012}.
The rest of the parameters are defined or calculated in the code or included via instrument data files;
which are updated before each release, if needed.

\section{ETC Cycle 21 changes}

With ETC 21.1, only a few changes will be implemented and described below:

\begin{itemize}
\item {\bf{ Improved date configuration for Bright Object checking.}}.

Portions of the ETC software are used to generate tables for
the BOT (Bright Object Tool),
which is used by APT to perform bright object protection
checking. For Cycle
21, this software has been modified to permit configuring
the date used to estimate
instrument sensitivities when generating these tables.  This
is particularly important for
instruments like the Cosmic Origins Spectrograph (COS) and
the Space Telescope
Imaging Spectrograph (STIS) for which there is a variation
of the sensitivity
with time. While the ETC is configured to perform all
calculations using the predicted
values for the middle of the cycle (April 1, 2014), the
configuration is modified for
the software used to generate the BOT tables so that it uses
the predicted values at
the beginning of the Cycle (October 1, 2013). This
conservatively performs the calculations
for the point at which the instruments have the highest
sensitivity, in order to avoid potentially
underestimating cases that might cause a hazard to the detector.

\item {\bf { Extended Zodiacal light contribution}}.
The Sun angle table, used to determine the contribution of the  Zodiacal light 
to the sky background, will be updated with a finer grid that extends closer to  the point where 
it is disallowable due to its closeness to the Sun direction. ETC 20.2 has limits calculations to heliocentric
angles of $60^o$ and $30^o$ degrees, for both heliocentric latitude and longitude ($\beta$, $\lambda$)
combinations.  The APT calculations, on the other hand, are still valid at  $\beta = 46.9^o$ and
$\lambda = 17.72^o$. In Cycle 21  the ETC and the APT will
be consistent, helping users determine the feasibility of their observations at all the angles allowed
by the APT and perform their corresponding BOP screening.

\item {\bf Updates to detector background and Throughput files}.
The dark rates will be updated for COS and STIS;
 along with new  throughput files for most of the
modes. As stated before, the changes in the throughput files incorporate the loss of sensitivity with
time, via Modified Julian date parameterization.

\item {\bf Normalization in the observing band}.
A survey made after the close of the Phase I for Cycle 19 revealed that users were
keen to normalize their spectra using the observed bandpass rather than a different one.
This option will be implemented in ETC 21.1 for all the HST modes; except for STIS modes using
gratings. The normalization units will be ABmag.


\end{itemize}

\section{Future Builds}

In a continuous effort to provide users with the most accurate calculations that cover
the whole range of possible observations with HST, the ETC group is working or considering
the following changes to the tool:

\subsection{Work under consideration}

\begin{itemize}

\item Support central-wavelength depended enclosed energy and extraction hight
for COS. Currently these are  defined for each  grating but the new modes have  
much larger cross-dispersion profiles compared to the other modes, making 
these changes necessary.

\item Include dead-time correction in the ETC calculations; which can
affect considerably the count rates of bright targets.

\item CTE mitigation.  Users of CCDs would benefit by requesting a preflash for their observations. 
A warning making observers aware of the CTE mitigation, if the background counts per pixel are too low,
should be implemented . This
affects mainly ACS and WFC3 and should be applied according to the total background 
counts per pixel.

\item Support calculation for WFC3 spatial scanning mode. This observing mode is currently
supported by WFC3 but not yet in the ETC. Using this observing strategy allows observers to
reduce the fraction of overhead in observations of very bright stars and enables observations 
of very bright primary calibrators that would otherwise saturate the IR detector \citep{TheMcs2011}. 

\item Support WFC3 secondary order calculations. WFC3 provides a slitless spectroscopy mode 
in the IR and UVIS channels. The dispersion of the grism is high enough such that the positive 
first and second-order spectra, (and for some of them the +3 order) as well as the zeroth-order 
image, are visible on the detector. For very bright sources, where the first order saturates, 
the secondary order, which is 90 time less sensitive, can be used.

\item Increase the wavelength range of non-stellar spectral models.

\end{itemize}

\section{Health and Safety of the Instruments}

Besides helping users to achieve their scientific objectives, the ETC assists observers and STScI Scientist to verify that the observations will not damage the detectors.
COS, STIS and ACS have instruments that can be damaged  by observing very bright sources. In order to make sure 
observations are safe, the
ETC has a warning system that identify these conflicting observations. A more extensive check is later done by the APT; which bases its information
on the results derived with the ETC.

\subsection{ETC Warnings}

The ETC determines peak per-pixel count
rates and total (integrated over the detector) count rates to aid observers with the
feasibility assessment. In the case of the
Spectroscopic ETC, the task produces a simulated one-dimensional spectrum for a given STIS
configuration and source, with a graphical interface that allows web browsers to plot the output.

The ETC makes sure the correct warning is displayed by the ETC if observations exceed the 
local or global brightness limits. 
The current
version handles these warning by comparing with the countrate for a given observing mode.


\subsection{Bright Object  Tool and the ETC}

The APT provides a more extensive check for bright objects that could be in the field
of view of the detector. Given the uncertainties of the pointing of the telescope,
checks
for all the star within 15000 (depending on the instrument
used) around the source of
interest. The APT does this via the BOT, which uses tables
that were generated by the ETC.  In this case the BO Tool for Cycle
21 constructs its tables by using ETC 21.1. 

\bibliography{editor}
