
\resetcounters

\bibliographystyle{asp2010}

\markboth{K\"ummel et al.}{Early \ssindex{astronomy!photometry}Photometry Studies for Euclid}

\title{Early \ssindex{astronomy!photometry}Photometry Studies for \ssindex{observatories!space-based!Euclid}Euclid}
\author{M.\ K\"ummel, J.\ Koppenhoefer, A. Riffeser, J.\ Mohr,
S.\ Desai, R.\ Henderson, K.\ Paech, and M.\ Wetzstein
\affil{Universit\"ats-Sternwarte M\"unchen, Scheinerstr. 1, 81679 M\"unchen,
Germany}}

\aindex{K\"ummel, M.}
\aindex{Koppenhoefer, J.}
\aindex{Riffeser, A.}
\aindex{Mohr, J.}
\aindex{Desai, S.}
\aindex{Henderson, R.}
\aindex{Paech, K.}
\aindex{Wetzstein, M.}

\begin{abstract}
The \ssindex{observatories!space-based!Euclid}Euclid mission aims to investigate the distance-\ssindex{astronomy!redshift}redshift relationship and the evolution of cosmic structures by measuring shapes and \ssindex{astronomy!redshift}redshifts of galaxies and \ssindex{astronomy!cluster!galaxy}clusters of galaxies. Data from the satellite instruments will be merged with ground based imaging data from large surveys such as the \ssindex{surveys!Dark Energy Survey (DES)}Dark Energy Survey and \ssindex{surveys!Pan-STARRS}Pan-STARRS. \ssindex{observatories!space-based!Euclid}Euclid's weak galaxy lensing experiment requires very accurate photometric \ssindex{astronomy!redshift}redshifts and thus precise galaxy colours. In this contribution we compare the 'traditional approach' of fitting object models to co-added images with a new technique that simultaneously fits a model to the ensemble of individual images. Preliminary results indicate that the \ssindex{astronomy!photometry}photometry using individual image fitting is more precise.
\end{abstract}

\section{The \ssindex{observatories!space-based!Euclid}Euclid Satellite}
\ssindex{observatories!space-based!Euclid}Euclid \citep{2011arXiv1110.3193L} is a medium class mission candidate expected to be launched in in 2019. \ssindex{observatories!space-based!Euclid}Euclid will measure galaxies and \ssindex{astronomy!cluster!galaxy}clusters of galaxies out to $z\sim2$ in a wide \ssindex{astronomy!extragalactic}extragalactic survey covering $15\,000\,\rm{deg}^2$, plus a deep survey covering an area of $40\,\rm{deg}^2$. The \ssindex{observatories!space-based!Euclid}Euclid payload is a $1.2\,\rm{m}$ Korsch mirror directing the light to its two instruments, the VISible Imaging Channel (\ssindex{instruments!individual!VIS}VIS) and the Near IR Spectrometer and imaging Photometer (\ssindex{instruments!individual!NISP}NISP). Both instruments cover a large common field-of-view of $\sim0.54\,\rm{deg}^2$. In order to compute photometric \ssindex{astronomy!redshift}redshift estimates for Weak gravitational Lensing (WL), one of the mission's primary cosmological probes, the \ssindex{observatories!space-based!Euclid}Euclid mission requires external data from deep surveys such as the \ssindex{surveys!Dark Energy Survey (DES)}Dark Energy Survey \citep[DES]{2012APS..APR.D7007F} and \ssindex{surveys!Pan-STARRS}Pan-STARRS \citep{2002SPIE.4836..154K}.

\section{Organizational Unit Merge (OU-MER)}
The \ssindex{observatories!space-based!Euclid}Euclid data processing is organized in several units. The Organizational Unit Merge (OU-MER) is responsible for developing the photometric algorithms that will be used to create the Euclid catalog from \ssindex{observatories!space-based!Euclid}Euclid instruments and external data. The Euclid catalog is essential for both the cosmological core science cases of the mission as well as for the legacy science projects. In the beginning of the implementation phase of the \ssindex{observatories!space-based!Euclid}Euclid Mission, various algorithms are being prototyped, compared, and eventually selected for a final implementation in the Euclid Science Data Centers.

The precise measurement of galaxy colours is an essential ingredient to achieve accurate photometric \ssindex{astronomy!redshift}redshift estimates for \ssindex{observatories!space-based!Euclid}Euclid's weak galaxy lensing experiment. Accurate \ssindex{astronomy!photometry}photometry is usually achieved by fitting adequate models to the object's intensity distribution on deep images that are co-added from several single images. Co-added images have several limitations which may reduce the photometric accuracy, such as complicated point-spread-functions (PSF's) resulting from the differing seeing \ssindex{observing!conditions}conditions on the individual \ssindex{observing!exposure}exposures and correlated noise introduced by rebinning. In the work presented here we investigate the alternative approach of simultaneously fitting galaxy models to the ensemble of single \ssindex{observing!exposure}exposure images and compare the photometric accuracy of the two methods.

%
\begin{figure}[t]
\plotone{part10/Kuemmel_P049/P049_f1.eps}
\caption{The processing \ssindex{data!pipelines!reduction}pipeline used for this work. Individual \ssindex{observing!exposure}exposure images including stars and galaxies are modelled with different observing \ssindex{observing!conditions}conditions. The images are co-added and models are fitted to the galaxy shapes (right). Alternatively, the galaxy shapes are directly fitted to the ensemble of individual images (left).}
\label{Kufig1}
\end{figure}
%
\section{Modeling Survey Images}
To generate imaging data for the comparison we use the processing \ssindex{data!pipelines!reduction}pipeline illustrated in Figure \ref{Kufig1}. Taking star and galaxy distributions from \ssindex{surveys!Dark Energy Survey (DES)}DES \ssindex{astronomy!simulation}simulations \citep{2010AAS...21547007L}, we generate $10$ images for each run with typical observing \ssindex{observing!conditions}conditions (seeing\ssindex{astronomy!point spread function (PSF)} PSF, zeropoint, sky brightness etc.) as expected for \ssindex{surveys!Dark Energy Survey (DES)}DES. Stars are simulated with a Gaussian\ssindex{astronomy!point spread function (PSF)} PSF, galaxies are simulated as Sersic profiles \citep{1968adga.book.S}, folded with the (Gaussian) image\ssindex{astronomy!point spread function (PSF)} PSF. Each image is shifted by a small ($\sim5$\,pix), random $x$- and $y$- offset from a central position.

For the co-added image fitting \ssindex{astronomy!photometry}photometry, the $10$ individual \ssindex{observing!exposure}exposures are co-added using {\tt \ssindex{applications!SWarp}SWarp}\ooindex{SWarp, ascl:1010.068} (see Figure~\ref{Kufig1}). On the co-added image, objects are detected with {\tt \ssindex{applications!SExtractor}SExtractor}\ooindex{SExtractor, ascl:1010.064}  and the\ssindex{astronomy!point spread function (PSF)} PSF is determined with {\tt PSFEx} \citep{2011ASPC..442..435B}. We then measure the object brightness by fitting Sersic ($I(a) = I_e \exp(-b_n[(\frac{a}{r_e})^{1/n}-1])$) models to each galaxy on the co-added image, taking into account the\ssindex{astronomy!point spread function (PSF)} PSF as determined by {\tt PSFEx}. For the individual image fitting \ssindex{astronomy!photometry}photometry, the\ssindex{astronomy!point spread function (PSF)} PSF on each individual image is determined using {\tt PSFEx}. The positions in the object list from the co-added image are projected onto the individual images, and we simultaneously fit Sersic profiles to the galaxy shapes on the individual images, taking into account the known pixel shifts between the images and their respective\ssindex{astronomy!point spread function (PSF)} PSF. To minimize the \ssindex{statistical analysis}statistical errors of the fitting results, we typically run many ($\sim100$) \ssindex{astronomy!simulation}simulations and use the mean value for each object for the comparison.

Both, the galaxy modeling and the Sersic profile fitting have been done using an extended version of the {\tt imfit} software package (\url{http://www.mpe.mpg.de/~erwin/code/imfit}). 
\begin{figure}[t]
\plotone{part10/Kuemmel_P049/P049_f2.eps}
\caption{Left: Deep co-added image for one \ssindex{astronomy!simulation}simulation run. Upper right: Co-added stamp image of one galaxy. Lower right: Individual stamp images for the same galaxy. Sersic models are fitted to these co-added and individual stamp images.}
\label{fig2}
\end{figure}

\section{Comparison of the Two Photometric Methods}
The left hand side of Figure \ref{fig2} shows the deep co-added image for one \ssindex{astronomy!simulation}simulation run. On the upper right the deep stamp for one object is shown, and on the lower right are the corresponding stamps of the individual images. The co-added and the individual image Sersic fits have been performed on the co-added and individual stamp images, respectively.
\begin{figure}[t]
\plotone{part10/Kuemmel_P049/P049_f3.eps}
\caption{Comparison of the photometric results from the co-added image fits (red) and the individual image fits (blue) for all simulated galaxies. The intensity measured by the respective fitting technique is divided by the modeled (input) intensity and shown as function of the SNR.}
\label{fig3}
\end{figure}

Figure \ref{fig3} shows a comparison of the \ssindex{astronomy!photometry}photometry from the co-added image fits (red) and individual image fits (blue) for all simulated objects. The data points show the intensity computed from the fitted Sersic parameters divided by the intensity of the input Sersic model as a function of the signal-to-noise ratio (SNR, from {\tt \ssindex{applications!SExtractor}SExtractor}\ooindex{SExtractor, ascl:1010.064}).

Both the co-added image fit and the individual image fit have values deviating from the ideal value of $1.0$, however the co-added image fitting method results in more outliers ($16\%< 0.98$ with respect to $6\% < 0.98$ for the individual image fit). Also the measurements from the co-added image fits show a larger systematic trend ($\sim1\%$) over the range of SNR's plotted in Figure~\ref{fig3} as compared to the individual image fits ($\sim0.5\%$). A close examination of objects with low values in the co-added image fit reveals that they also tend to have low values in the individual image fit, which indicates that these values are not caused by a problem with the co-added image fit, but rather a generic result for these objects that have rather large scale lengths $r_e$ and Sersic indices $n$.

\section{Summary and Outlook}
These initial results indicate that the method of fitting Sersic profiles to individual images yields a more accurate \ssindex{astronomy!photometry}photometry than the 'traditional' approach of fitting on the co-added image. Further steps to explore this new individual image fitting technique include refining the processing of the co-added images by e.g. adding the\ssindex{astronomy!point spread function (PSF)} PSF homogenization that is applied in \ssindex{surveys!Dark Energy Survey (DES)}DES \citep{2008SPIE.7016E..17M}, adding the pixel error in \ssindex{software!image processing}image processing and profile fitting and exploring the numerical details of the profile fitting which currently uses a standard Levenberg Marquardt minimization algorithm.

\bibliography{editor}
