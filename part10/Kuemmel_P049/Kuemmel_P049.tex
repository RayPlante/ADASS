
\resetcounters

\bibliographystyle{asp2010}

\markboth{K\"ummel et al.}{Early Photometry Studies for Euclid}

\title{Early photometry studies for Euclid}
\author{M.\ K\"ummel, J.\ Koppenhoefer, A. Riffeser, J.\ Mohr,
S.\ Desai, R.\ Henderson, K.\ Paech, and M.\ Wetzstein
\affil{Universit\"ats-Sternwarte M\"unchen, Scheinerstr. 1, 81679 M\"unchen,
Germany}}

\aindex{K\"ummel, M.}
\aindex{Koppenhoefer, J.}
\aindex{Riffeser, A.}
\aindex{Mohr, J.}
\aindex{Desai, S.}
\aindex{Henderson, R.}
\aindex{Paech, K.}
\aindex{Wetzstein, M.}

\begin{abstract}
The Euclid mission aims to investigate the distance-redshift
relationship and the evolution of cosmic structures by measuring shapes and
redshifts of galaxies and clusters of galaxies. Data from the satellite
instruments will be merged with ground based imaging data from large
surveys such as the Dark Energy Survey and Pan-STARRS. Euclid's weak galaxy lensing
experiment requires very accurate photometric redshifts and thus precise galaxy colours.
In this contribution
we compare the 'traditional approach' of fitting object models to co-added
images with a new technique that simultaneously fits a model to the ensemble of individual
images. Preliminary results indicate that the photometry using individual
image fitting is more precise.
\end{abstract}

\section{The Euclid satellite}
Euclid \citep{2011arXiv1110.3193L} is a medium class mission candidate expected to be launched in in 2019. 
Euclid will measure galaxies
and clusters of galaxies out to $z\sim2$ in a wide extragalactic survey
covering $15\,000\,\rm{deg}^2$, plus a deep survey covering an area of $40\,\rm{deg}^2$.
The Euclid payload is a $1.2\,\rm{m}$ Korsch mirror directing the light to its two instruments,
the VISible Imaging Channel (VIS) and the Near IR Spectrometer and imaging
Photometer (NISP). Both instruments cover a large common field-of-view of $\sim0.54\,\rm{deg}^2$.
In order to compute photometric redshift estimates for Weak gravitational Lensing (WL),
one of the mission's primary cosmological probes, the Euclid mission requires external data
from deep surveys such as the Dark Energy Survey \citep[DES]{2012APS..APR.D7007F} and
Pan-STARRS \citep{2002SPIE.4836..154K}.

\section{Organizational Unit Merge (OU-MER)}
The Euclid data processing is organized in several units.
The Organizational Unit Merge (OU-MER) is responsible for developing the photometric
algorithms that will be used to create the Euclid catalog from Euclid instruments and
external data. The Euclid catalog is essential for both the
cosmological core science cases of the mission as well as for the
legacy science projects. In the beginning of the implementation phase of
the Euclid Mission, various algorithms are being
prototyped, compared and eventually selected for a final implementation in
the Euclid Science Data Centers.

The precise measurement of galaxy colours is an essential ingredient to achieve
accurate photometric redshift estimates for Euclid's weak galaxy lensing experiment.
Accurate photometry is usually achieved by fitting adequate models to the object's
intensity distribution on deep images that are co-added from several single images.
Co-added images have several limitations which may reduce the photometric
accuracy, such as complicated point-spread-functions (PSF's) resulting from the
differing seeing conditions on the individual exposures and correlated noise
introduced by rebinning. In the work presented here we investigate
the alternative approach of simultaneously fitting galaxy models to the ensemble of single exposure
images and compare the photometric accuracy of the two methods.

%
\begin{figure}[t]
\plotone{part10/Kuemmel_P049/P049_f1.eps}
\caption{The processing pipeline used for this work. Individual exposure images including
stars and galaxies are modelled with different observing conditions. The images are co-added
and models are fitted to the galaxy shapes (right). Alternatively, the galaxy shapes
are directly fitted to the ensemble of individual images (left).}
\label{Kufig1}
\end{figure}
%
\section{Modelling survey images}
To generate imaging data for the comparison we use the processing pipeline illustrated
in Figure \ref{Kufig1}. Taking star and galaxy distributions from DES simulations
\citep{2010AAS...21547007L}, we generate $10$ images for each run
with typical observing conditions (seeing PSF, zeropoint, sky brightness etc.)
as expected for DES. Stars are simulated with a Gaussian PSF, galaxies
are simulated as Sersic profiles \citep{1968adga.book.S}, folded with the (Gaussian)
image PSF. Each image is shifted by a small ($\sim5$\,pix), random $x$- and $y$- offset from a
central position.

For the co-added image
fitting photometry, the $10$ individual exposures are co-added using {\tt SWarp}
(see Fig.\ \ref{Kufig1}). On the co-added image,
objects are detected with {\tt SExtractor} and the PSF is determined
with {\tt PSFEx} \citep{2011ASPC..442..435B}.
We then measure the object brightness by fitting
Sersic ($I(a) = I_e \exp(-b_n[(\frac{a}{r_e})^{1/n}-1])$)
models to each galaxy on the co-added image, taking into account the PSF as determined
by {\tt PSFEx}.
For the individual image fitting photometry, the PSF on each individual image is determined using
{\tt PSFEx}. The positions in the object list from the co-added image are projected onto the
individual images, and we simultaneously fit Sersic profiles to the galaxy shapes on the individual images,
taking into account the known pixel shifts between the images and their respective PSF.
To minimize the statistical errors of the fitting results, we typically run many ($\sim100$)
simulations and use the mean value for each object for the comparison.

Both, the galaxy modelling and the Sersic profile fitting have been done using an extended
version of the {\tt imfit} software
package (\url{http://www.mpe.mpg.de/~erwin/code/imfit}).
\begin{figure}[t]
\plotone{part10/Kuemmel_P049/P049_f2.eps}
\caption{Left: Deep co-added image for one simulation run. Upper right: Co-added stamp image
of one galaxy. Lower right: Individual stamp images for the same galaxy. Sersic models are fitted
to these co-added and individual stamp images.}
\label{fig2}
\end{figure}
\section{Comparison of the two photometric methods}
The left hand side of Figure \ref{fig2} shows the deep co-added image for one simulation run.
On the upper right the deep stamp for one object is shown, and on the lower right
are the corresponding stamps of the individual images. The co-added and the individual
image Sersic fits have been performed on the co-added and individual stamp images, respectively.
\begin{figure}[t]
\plotone{part10/Kuemmel_P049/P049_f3.eps}
\caption{Comparison of the photometric results from the co-added image fits (red) and the
individual image fits (blue) for all simulated galaxies. The intensity measured by the respective
fitting technique is divided by the modelled (input) intensity and shown as function of the
SNR.}
\label{fig3}
\end{figure}

Figure \ref{fig3} shows a comparison of the photometry from the co-added image fits (red)
and individual image fits (blue) for all simulated objects. The data points show the
intensity computed from the fitted Sersic parameters divided by the intensity of the
input Sersic model as a function of the signal-to-noise ratio (SNR, from {\tt SExtractor}).

Both the co-added image fit and the individual image fit have values deviating from
the ideal value of $1.0$, however the co-added image fitting method results in more
outliers ($16\%< 0.98$ with respect to $6\% < 0.98$ for the individual image fit).
Also the measurements from the co-added image fits show a larger
systematic trend ($\sim1\%$) over the range of SNR's plotted in Fig.\ \ref{fig3} as compared
to the individual image fits ($\sim0.5\%$). A close examination
of objects with low values in the co-added image fit reveals that they also
tend to have low values in the individual image fit, which indicates that
these values are not caused by a problem with the co-added image fit, but rather
a generic result for these objects that have rather large scale lengths $r_e$
and Sersic indices $n$.

\section{Summary and outlook}
These initial results indicate that the method of fitting Sersic profile to
individual images yields a more accurate photometry 
than the 'traditional' approach of
fitting on the co-added image. Further steps to explore this new individual image
fitting technique include refining the processing of the co-added images by e.g.
adding the PSF homogenization that is applied in DES \citep{2008SPIE.7016E..17M},
adding the pixel error in image processing and profile fitting and exploring the
numerical details of the profile fitting which currently uses a standard
Levenberg Marquardt minimization algorithm.

\bibliography{editor}
