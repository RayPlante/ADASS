% This is the aspauthor.tex LaTeX file
% Copyright 2010, Astronomical Society of the Pacific Conference Series

\documentclass[11pt,twoside]{article}
\usepackage{asp2010}

\newcommand{\xmm}{{XMM-{\em Newton} }}

\resetcounters

\bibliographystyle{asp2010}

\markboth{A. Ibarra, M. Kennedy, et al}{\xmm Mobile Web Application}

\begin{document}

\title{XMM-Newton Mobile Web Application}
\author{Ibarra, A., Kennedy, M., Rodr\'iguez, P., Hern\'andez, C., Saxton, R. and Gabriel, C.}
\affil{XMM-Newton SOC, European Space Astronomy Centre (ESAC) / ESA, Madrid, Spain}

\begin{abstract}
We present the first \xmm web mobile application, coded using new web
technologies such as HTML5, the jQuery mobile framework and D3
JavaScript data-driven library.

This new web mobile application focuses on re-formatted contents
extracted directly from the XMM-Newton web, optimizing the contents for
mobile devices. 

The main goals of this development were to reach all kind of handheld
devices and operating systems, while minimizing software
maintenance. The application therefore has been developed as a web mobile
implementation rather than a more costly native application. New
functionality will be added regularly.

\end{abstract}

\section{Introduction}

\xmm (~\cite{Jansen}) as a project, has as it's fundamental vehicle
for mission information, a dedicated website. There are several
reasons behind the development of an \xmm, mobile web application. The
most important one is that more and more \xmm observers try to reach
information contained in the standard web pages through handheld
devices, but a large part of the information contained there is far
from being optimized for this medium. And other reason is the quick
growth of the mobile devices market in the last years.  Both
smartphones and tables have emerged abruptly not only into the
information technology world, but also into the scientific
world. Therefore, the \xmm web mobile application is a step forward in
the modernization of the Internet services provided by the project.\\
 
There are two main issues when a standard web portal is accessed from a mobile device:
\begin{itemize}
\item Limited screen size.
\item Too much content.
\end{itemize}

It is of no use to display in a handheld device a full web page with
large quantities of text, links and images. Mobile device users are
continuously forced to resize the content to read the information. The
small screen size of mobile devices implies a simple or minimalist user
interface and content tailoring.\\

We have therefore developed this \xmm web mobile application taking
all this into account and trying to minimize the code maintenance. The
contents are automatically extracted from the official \xmm project
website. We have reduced the contents, showing only the most valuable
information. It is easy to surf between different pages. It has been
coded using new web technologies such as HTML5 and the jQuery mobile
framework\footnote{http://jquerymobile.com}.

\section{The \xmm web mobile application}

The \xmm web mobile application has been optimized to be displayed on
mobile devices. The navigation within the mobile app is controlled by
jQuery Mobile framework through ajax, giving the web app a look and
feel similar to a native application. In Figure 1. we show the concept
of the \xmm web mobile application.\\

The \xmm web mobile application is structured as following.
\begin{itemize}
\item \xmm satellite image.
\item Satellite daily events. Here we show live information such as: current revolution, 
current observation, mission operation duration and number of refereed papers based on 
\xmm data.
\item The main body. Several elements, described below, constitute the 
bulk of the web mobile application contents.
\end{itemize}

{\bf Highlights and Events:} this section, further structured into
Science News and Highlights, is based on all the extracted news,
highlights and events present in the the standard \xmm
web. Considerable reformatting is performed for optimal display in
mobile devices.\\

{\bf Live:} under this section the \xmm real-time information is available,
offering the first version of a visibility checker tool and a current
position locator.\\ 

In this section we have concentrated most of the project effort,
dedicated to new developments. Here we have used HTML5 technology to
interact with onboard mobile device features such as GPS. We have also
developed a suite of astronomical JavaScript functions to:
\begin{itemize}
\item convert coordinates for different reference systems,
\item calculate the \xmm attitude at any time using Two Line Elements,
\item calculate apparent positions of Solar System objects with respect to the \xmm position,
\item include Virtual Observatory interfaces to resolve target names.
\end{itemize}

With all these functions, we have been able to create a new dynamic
graphical version of a target visibility checker. Embedded in a
thin-layer web based interface, we calculate target visibility periods
for any \xmm revolution. This tool also shows the variations of the
avoidance regions within the selected \xmm orbit. The forms have been
created using jQuery mobile framework and the data visualization has
been developed using the D3 JavaScript
library\footnote{http://d3js.org}. D3 is a library for manipulating
documents based on data. \\

Under the {\bf Live} section we have also developed the current \xmm
position locator. This is a 3D visualization of the \xmm orbit with
the projection onto the Earth surface of the \xmm orbital track. An
animation of the complete current \xmm revolution is also available. \\

We plan further developments in this section, including usage of the
device gyro for obtaining a planetarium, highlighting the \xmm observations.\\

{\bf Mission overview:} in this section basic information about the satellite and
the mission concept can be found. We also link here the
most important \xmm documents for further reading. \\

\begin{figure}[h]
\epsscale{1.2}
\plotone{P015_1.eps}
\caption{ The new \xmm web mobile application extracts automatically
contents from the standard web pages. Then the information is formatted
to be displayed in handled devices.} \label{P061-fig-1}
\end{figure}


{\bf Calendar:} section where important \xmm events are recorded in a
calendar view. Clicking on any of the events, the user can
automatically import any event to his/her agenda. As of today, it is
only possible to add events to the google calendar. \\

{\bf Image gallery:} here we show a collection of astronomical images
and spectra taken with \xmm X-ray and optical instruments along with
other \xmm related images. This gallery of the official \xmm
website is a good example of web pages which are difficult to be
handled with a mobile device. Avoiding duplication of code, we have
reused the whole image gallery server infrastructure to extract the
information from the standard web pages.  Information and forms have
been designed and formatted to be displayed properly in a mobile
device, using jQuery mobile framework.\\

The last item in the main web page is {\bf Full web page} link that
redirect users to the official \xmm web pages, just in case users would
like to have access to them. \\

Making use of HTML5 technology, the \xmm web mobile application can be
installed in your desktop screen and will appear on it with an icon
similar to any native application. Once the web mobile app runs
through this icon, the web browser navigation and URL menu disappear,
giving the application a look and feel similar to a native
application.

\section{Web mobile vs native app}

The two ways to convey information and services to a handheld device
are web mobiles and native applications. Both ways present pros and
cons, which in our opinion have to be ranked according to the actual
needs of the project behind. The web mobile pros are simplicity,
cheaper development and maintenance costs, simple access (single URL)
and platform independence. The strengths of native applications lye in
the possibility of specific developments for a particular device and
OS, therefore taking advantage of the ability to leverage device
specific hardware and software. Also the fact that these can run
offline is an advantage in front of the web mobile, which needs a
connection to work (although HTML5 offers the possibility of an
off-line usage).\\

Cons of the web mobile are mainly the network connectivity mentioned
above and the limited capabilities and limited access to the onboard
hardware. However the HTML5 standards are evolving quickly towards
reducing these limitations. Fundamental for our decision against
developing a native application were the associated higher costs in
development and maintenance. In this way, we have also avoided forcing
users to download the app from a central repository.\\

Nevertheless, we reckon advantages and disadvantages by both
technologies. The final decision to build a native app or a web mobile
will depend ultimately on several factors such as: technical
requirements, customer profiles or budget constraints.

\bibliography{P015}

\end{document}
