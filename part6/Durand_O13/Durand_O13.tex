% This is the aspauthor.tex LaTeX file
% Copyright 2010, Astronomical Society of the Pacific Conference Series

\documentclass[11pt,twoside]{article}
\usepackage{asp2010}

\resetcounters

\bibliographystyle{asp2010}

\markboth{Fernique et al.}{HEALPix based cross-correlation in Astronomy}

\begin{document}

\title{HEALPix based cross-correlation in Astronomy}
\author{Pierre Fernique$^1$,\ Daniel Durand$^2$,\ Thomas Boch$^1$, \ Ana\"is Oberto$^1$, \ Fran\c{c}ois-Xavier Pineau$^1$
\affil{$^1$Centre de Donn{\'e}es de Strasbourg, Strasbourg, France}
\affil{$^2$Herzberg Institute of Astrophysics, National Research Council, Canada}}



\begin{abstract}
We are presenting our work on a cross correlation system based on
        HEALPix cells indexing. The system allows users to answer
        scientific
        questions like ``please find all HST images on which there is an
        observation of a radio quiet quasar'' in a single query. The baseline
        of this system is the creation of the HEALPix indexes grouped
        hierarchically and organized in a special format file called MOC
        (see \url{http://ivoa.net/Documents/Notes/MOC}) developed by the CDS. Using the MOC files,
         the cross correlation between images and or catalogs is reduced to searches only in meaning areas. Under the condition that the
        survey data base also uses internally an HEALPix positional index,
        the search result comes back almost immediately (typically a few
        seconds).
        We have started building the index for some surveys, catalogs (VizieR
        catalogs, Simbad, ...) and some pointed mode archives (like HST at
        CADC) and are developing an elementary library to support basic operations
        on any input MOC files. The usage of the MOC files is starting to
        be used though the VO community as a general indexing method and
        tools such as Aladin and TOPCAT are starting to
        make use of them.

\end{abstract}

\section{Selecting a sky index}
In this paper, we are presenting a very efficient method for answering questions like: find all HST images on which there is an observation of a radio quiet quasar;\ find regions of the sky which have been observed both by HST(optical), SCUBA(mm) and XMM(X-rays);\ do my lists objects have been already observed by Spitzer; \ retrieve all 2MASS observations mapping my set of images. 
The key to answer these questions is to able to perform fast comparison of sky coverage between series of observations or measurements. If we can easily generate spatial footprints from observations (either catalogs or pixel surveys), operate basic operations such as equality test, intersection, union, subtraction and complement on these footprints, we will then have a solution for all these problems. Presently, most spatial searches on astronomical databases are based on simple
cone queries, i.e. RA and Dec with a searching radius. We need to do more and manipulates not only circles but complex areas on the sphere.
The determination of a specific area on the celestial sphere is a classical geometric issue.
The basic method consists in defining an area as the union of basic shapes
(polygons, circles, ...) on the sphere (along great and small circles) implemented by STC (\cite{STC}). This method is simple but has one main disadvantage: there is no canonical form to describe a specific area, so comparison is difficult, often complex, and usually slow. We believe that a unique spatial index system is required if one want to use the data from multiple sources covering vast number of collections. Fast methods are generally based on regular partitioning of the sphere (the sky) in regular cells with sizes adapted to the resolution of the input data and compare the ``presence'' or ``absence'' of sources (catalogs) or illumination (pixel data) in these cells.
This process called tessellation is a well known mathematical tool designed to deal with spherical information.
The three main ones used in astronomy are: the Hierarchical Triangular Mesh \cite{HTM} (HTM),
the Hierarchical Equal Area isoLatitude Pixelization \cite{Healpix} (HEALPix) and the Q3C \cite{quadtreee}. Even if these systems are similar, they are not compatible. So for keeping fast operations (intersection, union, ...), we need to choose one of them. We have selected HEALPix as the tessellation system of choice. The reasons of our choice has been fully described in  \cite{Fernique}. In resume, the main argument to select the HEALPix tessellation scheme is to be able to get cells covering equal area. The computation time is also a constant for a given cell/level which is not the case with HTM. Also, the algorithm is quite simple and is already implemented in a number of languages and adopted for multiple missions  (WMAP, PLANCK, GAIA).


\section{Using the sky index}
After selecting HEALPix as the tessellation of choice, we derived from it four products: a multi-resolution visualization method, density maps (HEALPix map), coverage maps (MOC), and data access. The visualization system and density maps have been described in previous papers \cite{Fernique}, let's therefore discuss the MOC and data access.

\subsection{The Multi-Order Coverage maps or MOC files}
A MOC can be considered as a generic tool which can be used for various purposes. MOC files are HEALPix cells number packaged in a file for a given survey/catalogue in a FITS or ASCII file. The numbers are grouped recursively the valid contiguous cell in a hierarchical way. This provides a simple and useful method of describing any sky area. 
As an example, the MOC file of an image from the scuba (JCMT) survey
will only consist of the lowest resolution cells at the
border since the inner part will content only the few cells
at the lowest resolution, thus the MOC file is quite small.
See Fig. \ref{Fig1}.
For example, the MOC file for all the HST images for filter F555 is only 180 Kbytes! As the operations on MOC are very fast (a few milliseconds), what is usually done with the MOC files is using some logic
operators in order to create a ``resultant'' MOC files which
content the footprint of the answer. For example let's take
two series of observations, A and B, and a catalogue
C. The MOC operator could then perform operations like for objects in C which have z $>$ 2.0, give me all the areas of the sky observed by A and B.
\articlefigure{O013_f2.eps}{Fig1}{Scuba I images showing MOC superposed.}
The MOC encoding method has been described in an IVOA document and a java library is available and already implemented in the well known VO tools Aladin  \url{http://aladin.u-strasbg.fr} and TOPCAT  \url{http://www.star.bris.ac.uk/~mbt/topcat/}

\subsection{Data access}
A MOC file itself does not retain the list of original files or sources, it is describing, it is just a coverage descriptor. But under the condition that original data providers have built their data base index using the same HEALPix mechanism, access to original data becomes extremely efficient: each cell of MOC correspond to one index cell of the data base. 
To support this, the CDS has re-indexed using HEALPix a large number of survey catalogs and provides a new service called "query by MOC" which is able to return all sources from one dedicated catalog inside a MOC. Similarly, the CADC have been re-indexed all HST images using the HEALPix library and generated not only the MOCs of these observations, but the list of input which was used to produce the MOC. This allows the user to get back the original images links at any moment while performing MOC operations.


\section{Use cases}
We studied a few use cases mostly to show the entire processes which could be used in: create MOCs from various data sets (Simbad,VizieR tables, pixel surveys like HST, SDSS, XMM); find all the Quasars in CFHTLS, not yet observed by SLOAN for with we have both an XMM and an HST observations; find all HST infrared observations which coincide with bright NOMAD sources (see Fig. \ref{Fig2}).

\articlefigure{O013_f3.eps}{Fig2}{HST image list and NOMAD sources inside a specific MOC (HST F606W)}

\section{What's next}

We propose using the MOC files as defined to become an IVOA standard. 
The Virtual Observatory project would greatly benefit from adopting a common indexing scheme for all the various types of data collections it embraced. 
An attractive but simple form of virtual observatory user query is
to request all records from the whole VO at a given sky position.
While this is in principle possible by dispatching one narrow
positional query to every registered Cone Search, SSA or SIA service,
in practice the number of queries required leads to an unacceptable
load on both clients and services, and moreover most of these queries
will deliver no results because most services lack coverage in the
queried region. If footprint information is available for all registered services, only those
services with coverage in the region of interest, and hence which
might provide a non-empty result set, could be easily identified,
providing a great reduction in the number of service queries required.
The MOC usage offers the opportunity to provide
this coverage information in a uniform way. The MOC could be stored centrally or locally.

\bibliography{O013}

\end{document}
