
\resetcounters

\markboth{Winkelman and Rots}{An Observation-Centric View of the \ssindex{archives!individual!Chandra Data Archive}Chandra Data Archive}

\title{An Observation-Centric View of the \ssindex{archives!individual!Chandra Data Archive}\ssindex{observatories!space-based!Chandra}Chandra Data Archive}
\author{Sherry Winkelman and Arnold Rots}
\affil{Smithsonian Astrophysical Observatory}

\aindex{Winkelman, S.}
\aindex{Rots, A.}

\begin{abstract}
The \ssindex{archives!individual!Chandra Data Archive}\ssindex{observatories!space-based!Chandra}Chandra Data Archive is more than \ssindex{observatories!space-based!Chandra}Chandra data. It also includes proposal abstracts, observation parameters, and a bibliography.  In addition, there are logs and databases which record user interactions with archive.  It's the data which tie these seemingly disparate snapshots of the archive together.  By examining the interdependence between the various facets of the archive we can better visualize how our data is being utilized within the astronomical community and perhaps discover new or improved ways to serve the science needs of the community.

\end{abstract}

\section{Introduction}
The \ssindex{archives!individual!Chandra Data Archive}Chandra Data Archive\footnote{\url{http://cxc.harvard.edu/cda/}} (CDA) plays a crucial role in the Chandra X-ray Center (CXC) which manages the operations of the observatory.  The archive contains more than \ssindex{observatories!space-based!Chandra}Chandra data.  It contains proposals submitted to the observatory; scheduling information; observation parameters; a complete processing history of all data; and a bibliography.  In addition, the archive operations group maintains a search and retrieval database which records who searches for data; who downloads data; and where the data go.  It's the data which tie these disparate snapshots of the archive together.  So what significant events occur to the data during their lifetime? Can they tell us anything about the science impact of the archive? Can they provide a new way of revealing the data to the astronomical community?  In the end, an observation-centric view of the CDA can help us visualize how our data fit into the astronomical community and can provide insight into how the archive can best serve the the science needs of the community.

\section{Time to First Download}
\label{sec:Time2FirstDownload}

\begin{figure}[top]
\plotone[height=1.75in]{part6/Winkelman_O33/O33_f1.eps}
\caption[FirstDownload] 
  { \label{fig:FirstDownload}
The plot on the left shows the total percentage of observations downloaded for the first time.  The purple curves are public data which had a proprietary period of: $\leq 6$ months in light purple; 12 months in medium purple; and no proprietary time in dark purple. The green curves are proprietary data with a proprietary period of 12 months in light green and $\leq 6$ months in dark green.  The plot on the right shows that the release of some data is anticipated by the astronomical community.  The symbols represent downloads during the first week of being public in blue squares; the second week of being public in orange diamonds; the third week of being public in yellow triangles; and the fourth week of being public in green triangles.
}
\end{figure}
 
Most data in the CDA have a proprietary period of 12 months.  During this time the principal investigators (PI) have exclusive access to their data.  They are notified when the data are available for download; that notification starts the proprietary period.  So how long do PIs wait to download their data?  The left plot in Figure~\ref{fig:FirstDownload} shows that data with a shorter proprietary period are downloaded more quickly than data with the standard 12 month proprietary period and that $\sim$50\% of the data are downloaded with 3 days of notification.  When we look at the time to first download of data after being released to the public, we see that 40\% of the data are downloaded within the first week of going public, but 25\% of the data have not been downloaded after being public for 6 months.

\ssindex{observatories!space-based!Chandra}Chandra data are released to the public in batches twice a day which could lead to a 12 hour delay from the posted time of release to when the data are actually publicly available.  The CDA has received several inquiries from users awaiting the release of particular datasets, so we know that the public release of some data is highly anticipated.  Can we see this behavior in our \ssindex{data!metadata}metadata?  As shown in the right plot of Figure~\ref{fig:FirstDownload}, when we plot the number of observations downloaded within first, second, third and fourth week of going public versus the number of observations in the download, we see that within the first week of going public a noticeable number of downloads are in download sessions with one to seven other observations downloaded with the observation in question.  This suggests that those downloads are the result of a specific inquiry.  The trend drops dramatically in the second, third, and fourth week of being public which indicates that the release of some data is anticipated by the astronomical community.

This result has led us to explore providing a subscription service announcing important events related to data.  To be useful to researchers, the CDA will need to be clever at aggregating observations in ways that allow researchers to subscribe to a class of objects and have that class include new objects as they are added to the archive.

\section{Downloads and Aggregates}

\begin{figure}[ht]
\plotone[height=3.25in]{part6/Winkelman_O33/O33_f2.eps}
\caption[DownloadsVsPubs]
 { \label{fig:DownloadsVsPubs}
These scatter diagrams show the number of downloads versus the number of publications for public \ssindex{observatories!space-based!Chandra}Chandra data.  The top plot highlights the \ssindex{observatories!space-based!Chandra}Chandra Deep Fields, the C-Cosmos Survey, and a variety of long observing sequences which were observed in 2-4 chunks of time and released as a group.  The bottom plot focuses on the long observing sequences highlighted in the top plot. }
\end{figure}

Much of the data in the CDA are observationally linked to other data in the archive.  Long observations can be split into smaller segments, observations are observed in a monitoring fashion, or as a group in a grid pattern.  For much research, the aggregate dataset should be used, but the archive user may not be aware of the complete dataset.  By looking at the number of downloads versus the number of publications of observations we can learn something about how the astronomical community is using data aggregates.  Not too surprising, Figure~\ref{fig:DownloadsVsPubs} shows that observations which are often published are downloaded more frequently, although this correlation is rather loose.  The outliers to the right are observations which are downloaded at a high frequency compared to their publication rates, while the clumps at the top represent observations which are highly published compared to their download rate.  The clumps turn out to be the \ssindex{observatories!space-based!Chandra}Chandra Deep Fields while the far right outlier is ObsId 1. 

The top diagram in Figure~\ref{fig:DownloadsVsPubs} clearly demonstrates the success of the \ssindex{observatories!space-based!Chandra}Chandra Deep Field observing programs and the C-Cosmos Survey, a recent very large observing program.  The bottom diagram indicates how long observing sequences are being used.  A spread in the horizontal direction implies that the data are not being downloaded as a group, while a spread in the vertical direction implies the segments are not all published together.  It is the spread in the number of downloads of each segment of the observing sequence which the CDA can address.  We are in the process of linking these aggregates and exposing that linkage to archive users when they download data to make them aware of the associated data.    

\section{Descriptive Data Profiles}

\begin{figure}[ht]
\plotone[height=1.5in]{part6/Winkelman_O33/O33_f3.eps}
\caption[WordCloud]
 { \label{fig:WordCloud}
Word cloud composed of the proposal keywords assigned by PI and the keywords in papers linked to an observation of Abell 1835.
 }
\end{figure}

The CDA has collected \ssindex{data!metadata}metadata which describe the data in the archive in an indirect fashion:  PIs submit keywords with their proposals; authors attach keywords to their papers; and the CDA links data to publications.  From the \ssindex{data!metadata}metadata we can create a descriptive profile of observations.  Figure~\ref{fig:WordCloud} shows a word cloud of the these keywords for ObsId 495 which is a 20 ks observation of Abell 1835.  These profiles may be a way for the CDA to classify data for the subscription service mentioned in \S\ref{sec:Time2FirstDownload} or as a new search parameter in WebChaser (\url{http://cda.harvard.edu/chaser/}).  Curating these descriptive profiles is currently labor intensive as we have to determine keyword synonyms, but the development of the Unified Astronomy Thesaurus\footnote{\url{http://astrothesaurus.org/}} should aid in this endeavor.

\section{What's Next?}
We have just begun tapping into the information contained within an observation-centric view of the CDA.  We have several investigative projects in mind:
\begin{itemize}
\item Perform a \ssindex{statistical analysis}statistical study of timelines of events happening to data to identify data which are under-utilized and communicate that information to users and to identify data which are currently ``hot'' and pass that information along to \ssindex{observatories!space-based!Chandra}Chandra\ssindex{education!and public outreach} EPO for potential press release topics
\item Provide a meaningful ``users who downloaded the data also downloaded these data'' service
\item Study correlations between downloads and publications to identify successful and not so successful observing programs to inform the proposal review process\end{itemize}

\section{Conclusions}
In summary, taking an observation-centric view of an archive reveals new trending information which can guide development of new services and improvements to current services.  It can also lead to new \ssindex{metrics}metrics to measure the scientific impact of the archive and the mission.  Based on studies thus far we are now investigating: implementation of a subscription service announcing important events related to data; exposing aggregate information to users downloading data; and the creation of descriptive profiles of data.

\acknowledgements 
This work is supported by NASA contract NAS8-03060.
