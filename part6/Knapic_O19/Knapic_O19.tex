
\resetcounters


\markboth{Knapic, Molinaro, and Smareglia}{Full Tolerant Archiving System}

\title{Full Tolerant Archiving System}
\author{Cristina~Knapic, Marco~Molinaro, and Riccardo~Smareglia
\affil{INAF Astronomical Observatory of Trieste, Italy}}

\aindex{Knapic, C.}
\aindex{Molinaro, M.}
\aindex{Smareglia, R.}

\begin{abstract}
The archiving system at the Italian center for Astronomical Archives (IA2) manages
data from external sources like telescopes, observatories, or surveys and handles
them in order to guarantee preservation, dissemination, and reliability, in most
cases in a Virtual Observatory (VO) compliant manner. A metadata model dynamic
constructor and a data archive manager are new concepts aimed at automatizing the
management of different astronomical data sources in a fault tolerant environment.
The goal is a full tolerant archiving system, nevertheless complicated by the presence
of various and time changing data models, file formats (FITS, HDF5, ROOT, PDS, etc.)\footnote{Flexible Image Transport System (FITS), Hierarchical Data Format (HDF), Planetary Data System (PDS)} and metadata content, even inside the same project. To avoid this
unpleasant scenario a novel approach is proposed in order to guarantee data ingestion,
backward compatibility, and information preservation.
\end{abstract}

\section{Introduction: Issues and Constraints}
The Italian center for Astronomical Archives (IA2) infrastructure manages raw data and storage facilities principally
for LBT, TNG and Asiago Observatories; the request to archive and store
calibrated data for European Surveys is a recent request. The evolution of
data formats and the management of calibrated data emphasizes the need for
new tools and lead to a revision of the software architecture towards more
flexible solutions.

\subsection{Issues}
The main issues/desiderata may be summarized as follows:
\begin{itemize}
\item metadata change in content, format type, and header completeness whereas
they should be stable
\item data and metadata may use different formats than FITS; more in detail, 
other formats (ASCII or PH) may be used for calibrated data in addition of FITS
files, astronomical data can be stored in HDF5, PDS, ROOT and other formats
last but not least, the FITS standard evolves
\item difficult software reutilization between different purpose surveys
\item data type, although may affect the choice of database, should not affect
the software or the system architecture
\item effective scalability is a key feature
\item the data pipeline should access the database and the storage in a
transparent way, independent of their location
\item the data policy and the version of the pipeline should be defined in a reversible way
\item the archives may be geographically distributed in several sites (e.g. LBT)
\end{itemize}

\section{Constraints}
The proposed architecture should implement and natively support some new desirable features:
\begin{itemize}
  \item distributed systems
  \item different programming languages
  \item objects deployment shouldn't affect the archiving system performances
\end{itemize}
The Object Request Broker (ORB) architecture, that allows interoperation between
distributed and multilanguage objects, and the Component Container paradigm allow
the desired flexibility.

\subsection{Use Cases}
Some application use cases are presented and the features related to the
Archiving system discussed.
\begin{itemize}
  \item raw data distribution; the critical aspects are data flushing towards
  several geographically distributed sites and data consistency
  \item calibrated and reduced data; all relations between raw, intermediate,
  and final products are stored in the header in both hierarchical and mixed ways;
  consistency is guaranteed by pipeline developers but not guaranteed between
  several pipeline versions
  \item real-time processing of large amounts of data; thus, more than one archiving
  system may be needed to keep up with the different data types: raw data,
  several level intermediate products, science-ready data products, and VO publishing
  data products
\end{itemize}
In addition, archive managers need to be aware of all external information
available in the project proposals, i.e. policy of publication, private
data management, access permissions rules, etc. 

\section{Templates, Models, and Modules}
IA2 is intended to be a gateway between observing facilities and user science,
allowing the latter to easily extract the needed information at every level.
Centralizing the information on how the different topics should work, storing
in a database, and using it to make an archive system container are essential
steps to manage and control the data flux.

\subsection{Modules}
Four main modules have a big role in this schema: data modelization,
metadata ingestion, data storage, or metadata and data retrieval.
The Archiving system could be considered to be a container of different components
that interact with each other. Some components may also be containers of smaller
components. Each container is in charge of coordinating the relations and references
between components, i.e. message exchange, component behaviour, and actions on
files and storage. The top level container is the Archiving System Coordinator
(ASC) which will contain such components as the File Identifier and Notifier, Data
Modelization (DbCreator and DataFlux), and Project Manager (PM). The PM interacts
directly with data and data flux models in order to sort the information and
forward it to the proper destination. Other components will be the  Metadata
Inserter, File Archiver, VO Publisher, Policy Manager, Pipeline User and Updater.
The ASC is in charge of coordinating the actions depending on data and data flux
models, checking allowed or denied actions as well. Data and data flux models should
be created dynamically from data and, possibly, from small configuration files (or
from a pipeline analyzer tool) and stored in a dedicated database
or an XML schema. The advantages and disadvantages of the two methods are
summarized in Figure~1.

\articlefigure[]{part6/Knapic_O19/db_xml.eps}{fig:dbvsxml}{Advanteges and disadvantages of using a
database or XML schema to store data and data flux model information.}

The database solution seems to be more flexible to keep up with model content
evolution and easily tunable; also allows to avoid the additional conversion
of the XML schema into the database schema.

\subsection{Data and  Data Flux Models}
The centralization of information is a key point for the correct configuration of
all the actors involved in the Archiving System. Job parameters have to be abstracted
from code development, and stored in a dedicated database using two tables for Data
Model and Data Flux Model information respectively. The first describes the type and
format of the data that have to be ingested in the Archive, which libraries to use,
technical information on metadata types, hierarchical structures, references to other
files, etc. The second describes how relations between files work, where and when data
should be stored and flushed, how it should be ingested, which pipeline version has
been used, and how to store metadata in the database. Also, the association between
DBMS and data, the policy on data dissemination, ownerships, aliases and so on belong
to the second table.

\articlefigure[]{part6/Knapic_O19/framework.eps}{fig:framework}{Framework Interfaces representation and interactions.}

Metadata archiving may involve different DBMS's and rely on different geographical sites.
The DbCreator is a tool under development that maps metadata keys from new files reading
all metadata available in the data file header and assigning the correct type, structure,
and descriptors (UCDS for the VO). These information are stored as part of the Data Model schema.
A graphical user interface (DataFlux) helps the administrators to correctly configure the
data flux schema in agreement with a pipeline analyzer tool. The Project Manager that
manages all the information stored in the database and
coordinates all the subsystems that are in charge of inserting metadata in the DB, store,
manipulate, publish, or flush data over the system in a transparent way.

\subsection{Conclusion}
A CORBA based framework is a useful tool to hide the complexity of the whole Archiving system; several frameworks are mature and well suited to this development . The proposed approach is a complete redesign of the Archiving system, which is challenging and time consuming, but can dramatically improve the confidence in the whole archiving system.

\acknowledgements The authors would like to thank L. Pivetta for useful conversations and hints, A. Micol, F. Stoehr , P. DiMarcantonio and R. Cirami for suggestions and documentation. Special thanks to F. Pasian for hints and encouragement.
