
\resetcounters

\bibliographystyle{asp2010}

\markboth{Cardiel et al.}{TESELA: Blank Fields for Astronomical Observations}


\title{Searching for Deeper Blank Fields in the Sky with TESELA}
\author{N.~Cardiel,$^1$ F.M.~Jim\'{e}nez-Esteban,$^{2,3,4}$
A.~Cabrera-Lavers,$^{5,6}$ and J.~M.~Alacid$^{2,3}$
\affil{$^1$Departamento de Astrof\'{\i}sica y CC.\ de la Atm\'{o}sfera,
Facultad de CC.\ F\'{\i}sicas, Avenida Complutense s/n, E-28040 Madrid, Spain}
\affil{$^2$Centro de Astrobiolog\'{\i}a (INTA-CSIC), Departamento de
Astrof\'{\i}sica, PO~Box~78, E-28691, Villanueva de la Ca\~{n}ada, Madrid,
Spain}
\affil{$^3$Spanish Virtual Observatory, Spain}
\affil{$^4$Saint Louis University, Madrid Campus, Division of Science and
Engineering, Avenida del Valle~34, E-28003 Madrid, Spain}
\affil{$^5$Instituto de Astrof\'{\i}sica de Canarias, E-38205 La Laguna,
Tenerife, Spain}
\affil{$^6$GTC Project, E-38205 La Laguna, Tenerife, Spain}
}

\aindex{Cardiel, N.}
\aindex{Jim\'{e}nez-Esteban, F. M.}
\aindex{Cabrera-Lavers, A.}
\aindex{Alacid, J. M.}

\begin{abstract}
TESELA is a Virtual Observatory tool which provides a simple interface that allows the user to retrieve a list of Blank Fields ---regions devoid of bright stars down to a given threshold magnitude--- available near a given position in the sky. The initial version of this tool, already presented in ADASS 2011, made use of the Delaunay triangulation to determine a tessellation of the whole celestial sphere using the astrometric and photometric information for the 2.5 million brightest stars provided by the Tycho-2 stellar catalogue. More recently we have used the Delaunay Triangulation to search for deeper blank fields, with a minimum diameter of 10 arcmin and an increasing threshold
magnitude ranging from 15 to 18 in USNO-B R-band. However, instead of tessellating the whole sky at once, and considering the exponentially growing number of stars when moving to fainter stellar magnitudes, we started the searching procedure by exploring in the subsample of the initial Blank Field list derived from the Tycho-2 stellar catalogue containing Blank Fields with size larger than 10 arcmin. Some of these new Deep Blank Fields were tested with the 10.4 m Gran Telescopio Canarias and have been included in the nightly operation, demonstrating to be extremely useful for medium and large size telescopes. The new catalogue is also available through the TESELA interface which, in addition, provides galactic extinction information from the NASA/IPAC Infrared Science Archive. 
\end{abstract}

\section{Previous Work}

In \citet[][hereafter paper~I]{2011MNRAS.417.3061C} \citep[see also contribution in previous ADASS~XXI:][]{2012ASPC..461..173C} we already described a method, based on the use of the Delaunay Triangulation, that helps to determine the availability of blank fields in any region of the celestial sphere. For that purpose, we employed the Tycho-2 stellar catalogue, which contains astrometric and photometric information for the 2.5 million brightest stars in the sky, and is complete up to magnitude $V=11.5$~mag \citep{1997A&A...323L..49P}. In addition, we also presented TESELA, a Virtual Observatory tool that provides a simple interface that allows the user to retrieve the list of blank fields available near a given position in the sky.

\section{Recent Improvements}

More recently \citep[][hereafter paper~II]{2012arXiv1209.4861J} has extended that work by searching for even deeper blank fields, for which it has been necessary to apply the above method to the USNO-B Catalogue \citep{2003AJ....125..984M}, which contains astrometric and photometric information for more than a billion stars and galaxies in the sky. We have used the 10.4~m Gran Telescopio Canarias, located at the Observatorio Roque de los Muchachos (La Palma, Spain) as a test-bed for the validation of the deep blank fields. In particular, we have employed the OSIRIS instrument \citep{2000SPIE.4008..623C} which has an unvignetted field of view of $7.8'\times 7.8'$. 

Finally, we have created a set of 87 deep blank fields, suitable to be used with OSIRIS+GTC, but which is also useful for current medium- and large-size telescopes. An example of one of these deep blank fields is shown in Fig.~\ref{example_DBF}. In this figure we compare the deep blank field with a similar field extracted from the collection of blank fields by Marco Azzaro,\footnote{\url{http://ing.iac.es/~meteodat/blanks.htm}} who created a short list of 38 blank fields void of stars up to 10--16 mag, which for many years has been in use at the Isaac Newton Group of Telescopes.

\begin{figure}
\centerline{%
\includegraphics[angle=-90,width=0.90\textwidth]{part2/Cardiel_P08/P08_f1.eps}}
\caption{Comparison of two images taken with OSIRIS+GTC. They are \mbox{2-s exposure} time images observed with the Sloan $r$ filter, exhibiting a background level of$~\sim 30\,000$~ADUs, of a deep blank field from our catalogue (\emph{left panel}) and a blank field from Azzaro's catalogue (\emph{right panel}). Note that two apparent detections, both in the right-upper part in CCD1 (left side of the detector) and in the left-bottom part in CCD2 (right side of the detector) are dust grain effects on the OSIRIS CCDs, and they do not correspond with any star in the field (they are also visible in the Azzaro's field).}
\label{example_DBF}
\end{figure}

\begin{figure}
\centerline{%
\includegraphics[angle=-90,width=1.00\textwidth]{part2/Cardiel_P08/P08_f2.eps}}
\caption{Distribution in the celestial sphere, using galactic coordinates, of the deep blank fields with $m_{\mbox{\scriptsize th}}$ 15, 16, 17, and 18 mag (upper-left, upper-right, lower-left, and lower-right panel, respectively).}
\label{result_DBF}
\end{figure}


\section{The Procedure to Determine Deep Blank Fields}

Given the huge number of stars in the USNO-B Catalogue, in this work we have followed a methodology that is different from the one adopted in paper~I. Instead of tessellating the whole sky at once, we used the following workflow:
\begin{itemize}
  \item[i)] Select all the blank fields with a size larger than 10~arcmin using
  the catalogue derived from paper~I.
  \item[ii)] Tessellate each of these regions individually, using the positions
  of the USNO-B objects down to a given threshold magnitude
  $m_{\mbox{\scriptsize th}}$.
  \item[iii)] Create the new deep blank field catalogue from the new collection
  of deeper blank fields, selecting those with a size larger than 10~arcmin.
  \item[iv)] Repeat the whole process, increasing 
  $m_{\mbox{\scriptsize th}}$ from 15 to 18, in steps of 1~mag (see
  Fig.~\ref{result_DBF}).
\end{itemize}

It is important to highlight that since the USNO-B Catalogue was built from photographic plates (taken from different surveys) which were scanned and whose sources were extracted in an automatic way from the scanned data, the Catalogue contains defects and artifacts. As explained in paper~II, in this work an important effort has been devoted to reduce the impact of those defects in the determination of the Deep Blank Field Catalogue. Finally, we visually checked all the deep blank fields with $m_{\mbox{\scriptsize th}}=18$~mag in order to confirm their validity before transferring them to the operation staff of the Gran Telescopio Canarias.

\section{Using the Catalogue}

The new Catalogue is accessible through TESELA, the Virtual Observatory tool presented in Paper~I, that was developed by the Spanish Virtual Observatory and which is publicly available at\,\,\url{http://sdc.cab.inta-csic.es/tesela}. TESELA presents the result of the cone search in a table that can be downloaded in CSV format for further use. In addition, the search result can be easily displayed with the help of Aladin,\ooindex{Aladin, ascl:1112.019} providing the user with the full capacity and power of the Virtual Observatory.

Although our main goal in this work was to provide a list of deep blank fields for data calibration, these fields can also be used for other purposes, such as deep pencil-beam extragalactic surveys. For that reason we also provide extinction parameters determined from The Galactic Dust Reddening and Extinction utility of the NASA/IPAC Infrared Science Archive, making use of the data from \citet{1998ApJ...500..525S}.

\acknowledgements This work was partially funded by the Spanish MICINN under the Consolider-Ingenio 2010 Program grant CSD2006-00070: First Science with the GTC. This work was also supported by the Spanish Programa Nacional de Astronom\'{\i}a y Astrof\'{\i}sica under grants AYA2011-24052 and AYA2009-10368, and by AstroMadrid under project CAM~S2009/ESP-1496. This work is based on observations made with the GTC, installed in the Spanish Observatorio del Roque de los Muchachos of the Instituto de Astrof\'{\i}sica de Canarias, in the island of La Palma. This work has made use of Aladin developed at the Centre de Donn\'{e}es Astronomiques de Strasbourg, France. This research has made use of the NASA/IPAC Infrared Science Archive, which is operated by the Jet Propulsion Laboratory, California Institute of Technology, under contract with the National Aeronautics and Space Administration.

\bibliography{editor}

