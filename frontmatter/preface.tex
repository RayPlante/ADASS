% This is the preface.tex LaTeX file
% Copyright 2009, Astronomical Society of the Pacific Conference Series
\markboth{Friedel}{Preface}
\pagestyle{myheadings}
\thispagestyle{plain}
\aindex{Friedel, D. N.}
\section*{\Large\bfseries Preface}
\bigskip
\parindent=2em
\setcounter{section}{1}
\noindent 
This volume of the Astronomical Society of the Pacific (ASP) Conference Series contains papers that were presented at the 22nd annual conference on Astronomical Data Analysis Software and Systems (ADASS XXII), which was held at the I-Hotel and Conference Center, Champaign, Illinois, USA, on 4-8 November, 2012. The ADASS XXII conference was hosted jointly by the University of Illinois and the National Center for Supercomputing Applications (NCSA).

\subsection{Conference Overview}
Champaign County, Illinois, USA, was founded in 1833, encompassing a large swath of fertile farmland, and to this day most of the land in the county is used for agriculture. Due to its proximity to Chicago, St. Louis, and Indianapolis, railroad lines were run through the county in the 1850's, bringing a boon to the Champaign County. In 1867 the land grant university, the University of Illinois, was founded in 1867, in the county seat of Urbana.

The conference started with a tutorial {\it How VO Gets Richer and Helps Solving Fundamental Problems of Extragalactic Astrophysics} on Sunday. This was followed by the opening reception at the I-Hotel and Conference Center. The conference got into full swing on Monday morning and included 10 invited talks, 34 oral presentations, and 66 posters. The topics included: \textit{Extreme-scale Computing}, \textit{Large Ground Based Observatories and Telescope Arrays}, \textit{Community Development}, \textit{Advances in Algorithms}, \textit{Service-oriented Processing and Analysis}, \textit{Enabling Large Catalog Research}, \textit{Enabling Transient Research}, \textit{Visualization and Software for EPO and Public Affairs}, \textit{Computing for the Solar System}, \textit{Astronomy Apps for Handheld Devices}, \textit{Enhancing Science Return with Scientific Archives}, and \textit{Challenges of Modern Archives}.

This year's conference had four Birds of a Feather (BoF) sessions entitled \textit{Bring out your codes! Bring out your codes! (Increasing Software Visibility and Re-use)}, \textit{Application Interoperability with SAMP}, \textit{FITS}, and \textit{Major Instrumentation at Ground-based Observatories: Software systems, Data \& Metadata}. There were 6 floor demonstrations, including a booth for the newly introduced journal \textit{Astronomy \& Computing} from Elsevier; and four focus demonstrations to introduce the attendees to the newest tools of the field. Tours of the National Petascale Computing Facility, the home of Blue Waters, were offered throughout the week to attendees.

\subsection{Organizing Committees and Sponsors}

The Program Organizing Committee (POC) responsible for the content of the ADASS XXII conference consisted of Carlos Gabriel (ESA - ESAC, Spain; Chair), Pascal Ballester (ESO, Germany), Daniel Devost (CFHT), Daniel Durand (NRCC, Canada), Daniel Egret (Observatoire de Paris, France), Tony Krueger (STScI, USA), Deborah Levine (IPAC, USA), Jim Lewis (University of Cambridge, UK), Nuria Lorente (NRAO, USA), François Ochsenbein (CDS, France), Ray Plante (NCSA, USA), Arnold Rots (CfA, USA), Keith Shortridge (AAO, Australia), Betty Stobie (NOAO, USA), Tadafumi Takata (NAO, Japan), Harry Teplitz (IPAC, USA), and Christian Veillet (CFHT, USA).

The Local organizing committee (LOC) consisted of Ray Plante (NCSA, Astronomy; Chair), Mike Freemon (NCSA), Douglas Friedel (Astronomy), Athol Kemball (Astronomy, NCSA), Don Petravick (NCSA), Jean Soliday (NCSA), and Jon Thaler (Physics, Astronomy, and NCSA).
    
The organizers wish to express their deep appreciation to the conference host institutions, sponsors, and supporting organizations. The ADASS~XXII conference was supported by the National Center for Supercomputing Applications (NCSA), NASA Infrared Processing and Analysis Center (IPAC), University of Illinois Department of Astronomy, The National Radio Astronomy Observatory (NRAO), The Australian Astronomical Observatory (AAO), The National Optical Astronomy Observatory (NOAO), The Canada-France-Hawaii Telescope (CFHT), L'observatoire de Paris, Elsevier Publishing, The Smithsonian Astrophysical Observatory, ESA European Space Astronomy Center (ESAC), The Space Telescope Science Institute, and European Southern Observatory. The generous contributions of the host institution, sponsors, and supporting organizations helped to keep costs manageable, and maximized the value of the conference.

\subsection{Further Information}

ADASS XXIII will be held in Waikoloa, Hawaii, USA in September, 2013. Details concerning this and all of the ADASS meetings, as well as electronic versions of many of the proceedings, may be found on the ADASS web site:
%\centering{
\begin{verbatim}
                        http://www.adass.org.
\end{verbatim}
%}

\bigskip

\noindent Douglas N. Friedel\\
Department of Astronomy, University of Illinois\\
Urbana, IL, USA\\
The ADASS XXII Proceedings editor, March 2013
\vfill\eject
\thispagestyle{empty}
