
\documentclass[11pt,twoside]{article}
\usepackage{asp2010}

\resetcounters


\begin{document}

\title{The Astropy Project: A Community Python Library for Astrophysics}
\author{Erik J. Tollerud$^1$, Perry E. Greenfield$^2$, Thomas P. Robitaille$^3$, and the Astropy Developers
\affil{$^1$Astronomy Department, Yale University, P.O. Box 208101, New Haven, CT 06510, USA}
\affil{$^2$Space Telescope Science Institute, 3700 San Martin Dr., Baltimore, MD 21218, USA}
\affil{$^3$Max Planck Institute for Astronomy, K\"onigstuhl 17, Heidelberg 69117, Germany}}

\begin{abstract}
I introduce and describe progress on Astropy, a large, community effort to provide common astronomy/astrophysics utilities and promote reuse of software. It is based on a model of a collaborative open-source core package (currently under heavy development) and independent but affiliated packages contributed by individuals or organizations. I  describe some of the features in the current core package, the organizational structure of the community, and the direction the project is headed in the near future.\end{abstract}

\section{Introduction}
Python is fast becoming one of the main programming languages used by research astronomers. It's multi-paradigm nature, large developer community, well-developed libraries for scientific work, and free open source licensing make it well matched to the needs of astronomy and astrophysics work. However, it faces a major problem of fragmentation; much astronomy functionality is duplicated by many different libraries by many different authors, all of which are mutually incompatible.  This led to the creation (in fall 2011) of Astropy (\url{http://www.astropy.org}), an effort to foster intercompatibility and software reuse in the python astronomy community. Astropy consists of two main components: the  community effort to connect specific-use packages, and the {\it astropy} core package, a python package to provide common utilities and infrastructure to support astronomy in Python.

\section{The Astropy Project}
The Astropy Project focuses on promoting compatibility between python packages in an astronomy context.  A major aspect of this is simply agreeing that this is a problem that needs to be solved, and attempting to work towards compatibility.  To this end, Astropy includes the concept of ``affiliated packages'', python packages that agree with the aims of Astropy and wish to be listed a such on the Astropy web site (\url{http://affiliated.astropy.org}).  Astropy also provides a template (\url{http://github.com/astropy/package-template}) that simplifies packaging and documenting astronomy packages, allowing the developer/scientist to focus on just the functionality rather than learning more arcane aspects of Python packaging.  In an upcoming release, we expect to also implement an install tool that will allow automatic installation of affiliated packages, further simplifying the use of Python in astronomy.

\section{The {\it astropy} Core Package} 
The majority of the work for Astropy has focused on creation of the {\it astropy} core package (\url{https://github.com/astropy/astropy}).  The package is still under heavy development, but already includes a variety of functionality at various levels of completion.  This includes as an astronomy-appropriate representation of time (with nanosecond accuracy over a Hubble Time), classes for storing data with propagation of uncertainties, utilities for working with various cosmologies, a full-featured representation of units and astronomically-relevant constants, a {\it Table} class with readers for common astronomy formats, pythonic Virtual Observatory (VO) tools, and classes for celestial coordinates.  For up-to-date information on what has been implemented, see the core package documentation (\url{http://docs.astropy.org}).

Development of the core package has occurred entirely in Github, a free (for open source) code hosting service which leverages the {\it git} version control system to provide a range of powerful collaboration tools.  Our development model operates around ``pull requests,'' a scheme where anyone can contribute code, but it is first vetted by the entire community via the Github web site before being merged by the authorized repository coordinators. This scheme has allowed the community to produce some 80,000 lines of code in 16 months, all without any direct funding; it is all contributed by 20+  science software support developers and scientists who intend to actually use the code in their day-to-day work.  This model has thus far been quite successful, and we hope it can serve as a model as well for helping to enable transparency and coordination of more domain-specific codes that wish to sign on as affiliated packages.

\section{Conclusion}
Astropy is continuing its development efforts, and the community is continuing to grow.  We expect a beta-level (0.2) core package release in the coming weeks, with a first ``public'' release following after.  We will continue to expand on the core package, while also developing some major functionality in affiliated packages (e.g., \url{http://github.com/astropy/specutils} and \url{http://github.com/astropy/photutils}, packages focused on spectroscopy and photometry algorithms, respectively). Thus we hope to finally turn the Python Astronomy community around, stopping the harmful proliferation of packages with near-identical but incompatible functionality.


\end{document}
