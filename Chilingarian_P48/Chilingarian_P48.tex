% This is the aspauthor.tex LaTeX file
% Copyright 2010, Astronomical Society of the Pacific Conference Series

\documentclass[11pt,twoside]{article}
\usepackage{asp2010}

\resetcounters

%\bibliographystyle{asp2010}

\markboth{Author1, Author2, and Author3}{Author's Final Checklist}

\begin{document}

\title{Data reduction pipeline for the MMT Magellan Infrared Spectrograph
}
\author{Igor~Chilingarian$^1,2$, 
Warren~Brown$^1$, Daniel~Fabricant$^1$, 
Brian~McLeod$^1$, John~Roll$^1$,
and Andrew~Szentgyorgyi$^1$
\affil{$^1$Smithsonian Astrophysical Observatory, 60 Garden Street,
Cambridge, MA 02138, USA}
\affil{$^2$Sternberg Astronomical Institute, Moscow State University, 
13 Universitetsky prospect, Moscow 119992, Russia}}

\begin{abstract}
We describe a new spectroscopic data pipeline for the MMT/Magellan Infrared
Spectrograph (MMIRS).  MMIRS can operate at f/5 focii of either the MMT or
the Magellan Clay 6.5m telescopes.  MMIRS addresses a 4 by 7 arcminute field
of view for multi-object spectroscopy and is equipped with a 2Kx2K HAWAII-2
array.  The pipeline handles data obtained in multi-slit and single-object
long slit modes.  All of the pipeline blocks are implemented in IDL with the
exception that up-the-ramp fitting of a sequence of raw frames is performed
in a C++ routine.  Up-the-ramp fitting allows us to reject cosmic ray events
and correct non-linearity and saturated pixels.  The most sophisticated
algorithm is sky subtraction, where we take a hybrid approach that uses both
the ``classical'' dithered difference image approach and a modified version
of the Kelson (2003) sky subtraction technique.  Our tests show that the
pipeline gets close to the Poisson-limited sky subtraction quality.  The
final data products include flux-calibrated 2D and extracted 1D spectra
corrected for telluric absorption.  Data files are made available as
classical ``stacked spectra'' and in the Euro3D-FITS format with Virtual
Observatory compliant metadata.  We will describe the principal components
of the pipeline and present examples of fully reduced scientific data.
\end{abstract}

\section{Introduction}

\section{Pipeline design and implementation}

%\subsection{The Author Checklist}

\begin{itemize}
\item  Data pre-processing and primary reduction 
\item  Flat-fielding and 2D slit extraction
\item  Wavelength calibration
\item  Sky subtraction
\item  Linearization and rectification
\item  Telluric correction
\end{itemize}

The data post-processing is run after the pipeline processing and it
includes the following steps
\begin{itemize}
\item Co-adding multiple observing blocks
\item 1D-extraction
\item Output format conversion into Euro-3D FITS (optional)
\end{itemize}

\section{Bulk reduction of MMIRS data}
We are now reducing the entire volume of spectral data collected with MMIRS
since 2010 at the Telescope Data Center, Smithsonian Astrophysical
Observatory. The data will be provided to PIs in a fully reduced form: 1D-
and 2D, telluric absorption corrected, flux calibrated (relative). Our goal
is to facilitate the data publication and increase the scientific impact of
MMIRS.

\acknowledgements 
IC acknowledges the feedback from Yu.~Beletsky (Las Campanas Observatory) on
the implementation of the pipeline.

%\bibliography{aspauthor}

\end{document}
