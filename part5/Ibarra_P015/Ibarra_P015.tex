
\resetcounters

\bibliographystyle{asp2010}

\markboth{Ibarra  et al.}{\ssindex{observatories!space-based!XMM-Newton}XMM-{\em Newton} Mobile Web Application}

\title{\ssindex{observatories!space-based!XMM-Newton}XMM-Newton Mobile Web Application}
\author{Ibarra, A., Kennedy, M., Rodr\'iguez, P., Hern\'andez, C., Saxton, R. and Gabriel, C.}
\affil{XMM-Newton SOC, European Space Astronomy Centre (ESAC) / ESA, Madrid, Spain}

\begin{abstract}
We present the first \ssindex{observatories!space-based!XMM-Newton}XMM-{\em Newton} web mobile application, coded using new web technologies such as HTML5, the \ssindex{computing!mobile!framework!jQuery}\ssindex{libraries!jQuery}jQuery mobile framework, and D3 \ssindex{computer languages!JavaScript}JavaScript data-driven library.

This new web mobile\ssindex{web!applications} application focuses on re-formatted contents extracted directly from the \ssindex{observatories!space-based!XMM-Newton}XMM-Newton web, optimizing the contents for mobile devices. 

The main goals of this development were to reach all kind of handheld devices and operating systems, while minimizing software maintenance. The application therefore has been developed as a web mobile implementation rather than a more costly native\ssindex{web!applications} application. New functionality will be added regularly.

\end{abstract}

\section{Introduction}

\ssindex{observatories!space-based!XMM-Newton}XMM-{\em Newton} (~\cite{Jansen}) as a project, has as it's fundamental vehicle for mission information, a dedicated website. There are several reasons behind the development of an \ssindex{observatories!space-based!XMM-Newton}XMM-{\em Newton,} mobile web\ssindex{web!applications} application. The most important one is that more and more \ssindex{observatories!space-based!XMM-Newton}XMM-{\em Newton} observers try to reach information contained in the standard web pages through handheld devices, but a large part of the information contained there is far from being optimized for this medium. Another reason is the quick growth of the mobile device market in the last few years.  Both smartphones and tablets have emerged abruptly not only into the information technology world, but also into the scientific world. Therefore, the XMM-{\em Newton} web mobile\ssindex{web!applications} application is a step forward in the modernization of the Internet services provided by the project.
 
There are two main issues when a standard web portal is accessed from a mobile device:
\begin{itemize}
\item Limited screen size.
\item Too much content.
\end{itemize}

It is of no use to display in a handheld device a full web page with large quantities of text, links and images. Mobile device users are continuously forced to resize the content to read the information. The small screen size of mobile devices implies a simple or minimalist user interface and content tailoring.

We have therefore developed this XMM-{\em Newton} web mobile\ssindex{web!applications} application taking all this into account and trying to minimize the code maintenance. The contents are automatically extracted from the official \ssindex{observatories!space-based!XMM-Newton}XMM-{\em Newton} project website. We have reduced the contents, showing only the most valuable information. It is easy to surf between different pages. It has been coded using new web technologies such as HTML5 and the \ssindex{computing!mobile!framework!jQuery}\ssindex{libraries!jQuery}jQuery mobile framework\footnote{http://jquerymobile.com}.

\section{The \ssindex{observatories!space-based!XMM-Newton}XMM-{\em Newton} Web Mobile\ssindex{web!applications} Application}

The XMM-{\em Newton} web mobile\ssindex{web!applications} application has been optimized to be displayed on mobile devices. The navigation within the mobile app is controlled by \ssindex{computing!mobile!framework!jQuery}\ssindex{libraries!jQuery}jQuery Mobile framework through \ssindex{software!tools!AJAX}ajax, giving the web app a look and feel similar to a native\ssindex{web!applications} application. In Figure 1 we show the concept of the \ssindex{observatories!space-based!XMM-Newton}XMM-{\em Newton} web mobile\ssindex{web!applications} application.

The \ssindex{observatories!space-based!XMM-Newton}XMM-{\em Newton} web mobile\ssindex{web!applications} application is structured as following.
\begin{itemize}
\item \ssindex{observatories!space-based!XMM-Newton}XMM-{\em Newton} satellite image.
\item Daily satellite events. Here we show live information such as: current revolution, current observation, mission operation duration and number of refereed papers based on 
\ssindex{observatories!space-based!XMM-Newton}XMM-{\em Newton} data.
\item The main body. Several elements, described below, constitute the bulk of the web mobile\ssindex{web!applications} application contents.
\end{itemize}

{\bf Highlights and Events:} this section, further structured into Science News and Highlights, is based on all the extracted news, highlights, and events present in the the standard \ssindex{observatories!space-based!XMM-Newton}XMM-{\em Newton} web. Considerable reformatting is performed for optimal display in mobile devices.

{\bf Live:} under this section the \ssindex{observatories!space-based!XMM-Newton}XMM-{\em Newton} real-time information is available, offering the first version of a visibility checker tool and a current position locator.

In this section we have concentrated most of the project effort, dedicated to new development. Here we have used HTML5 technology to interact with onboard mobile device features such as GPS. We have also developed a suite of astronomical \ssindex{computer languages!JavaScript}JavaScript functions to: 
\begin{itemize}
\item convert coordinates for different reference systems,
\item calculate the \ssindex{observatories!space-based!XMM-Newton}XMM-{\em Newton} attitude at any time using Two Line Elements,
\item calculate apparent positions of \ssindex{astronomy!solar system}Solar System objects with respect to the \ssindex{observatories!space-based!XMM-Newton}XMM-{\em Newton} position,
\item include Virtual Observatory interfaces to resolve target names.
\end{itemize}

With all these functions, we have been able to create a new dynamic graphical version of a target visibility checker. Embedded in a thin-layer web based interface, we calculate target visibility periods for any \ssindex{observatories!space-based!XMM-Newton}XMM-{\em Newton} revolution. This tool also shows the variations of the avoidance regions within the selected \ssindex{observatories!space-based!XMM-Newton}XMM-{\em Newton} orbit. The forms have been created using \ssindex{computing!mobile!framework!jQuery}\ssindex{libraries!jQuery}jQuery mobile framework and the data \ssindex{visualization}visualization has been developed using the D3 \ssindex{computer languages!JavaScript}JavaScript library\footnote{http://d3js.org}. D3 is a library for manipulating documents based on data. 

Under the {\bf Live} section we have also developed the current \ssindex{observatories!space-based!XMM-Newton}XMM-{\em Newton} position locator. This is a \ssindex{visualization!3D}3D visualization of the \ssindex{observatories!space-based!XMM-Newton}XMM-{\em Newton} orbit with the projection onto the Earth surface of the \ssindex{observatories!space-based!XMM-Newton}XMM-{\em Newton} orbital track. An animation of the complete current \ssindex{observatories!space-based!XMM-Newton}XMM-{\em Newton} revolution is also available.

We plan further developments in this section, including usage of the device gyro for generating a planetarium, highlighting the \ssindex{observatories!space-based!XMM-Newton}XMM-{\em Newton} observations.

{\bf Mission overview:} in this section basic information about the satellite and the mission concept can be found. We also link here the most important \ssindex{observatories!space-based!XMM-Newton}XMM-{\em Newton} documents for further reading. 

\begin{figure}[h]
\epsscale{1.2}
\plotone{part5/Ibarra_P015/P015_1.eps}
\caption{ The new XMM-{\em Newton} web mobile\ssindex{web!applications} application automatically extracts contents from the standard web pages. Then the information is formatted to be displayed in handled devices.} \label{P061-fig-1}
\end{figure}


{\bf Calendar:} section where important \ssindex{observatories!space-based!XMM-Newton}XMM-{\em Newton} events are recorded in a calendar view. Clicking on any of the events, the user can automatically import any event to his/her agenda. As of today, it is only possible to add events to the Google calendar.

{\bf Image gallery:} here we show a collection of astronomical images and spectra taken with \ssindex{observatories!space-based!XMM-Newton}XMM-{\em Newton} X-ray and optical instruments along with other \ssindex{observatories!space-based!XMM-Newton}XMM-{\em Newton} related images. This gallery of the official \ssindex{observatories!space-based!XMM-Newton}XMM-{\em Newton} website is a good example of web pages which are difficult to handle with a mobile device. Avoiding duplication of code, we have \ssindex{software!reuse}reused the whole image gallery server infrastructure to extract the information from the standard web pages.  Information and forms have been designed and formatted to be displayed properly in a mobile device, using the \ssindex{computing!mobile!framework!jQuery}\ssindex{libraries!jQuery}jQuery mobile framework.

The last item in the main web page is the {\bf Full web page} link that redirect users to the official \ssindex{observatories!space-based!XMM-Newton}XMM-{\em Newton} web pages, just in case users would like to have access to them.

Making use of HTML5 technology, the XMM-{\em Newton} web mobile\ssindex{web!applications} application can be installed in your desktop screen and will appear on it with an icon similar to any native application. Once the web mobile app runs through this icon, the web browser navigation and URL menu disappear, giving the application a look and feel similar to a native application.

\section{Web Mobile vs Native App}

The two ways to convey information and services to a handheld device are web mobiles and native applications. Both ways present pros and cons, which in our opinion have to be ranked according to the actual needs of the individual project. The web mobile pros are simplicity, cheaper development and maintenance costs, simple access (single URL), and platform independence. The strengths of native applications lay in the possibility of specific developments for a particular device and OS, therefore taking advantage of the ability to leverage device specific hardware and software. Also the fact that these can run offline is an advantage in front of the web mobile, which needs a connection to work (although HTML5 offers the possibility of off-line usage).

Cons of the web mobile are mainly the network connectivity mentioned above and the limited capabilities and limited access to the onboard hardware. However the HTML5 standards are evolving quickly towards reducing these limitations. Fundamental for our decision against developing a native application were the associated higher costs in development and maintenance. In this way, we have also avoided forcing users to download the app from a central repository.

Nevertheless, we see advantages and disadvantages by both technologies. The final decision to build a native app or a web mobile will depend ultimately on several factors such as: technical requirements, customer profiles or budget constraints.

\bibliography{editor}
