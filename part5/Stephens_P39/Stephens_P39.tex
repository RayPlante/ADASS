
\resetcounters

\markboth{Stephens}{A Mobile Data Application for the Fermi Mission}

\title{A Mobile Data Application for the Fermi Mission}
\author{Thomas~E.~Stephens$^1$}
\affil{$^1$Wyle ST\&E/Fermi Science Support Center, 977S 1830E, Spanish Fork, UT 84660}

\aindex{Stephens, T. E.}

\begin{abstract}
With the ever increasing use of smartphones and tablets among scientists and the world at large, it becomes increasingly important for projects and missions to have mobile friendly access to their data. This access could come in the form of mobile friendly websites and/or native mobile applications that allow the users to explore or access the data. The Fermi Gamma-ray Space Telescope Mission has begun work along the latter path.

In this poster I present the initial version of the Fermi Mobile Data Portal, a native application for both Android and iOS devices that allows access to various high level public data products from the Fermi Science Support Center (FSSC), the Gamma-ray Coordinate Network (GCN), and other sources. While network access is required to download data, most of the data served by the app are stored locally and are available even when a network connection is not available. This poster discusses the application's features as well as the development experience and lessons learned so far along the way. 
\end{abstract}

\section{Application Overview}
The Fermi Mobile Data Portal application provides mobile access to some of the high level products  that are provided by the Fermi Science Support Center's web site as well as other sources.  The application is developed for both the Android and iOS platforms and provides a native UI experience on both platforms.  The data products provided are intended to provide overview data relevant to the Fermi mission.  Each data product is briefly described in the following sections.

\subsection{Monitored Sources}
\articlefigure{part5/Stephens_P39/P39-img1.eps}{fig1}{Screenshots of various UI elements of the application. a) A light curve plot showing the data points, upper limits and buttons to allow the selection of the energy range data to display. b) GCN Notice list c) Astronomer Telegram list}

One of the products supplied via the FSSC website are light curves, both daily and weekly, of all sources routinely monitored by the Fermi Large Area Telescope (LAT).  The data are provided for three flux ranges and for each object from the time it first reached the monitoring threshold through the present.

The app checks for new data each time it is launched and downloads any new data from the FSSC servers.  All downloaded data is always available whether the device is on-line or not.  The user has the option to view weekly or daily binned light curves in one of the three energy ranges for each object.  Data are plotted either as flux points with error bars or upper limits. (See figure 1a.)

\subsection{GCN Notices}
Fermi uses the GCN to send out notices about detected gamma-ray bursts (GRBs) and other triggers that are detected by both the LAT and the Gamma-ray Burst Monitor (GBM) instruments onboard.  The app provides an interface to view all recent GCN notices sent out by Fermi.  As these notices are time sensitive, only notices for the last thirty days are stored on the device.

Updated each time the app loads, the notices are displayed in reverse chronological order (newest first, see figure 1b).  The overview list gives the date, position, and type of notices.  Clicking on an entry takes you to a detailed view of the data from all notices for that particular trigger.  The data is aggregated by filling in field from the best (newest) trigger and working backwards.  A link to the GCN website with the full details of each individual trigger is also provided on the detailed data view.

\subsection{Astronomer's Telegrams}
Additional high level information is supplied to the astronomical community from the Fermi mission via Astronomer's Telegrams (Atels).  These are typically announcements of interesting objects that have had a more in depth analysis by Fermi scientists.  The application provides a list of all Atels that have been issues since the app was installed.  The Atel list view gives the title, date, author, and a short summary of the Atel (see figure 1c).  Selecting an Atel takes the user to the Astronomer's Telegram website to view the entire Atel.

\subsection{Fermi Sky Blog}
The LAT team maintains a blog that is updated weekly with significant events that occurred in the gamma-ray sky as observed by Fermi at http://fermisky.blogspot.com.  The app provides a link to this mobile optimized site so that users can browse the blog and see that has been happening.

\section{Backend Services}
The data for the application are provided via a variety of backend services running either on the FSSC servers or on the servers for the original data sources.  For example, the Fermi Sky Blog functionality is provided simply by linking to the blog's website.

Data for the Astronomer's Telegrams is obtained by parsing the Astronomer's Telegram RSS feed each time the application is loaded.  The app looks for Fermi related telegrams (containing the word Fermi in either the title or description) and if found, adds these to the internal data set.

The GCN data is provided by a web service on the FSSC servers.  The FSSC has a VOEvent client that runs constantly, listening to the VOEvent stream published by the GCN.  When new GCN Notices are published to this stream, they are captured, parsed, and stored in a database at the FSSC.  When the application loads, it sends a request to the FSSC servers that queries this database and returns any new GCN Notices since the app last updated.  These data are then stored locally on the device and available even when the device is off-line.

A similar model is used for the Monitored Sources data.  In this case the data are updated daily at the FSSC as new data are received each morning from the LAT team.  These data are used to both generate plots on the FSSC website and populate the database used by the mobile app.  When the app loads, it sends a new data request to the FSSC servers and receives any data that as arrived since its last update.  Like the GCN Notice data, the light curve data is stored locally on the device and is available even when the device is not on a network.

\section{Development Framework}
An initial prototype of the app was developed as a native Android application in Java.  This prototype included the Monitored Sources, Atel, and Fermi Sky Blog functionality as a proof of concept.  Once I had an understanding of what the app could do, I shifted development to a cross device development framework, Titanium Studio.  

There are both advantages and disadvantages of working in a cross device framework.  On the positive side, Titanium Studio allows for simultaneous development of both iOS and Android applications in a single language (JavaScript) from a single code base instead of two distinct code bases in two different languages (Objective C and Java respectively).  While there is some unique code relative to the two platforms, it is only a small amount of code and related to the variations in the UI (i.e. the title bar buttons on iOS vs. use of hardware buttons on Android).  This was a big enough win in terms of maintaining consistency between the different versions of the application and only have to code in a single language to the point that it has so far outweighed the shortfalls of the framework.

On the downside, you have to sacrifice some things when working in a cross development framework.  One of the biggest frustrations I found for this app was the lack of native canvas support in Titanium.  While you have native functions to draw to the screen in both Java and Objective C, this is not supported in the current version of Titanium Studio.  Thus I had to work around this limitation in order to draw the light curves by rendering the plots as an HTML 5 canvas and displaying it in a web view.  This work around results in a small performance hit on both platforms.

Another sacrifice is application responsiveness.  The Android app developed in Titanium is noticeably sluggish compared the the native prototype I developed.  This is due to the fact that the final application is interpreted JavaScript instead of native Java (or Objective C).  This sluggishness is not really noticeable on iOS devices but is close enough to the tipping point on Android that I'm considering giving up the single code base benefits in favor of an improved user experience.

A final area of friction in working with Titanium Studio was the poorer integration with the Android Emulator as compared to the iOS simulator.  This may have been partially due to the fact that development had to be done on a Mac due to restrictions on iOS development imposed by Apple, Inc. but is partially due to the working of Titanium Studio itself.  The interface to the emulator was so slow that it was actually faster to test the application directly on my Android phone.

\section{Future Work}
There are several areas in which the application can be improved and extended. 
\begin{itemize}
\item{Tablet interface -- Currently the application will run on a tablet but it is just a scaled up version of the phone interface.  A tablet sensible layout in planned to take advantage of the larger screen sizes.}
\item{Improve Atel data feed -- Currently, if you go a long time without starting the app, it is possible that some of the recent Fermi Atels will no longer be in the RSS feed and you'll miss them.  I plan to rework the Atel backend to be more like the GCN Notices data feed.}
\item{Push Notifications -- Enable push notifications when new GCN Notices and Atels become available.}
\item{Favorites -- Allow users to mark specific monitored sources and pin them to the top of the sources list (which is currently ordered by RA). This would allow quicker access to sources the user is interested in, improving the user experience.}
\item{Additional data sources -- There are other data sources such as GCN circulars and optical data for the monitored sources that could be folded into the application.}
\end{itemize}

\section{Conclusions}
The Fermi Mobile Data Portal application provides a mobile friendly interface to several of the high level data products provided by the Fermi mission.  By leveraging a cross platform development, framework I was able to provide a consistent user experience across both Android and iOS devices while also providing native UI interfaces.  The application provides a set of core data products and can be expanded to provide additional products as they are identified as desirable by the user community.
