\documentclass[11pt,twoside]{article}
\usepackage{asp2010}

\resetcounters

\bibliographystyle{asp2010}

\markboth{M.~Fitzpatrick, O.~Laurino, L.~Paioro and M.~B.~Taylor}
         {Application Interoperability with SAMP}

\begin{document}

\title{Application Interoperability with SAMP}
\author{M.~Fitzpatrick$^1$, O.~Laurino$^2$, L.~Paioro$^3$ and M.~B.~Taylor$^4$}
\affil{$^1$National Optical Astronomy Observatory, Tucson AZ, U.S.A.}
\affil{$^2$Harvard-Smithsonian Center for Astrophysics, 60 Garden Street,
           Cambridge, MA 02138, U.S.A.}
\affil{$^3$National Institute for Astrophysics, IASF, Milano, Italy}
\affil{$^4$H.~H.~Wills Physics Laboratory, University of Bristol, U.K.}

\begin{abstract}
The Simple Applications Messaging Protocol (SAMP) is a Virtual Observatory
(VO) specification that enables astronomy software tools to exchange
control information and data, allowing desktop applications to work as
an integrated suite of tools rather than requiring complex functionality
to be (redundantly) built into tools individually. In addition, SAMP
allows new workflows to be created for the science user that leverages
the advantages of each tool (e.g.\ visualization of tables or images,
analysis, etc), greatly reducing the time needed to switch between
applications and tasks.
We present here a short introduction to the protocol itself,
a survey of some toolkits for application authors who wish to
introduce SAMP functionality into their tools,
and some examples of real-world usage.
\end{abstract}

\section{Introduction}

The Simple Applications Messaging Protocol provides
platform-independent messaging between tools based either on
the desktop or the browser.
The messaging architecture is based on a free-standing {\em hub\/} process
that provides message brokering to external clients,
providing the illusion of direct client-client interaction
with the convenience of a single communication point.
Messaging is built around the publish/subscribe model in which
each client flags those message types (MTypes in the SAMP terminology),
if any, it is willing to receive.
SAMP is defined by the SAMP standard \citep{samp1.3}, and
the design principles are discussed further in \citet{taylor2011}.

Section~\ref{sec:B2_libs} below lists a number of language-specific
libraries and toolkits available that can help developers to work
with SAMP and incorporate SAMP usage into their applications.
Section~\ref{sec:B2_usage} gives some diverse examples of
how these tools can be used to deliver improved science workflows.

\section{Toolkits and Implementations}
\label{sec:B2_libs}

A number of toolkits and libraries for use with SAMP are listed at
\url{http://www.ivoa.net/samp}; this section describes some of them.

\subsection{JSAMP}

JSAMP is a hub implementation, toolkit and client library written in Java.
As well as a
basic interface to the SAMP Hub and Client APIs,
JSAMP provides easy-to-use GUI components
for integrating SAMP use into interactive Java applications.
JSAMP also incorporates a number of diagnostic tools,
including extensive message logging capabilities and
a graphical hub view that shows the details of currently registered
clients and recently transmitted messages.

\subsection{SAMPy}

SAMPy is a Python toolkit and hub implementation.
SAMPy will be part of 
astropy \citep{O30_adassxxii}. Astropy is
a common effort to develop a single Python core
package for astronomy, involving about 100 developers from around the world,
and is available from PyPI.\footnote{\url{http://pypi.python.org/pypi/sampy/}}.

To start SAMPy's hub implementation it is sufficient to start the sampy
executable, installed with the main distribution. As with other libraries,
registering a client requires the instantiation of the client itself, its
connection to the hub, and the binding of a Python function to specific
MTypes. The function is used as a callback when a message with a bound
MType is sent to the client. SAMPy also offers means to discover clients
connected to the hub and send messages to them.

\subsection{Libsamp}

{\em Libsamp\/} is a library within the {\em VOClient\/} package
(in development)
that provides a C-language interface to enable applications to send and
receive SAMP messages.  The API is designed to simplify and hide the details
of the SAMP protocol from the application writer, providing standard methods
to initialize the interface, declare metadata, post message callbacks,
send specific message MTypes, and startup/shutdown messaging.
Details of the hub discovery and registration, as well as handling of
specific messaging patterns, are handled internally and are also fully
accessible through low-level procedures.  These low-level procedures
similarly allow application developers fine-grained control over the
formatting of outgoing messages or parsing of return values.  Because the
interface is implemented in C, bindings for many other languages can be
easily generated automatically using SWIG,\footnote{\url{http://www.swig.org/}}
or custom interfaces can be hand-generated to provide a more language-specific
interface (e.g.,\ one that uses idioms of the language as in a {\em Pythonic\/}
interface, or a binding for languages not supported by SWIG such as {\em SPP\/}
used in IRAF).

\subsection{sampjs}

Sampjs is a small JavaScript library for use by browser-based
applications that performs SAMP messaging using the Web Profile.
Sampjs makes it easy to add SAMP messaging capabilities to web pages by
adding a few lines of JavaScript, as well as allowing the possibility
of fully SAMP-integrated web applications.

\section{Usage Examples}
\label{sec:B2_usage}

\subsection{Integration of GUI tools}

A common usage scenario for SAMP is integrated use of
multiple interactive desktop applications specialised for different data types.
SAMP's data exchange enables them to work together as a single
integrated suite with the union of the capabilites of the component tools.
An example workflow involving TOPCAT (a table analysis tool)
and Aladin (a sky image analysis tool) might be:

1.\ display an image of a region of sky in Aladin

2.\ acquire a catalogue in Aladin with multi-band photometry
    corresponding to sources visible in the region

3.\ overplot the catalogue positions on the sky imagery

4.\ send the catalogue to TOPCAT using SAMP

5.\ plot a color-magnitude diagram in TOPCAT

6.\ identify a sub-population in TOPCAT from the color-magnitude plot

7.\ send the sub-population referencing the original catalogue
    back to Aladin using SAMP

8.\ Aladin displays the sub-population sources in a way which
    distinguishes them visually from the others

The SAMP send operations are typically initiated by the user
simply hitting an appropriate ``Send'' button in the GUI.
The loose semantics of the messages typically exchanged by SAMP
applications mean that this workflow could work in just the same
way if different image- and/or table-analysis tools were used.

\subsection{SAMP as a lightweight remote procedure call protocol}

Some projects have used SAMP as a lightweight protocol for remote procedure
calls. The advantage of this approach is that robust off the shelf SAMP
libraries can be used to build a thin layer on top of existing applications
in different programming languages in order to make them communicate. Such
a private interface can also be exercised by different clients than those
that were targeted originally.

Iris \citep{2012ASPC..461..893D}, the Virtual Astronomical Observatory
tool for the analysis of Spectral Energy distributions, for example,
employed SAMP to make the connection between a Java application for
spectral analysis (Specview, by STScI, \citet{2000ASPC..216...79B}) and a
Python fitting engine (Sherpa, by SAO, \citet{2007ASPC..376..543D}). The
design is straightforward and requires the specification of methods,
identified by MTypes; arguments, in the form of SAMP dictionaries;
exceptions, serialized as SAMP messages and as such propagated from one
programming language to the other.

While Python offers natural means for deserializing dictionaries in the
form of instances, a specific library was developed in Java for
(de)serializing Java interfaces as SAMP messages. This makes the
implementation of a simple inter-language remote API very straightforward
and lightweight.

\subsection{SAMP from the command line}

The {\em Libsamp\/} library was used to fully SAMP-enable
the IRAF Command Language
(CL) as well as to build a command-line tool (called {\em vosamp\/}) to
allow scripts to send (and optionally receive) messages.  In both cases,
a simplified command interface further hides the details of the SAMP
protocol from the user.  For example, a {\em load\/} command takes as a single
argument the name of a local file or a URL. The IRAF CL or {\em vosamp\/}
task determine whether this file is a FITS image or a VOTable and format
the appropriate message type or supply additional arguments as needed.
Options exist to send directed messages to specific applications or send
messages using a particular message pattern.

For the {\em vosamp\/} command-line tool, the overhead of connecting to the
Hub with each command in scripts is avoided by having the task run in
the background as a persistent proxy.  On the first invocation the task
registers with the Hub and then forks itself to run in the background
while remaining connected to the messaging session.  Subsequent commands
from the terminal or script are sent to this background proxy via IPC
for execution, allowing a script to process many commands using a single
application registration.  This capability means that any scripting language
(e.g.\ Python, Perl, IDL, Bourne or C-shell, etc) that can execute a host
command can send SAMP messages without requiring detailed knowledge of the
protocol by the script writer.  In cases where tighter integration
with the language is required, bindings can be generated as needed.

\subsection{SAMP from archive query web pages}

Many data centers provide web-based access to their data holdings
along the lines of a form which a user fills in, resulting in a
web page listing one or more data products such as images, spectra
or catalogues, with the expectation that users will download these
files to disk and then load them into a suitable viewer application.

Using the Web Profile introduced in SAMP 1.3 and a JavaScript library
like sampjs, it is very easy (10--20 lines of JavaScript)
to associate a button with each such link that sends the relevant
file (in fact, its URL)
directly to whatever suitable SAMP-aware viewer the user
happens to be running, if any.
It is straightforward to arrange for such buttons to be hidden
in the absence of a SAMP hub, so non-SAMP-aware users do not
experience unavailable functionality as increased clutter.

\bibliography{B2}

\end{document}

